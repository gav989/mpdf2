\documentclass[13.5pt, varwidth=true]{beamer} 
\usepackage[scaled=1]{helvet}
\usepackage[T1]{fontenc}
\usepackage{overpic}
\usepackage{sansmath}
\usepackage{multicol}
\usepackage{gnuplottex}
\sansmath
\renewcommand\familydefault{\sfdefault}
%\DeclareMathSizes{11}{19}{13}{9}   % For size 11 text
\usefonttheme{professionalfonts}
\usepackage{colortbl}
\usepackage{xcolor}
\usepackage{booktabs}
\usepackage{fancybox}
\usepackage{relsize}
\usepackage{siunitx}
\sisetup{detect-all}
\definecolor{que1}{HTML}{FFC5FA}
\definecolor{que2}{HTML}{C6D7FD}
\definecolor{bgre}{HTML}{DCDCDC}
\definecolor{tgre}{HTML}{BCBCBC}
\definecolor{ared}{HTML}{850000}
\setbeamercolor{background canvas}{bg=bgre}
\setbeamercolor{que1}{fg=black,bg=que1}
\setbeamercolor{que2}{fg=black,bg=que2}
\usetheme{CambridgeUS}
\usecolortheme{beetle}
\setbeamertemplate{navigation symbols}{}
\setbeamertemplate{footline}{\centerline{\textcolor{tgre}{This work is licensed under a Creative Commons Attribution-ShareAlike 4.0 International License}}}
\usepackage{graphicx}
\setbeamercolor*{palette tertiary}{bg=que1}
\setbeamercolor*{palette primary}{bg=que2}
\begin{document}
\date{}

\section[Graphs]{what is this bit?}
\subsection[title]{and this?}

\begin{frame}[shrink=19,fragile]
	\begin{beamercolorbox}[rounded=true, left, shadow=true,wd=14.8cm]{que1}
		 Complete this table for y = 7 - 3x and draw the graph. \\[0.3cm] \renewcommand{\arraystretch}{1.2}\newcolumntype{C}{>{\centering\arraybackslash} m{1cm} }\begin{tabular}{|C|C|C|C|} \hline x & 1 & 2 & 3 \\ \hline y & & & \\ \hline \end{tabular}\\[0.3cm]
	\end{beamercolorbox}
\end{frame}
\begin{frame}[shrink=19,fragile]
	\begin{beamercolorbox}[rounded=true, left, shadow=true,wd=14.8cm]{que1}
		y = 7 - 3x\renewcommand{\arraystretch}{1.2}\newcolumntype{C}{>{\centering\arraybackslash} m{1cm} }\begin{tabular}{|C|C|C|C|} \hline x & 1 & 2 & 3 \\ \hline y & 4 & 1 & -2\\ \hline \end{tabular}\begin{gnuplot}[terminal=pdf] set nokey; set grid; set border 0; set xtics axis 1; set ytics axis 1; set size ratio 2; set size 1,1; set zeroaxis linewidth 2 linetype 1; set yrange [-3:8]; set xrange [-2:4]; plot -3*x+7 \end{gnuplot}
	\end{beamercolorbox}
\end{frame}

\begin{frame}[shrink=19,fragile]
	\begin{beamercolorbox}[rounded=true, left, shadow=true,wd=14.8cm]{que2}
		 Complete this table for y = 2 - 2x and draw the graph. \\[0.3cm] \renewcommand{\arraystretch}{1.2}\newcolumntype{C}{>{\centering\arraybackslash} m{1cm} }\begin{tabular}{|C|C|C|C|} \hline x & 1 & 2 & 3 \\ \hline y & & & \\ \hline \end{tabular}\\[0.3cm]
	\end{beamercolorbox}
\end{frame}
\begin{frame}[shrink=19,fragile]
	\begin{beamercolorbox}[rounded=true, left, shadow=true,wd=14.8cm]{que2}
 		y = 2 - 2x\renewcommand{\arraystretch}{1.2}\newcolumntype{C}{>{\centering\arraybackslash} m{1cm} }\begin{tabular}{|C|C|C|C|} \hline x & 1 & 2 & 3 \\ \hline y & 0 & -2 & -4\\ \hline \end{tabular}\begin{gnuplot}[terminal=pdf] set nokey; set grid; set border 0; set xtics axis 1; set ytics axis 1; set size ratio 2; set size 1,1; set zeroaxis linewidth 2 linetype 1; set yrange [-5:3]; set xrange [-2:4]; plot -2*x+2 \end{gnuplot}
	\end{beamercolorbox}
\end{frame}

\begin{frame}[shrink=19,fragile]
	\begin{beamercolorbox}[rounded=true, left, shadow=true,wd=14.8cm]{que1}
		 Complete this table for y = 5 - 2x and draw the graph. \\[0.3cm] \renewcommand{\arraystretch}{1.2}\newcolumntype{C}{>{\centering\arraybackslash} m{1cm} }\begin{tabular}{|C|C|C|C|} \hline x & 1 & 2 & 3 \\ \hline y & & & \\ \hline \end{tabular}\\[0.3cm]
	\end{beamercolorbox}
\end{frame}
\begin{frame}[shrink=19,fragile]
	\begin{beamercolorbox}[rounded=true, left, shadow=true,wd=14.8cm]{que1}
		y = 5 - 2x\renewcommand{\arraystretch}{1.2}\newcolumntype{C}{>{\centering\arraybackslash} m{1cm} }\begin{tabular}{|C|C|C|C|} \hline x & 1 & 2 & 3 \\ \hline y & 3 & 1 & -1\\ \hline \end{tabular}\begin{gnuplot}[terminal=pdf] set nokey; set grid; set border 0; set xtics axis 1; set ytics axis 1; set size ratio 2; set size 1,1; set zeroaxis linewidth 2 linetype 1; set yrange [-2:6]; set xrange [-2:4]; plot -2*x+5 \end{gnuplot}
	\end{beamercolorbox}
\end{frame}

\begin{frame}[shrink=19,fragile]
	\begin{beamercolorbox}[rounded=true, left, shadow=true,wd=14.8cm]{que2}
		 Complete this table for y = 4 - x and draw the graph. \\[0.3cm] \renewcommand{\arraystretch}{1.2}\newcolumntype{C}{>{\centering\arraybackslash} m{1cm} }\begin{tabular}{|C|C|C|C|} \hline x & 1 & 2 & 3 \\ \hline y & & & \\ \hline \end{tabular}\\[0.3cm]
	\end{beamercolorbox}
\end{frame}
\begin{frame}[shrink=19,fragile]
	\begin{beamercolorbox}[rounded=true, left, shadow=true,wd=14.8cm]{que2}
		y = 4 - x\renewcommand{\arraystretch}{1.2}\newcolumntype{C}{>{\centering\arraybackslash} m{1cm} }\begin{tabular}{|C|C|C|C|} \hline x & 1 & 2 & 3 \\ \hline y & 3 & 2 & 1\\ \hline \end{tabular}\begin{gnuplot}[terminal=pdf] set nokey; set grid; set border 0; set xtics axis 1; set ytics axis 1; set size ratio 2; set size 1,1; set zeroaxis linewidth 2 linetype 1; set yrange [-1:5]; set xrange [-2:4]; plot -1*x+4 \end{gnuplot}
	\end{beamercolorbox}
\end{frame}

\begin{frame}[shrink=19,fragile]
	\begin{beamercolorbox}[rounded=true, left, shadow=true,wd=14.8cm]{que1}
		 Complete this table for y = 4 - x and draw the graph. \\[0.3cm] \renewcommand{\arraystretch}{1.2}\newcolumntype{C}{>{\centering\arraybackslash} m{1cm} }\begin{tabular}{|C|C|C|C|} \hline x & 1 & 2 & 3 \\ \hline y & & & \\ \hline \end{tabular}\\[0.3cm]
	\end{beamercolorbox}
\end{frame}
\begin{frame}[shrink=19,fragile]
	\begin{beamercolorbox}[rounded=true, left, shadow=true,wd=14.8cm]{que1}
		y = 4 - x\renewcommand{\arraystretch}{1.2}\newcolumntype{C}{>{\centering\arraybackslash} m{1cm} }\begin{tabular}{|C|C|C|C|} \hline x & 1 & 2 & 3 \\ \hline y & 3 & 2 & 1\\ \hline \end{tabular}\begin{gnuplot}[terminal=pdf] set nokey; set grid; set border 0; set xtics axis 1; set ytics axis 1; set size ratio 2; set size 1,1; set zeroaxis linewidth 2 linetype 1; set yrange [-1:5]; set xrange [-2:4]; plot -1*x+4 \end{gnuplot}
	\end{beamercolorbox}
\end{frame}

\begin{frame}[shrink=19,fragile]
	\begin{beamercolorbox}[rounded=true, left, shadow=true,wd=14.8cm]{que2}
		 Complete this table for y = 1 - 3x and draw the graph. \\[0.3cm] \renewcommand{\arraystretch}{1.2}\newcolumntype{C}{>{\centering\arraybackslash} m{1cm} }\begin{tabular}{|C|C|C|C|} \hline x & 1 & 2 & 3 \\ \hline y & & & \\ \hline \end{tabular}\\[0.3cm]
	\end{beamercolorbox}
\end{frame}
\begin{frame}[shrink=19,fragile]
	\begin{beamercolorbox}[rounded=true, left, shadow=true,wd=14.8cm]{que2}
		y = 1 - 3x\renewcommand{\arraystretch}{1.2}\newcolumntype{C}{>{\centering\arraybackslash} m{1cm} }\begin{tabular}{|C|C|C|C|} \hline x & 1 & 2 & 3 \\ \hline y & -2 & -5 & -8\\ \hline \end{tabular}\begin{gnuplot}[terminal=pdf] set nokey; set grid; set border 0; set xtics axis 1; set ytics axis 1; set size ratio 2; set size 1,1; set zeroaxis linewidth 2 linetype 1; set yrange [-9:2]; set xrange [-2:4]; plot -3*x+1 \end{gnuplot}
	\end{beamercolorbox}
\end{frame}

\begin{frame}[shrink=19,fragile]
	\begin{beamercolorbox}[rounded=true, left, shadow=true,wd=14.8cm]{que1}
		 Complete this table for y = 2 - x and draw the graph. \\[0.3cm] \renewcommand{\arraystretch}{1.2}\newcolumntype{C}{>{\centering\arraybackslash} m{1cm} }\begin{tabular}{|C|C|C|C|} \hline x & 1 & 2 & 3 \\ \hline y & & & \\ \hline \end{tabular}\\[0.3cm]
	\end{beamercolorbox}
\end{frame}
\begin{frame}[shrink=19,fragile]
	\begin{beamercolorbox}[rounded=true, left, shadow=true,wd=14.8cm]{que1}
		y = 2 - x\renewcommand{\arraystretch}{1.2}\newcolumntype{C}{>{\centering\arraybackslash} m{1cm} }\begin{tabular}{|C|C|C|C|} \hline x & 1 & 2 & 3 \\ \hline y & 1 & 0 & -1\\ \hline \end{tabular}\begin{gnuplot}[terminal=pdf] set nokey; set grid; set border 0; set xtics axis 1; set ytics axis 1; set size ratio 2; set size 1,1; set zeroaxis linewidth 2 linetype 1; set yrange [-2:3]; set xrange [-2:4]; plot -1*x+2 \end{gnuplot}
	\end{beamercolorbox}
\end{frame}

\begin{frame}[shrink=19,fragile]
	\begin{beamercolorbox}[rounded=true, left, shadow=true,wd=14.8cm]{que2}
		 Complete this table for y = 4 - 3x and draw the graph. \\[0.3cm] \renewcommand{\arraystretch}{1.2}\newcolumntype{C}{>{\centering\arraybackslash} m{1cm} }\begin{tabular}{|C|C|C|C|} \hline x & 1 & 2 & 3 \\ \hline y & & & \\ \hline \end{tabular}\\[0.3cm]
	\end{beamercolorbox}
\end{frame}
\begin{frame}[shrink=19,fragile]
	\begin{beamercolorbox}[rounded=true, left, shadow=true,wd=14.8cm]{que2}
		y = 4 - 3x\renewcommand{\arraystretch}{1.2}\newcolumntype{C}{>{\centering\arraybackslash} m{1cm} }\begin{tabular}{|C|C|C|C|} \hline x & 1 & 2 & 3 \\ \hline y & 1 & -2 & -5\\ \hline \end{tabular}\begin{gnuplot}[terminal=pdf] set nokey; set grid; set border 0; set xtics axis 1; set ytics axis 1; set size ratio 2; set size 1,1; set zeroaxis linewidth 2 linetype 1; set yrange [-6:5]; set xrange [-2:4]; plot -3*x+4 \end{gnuplot}
	\end{beamercolorbox}
\end{frame}

\begin{frame}[shrink=19,fragile]
	\begin{beamercolorbox}[rounded=true, left, shadow=true,wd=14.8cm]{que1}
		 Complete this table for y = 3 - 3x and draw the graph. \\[0.3cm] \renewcommand{\arraystretch}{1.2}\newcolumntype{C}{>{\centering\arraybackslash} m{1cm} }\begin{tabular}{|C|C|C|C|} \hline x & 1 & 2 & 3 \\ \hline y & & & \\ \hline \end{tabular}\\[0.3cm]
	\end{beamercolorbox}
\end{frame}
\begin{frame}[shrink=19,fragile]
	\begin{beamercolorbox}[rounded=true, left, shadow=true,wd=14.8cm]{que1}
		y = 3 - 3x\renewcommand{\arraystretch}{1.2}\newcolumntype{C}{>{\centering\arraybackslash} m{1cm} }\begin{tabular}{|C|C|C|C|} \hline x & 1 & 2 & 3 \\ \hline y & 0 & -3 & -6\\ \hline \end{tabular}\begin{gnuplot}[terminal=pdf] set nokey; set grid; set border 0; set xtics axis 1; set ytics axis 1; set size ratio 2; set size 1,1; set zeroaxis linewidth 2 linetype 1; set yrange [-7:4]; set xrange [-2:4]; plot -3*x+3 \end{gnuplot}
	\end{beamercolorbox}
\end{frame}

\begin{frame}[shrink=19,fragile]
	\begin{beamercolorbox}[rounded=true, left, shadow=true,wd=14.8cm]{que2}
		 Complete this table for y = -3x and draw the graph. \\[0.3cm] \renewcommand{\arraystretch}{1.2}\newcolumntype{C}{>{\centering\arraybackslash} m{1cm} }\begin{tabular}{|C|C|C|C|} \hline x & 1 & 2 & 3 \\ \hline y & & & \\ \hline \end{tabular}\\[0.3cm]
	\end{beamercolorbox}
\end{frame}
\begin{frame}[shrink=19,fragile]
	\begin{beamercolorbox}[rounded=true, left, shadow=true,wd=14.8cm]{que2}
		y = -3x\renewcommand{\arraystretch}{1.2}\newcolumntype{C}{>{\centering\arraybackslash} m{1cm} }\begin{tabular}{|C|C|C|C|} \hline x & 1 & 2 & 3 \\ \hline y & -3 & -6 & -9\\ \hline \end{tabular}\begin{gnuplot}[terminal=pdf] set nokey; set grid; set border 0; set xtics axis 1; set ytics axis 1; set size ratio 2; set size 1,1; set zeroaxis linewidth 2 linetype 1; set yrange [-10:1]; set xrange [-2:4]; plot -3*x+0 \end{gnuplot}
	\end{beamercolorbox}
\end{frame}

\begin{frame}[shrink=19,fragile]
	\begin{beamercolorbox}[rounded=true, left, shadow=true,wd=14.8cm]{que1}
		 Complete this table for y = 2 - x and draw the graph. \\[0.3cm] \renewcommand{\arraystretch}{1.2}\newcolumntype{C}{>{\centering\arraybackslash} m{1cm} }\begin{tabular}{|C|C|C|C|} \hline x & 1 & 2 & 3 \\ \hline y & & & \\ \hline \end{tabular}\\[0.3cm]
	\end{beamercolorbox}
\end{frame}
\begin{frame}[shrink=19,fragile]
	\begin{beamercolorbox}[rounded=true, left, shadow=true,wd=14.8cm]{que1}
		y = 2 - x\renewcommand{\arraystretch}{1.2}\newcolumntype{C}{>{\centering\arraybackslash} m{1cm} }\begin{tabular}{|C|C|C|C|} \hline x & 1 & 2 & 3 \\ \hline y & 1 & 0 & -1\\ \hline \end{tabular}\begin{gnuplot}[terminal=pdf] set nokey; set grid; set border 0; set xtics axis 1; set ytics axis 1; set size ratio 2; set size 1,1; set zeroaxis linewidth 2 linetype 1; set yrange [-2:3]; set xrange [-2:4]; plot -1*x+2 \end{gnuplot}
	\end{beamercolorbox}
\end{frame}

\begin{frame}[shrink=19,fragile]
	\begin{beamercolorbox}[rounded=true, left, shadow=true,wd=14.8cm]{que2}
		 Complete this table for y = 4 - x and draw the graph. \\[0.3cm] \renewcommand{\arraystretch}{1.2}\newcolumntype{C}{>{\centering\arraybackslash} m{1cm} }\begin{tabular}{|C|C|C|C|} \hline x & 1 & 2 & 3 \\ \hline y & & & \\ \hline \end{tabular}\\[0.3cm]
	\end{beamercolorbox}
\end{frame}
\begin{frame}[shrink=19,fragile]
	\begin{beamercolorbox}[rounded=true, left, shadow=true,wd=14.8cm]{que2}
		y = 4 - x\renewcommand{\arraystretch}{1.2}\newcolumntype{C}{>{\centering\arraybackslash} m{1cm} }\begin{tabular}{|C|C|C|C|} \hline x & 1 & 2 & 3 \\ \hline y & 3 & 2 & 1\\ \hline \end{tabular}\begin{gnuplot}[terminal=pdf] set nokey; set grid; set border 0; set xtics axis 1; set ytics axis 1; set size ratio 2; set size 1,1; set zeroaxis linewidth 2 linetype 1; set yrange [-1:5]; set xrange [-2:4]; plot -1*x+4 \end{gnuplot}
	\end{beamercolorbox}
\end{frame}

\begin{frame}[shrink=19,fragile]
	\begin{beamercolorbox}[rounded=true, left, shadow=true,wd=14.8cm]{que1}
		 Complete this table for y = -3x and draw the graph. \\[0.3cm] \renewcommand{\arraystretch}{1.2}\newcolumntype{C}{>{\centering\arraybackslash} m{1cm} }\begin{tabular}{|C|C|C|C|} \hline x & 1 & 2 & 3 \\ \hline y & & & \\ \hline \end{tabular}\\[0.3cm]
	\end{beamercolorbox}
\end{frame}
\begin{frame}[shrink=19,fragile]
	\begin{beamercolorbox}[rounded=true, left, shadow=true,wd=14.8cm]{que1}
		y = -3x\renewcommand{\arraystretch}{1.2}\newcolumntype{C}{>{\centering\arraybackslash} m{1cm} }\begin{tabular}{|C|C|C|C|} \hline x & 1 & 2 & 3 \\ \hline y & -3 & -6 & -9\\ \hline \end{tabular}\begin{gnuplot}[terminal=pdf] set nokey; set grid; set border 0; set xtics axis 1; set ytics axis 1; set size ratio 2; set size 1,1; set zeroaxis linewidth 2 linetype 1; set yrange [-10:1]; set xrange [-2:4]; plot -3*x+0 \end{gnuplot}
	\end{beamercolorbox}
\end{frame}

\begin{frame}[shrink=19,fragile]
	\begin{beamercolorbox}[rounded=true, left, shadow=true,wd=14.8cm]{que2}
		 Complete this table for y = -3x and draw the graph. \\[0.3cm] \renewcommand{\arraystretch}{1.2}\newcolumntype{C}{>{\centering\arraybackslash} m{1cm} }\begin{tabular}{|C|C|C|C|} \hline x & 1 & 2 & 3 \\ \hline y & & & \\ \hline \end{tabular}\\[0.3cm]
	\end{beamercolorbox}
\end{frame}
\begin{frame}[shrink=19,fragile]
	\begin{beamercolorbox}[rounded=true, left, shadow=true,wd=14.8cm]{que2}
		y = -3x\renewcommand{\arraystretch}{1.2}\newcolumntype{C}{>{\centering\arraybackslash} m{1cm} }\begin{tabular}{|C|C|C|C|} \hline x & 1 & 2 & 3 \\ \hline y & -3 & -6 & -9\\ \hline \end{tabular}\begin{gnuplot}[terminal=pdf] set nokey; set grid; set border 0; set xtics axis 1; set ytics axis 1; set size ratio 2; set size 1,1; set zeroaxis linewidth 2 linetype 1; set yrange [-10:1]; set xrange [-2:4]; plot -3*x+0 \end{gnuplot}
	\end{beamercolorbox}
\end{frame}

\begin{frame}[shrink=19,fragile]
	\begin{beamercolorbox}[rounded=true, left, shadow=true,wd=14.8cm]{que1}
		 Complete this table for y = 9 - 2x and draw the graph. \\[0.3cm] \renewcommand{\arraystretch}{1.2}\newcolumntype{C}{>{\centering\arraybackslash} m{1cm} }\begin{tabular}{|C|C|C|C|} \hline x & 1 & 2 & 3 \\ \hline y & & & \\ \hline \end{tabular}\\[0.3cm]
	\end{beamercolorbox}
\end{frame}
\begin{frame}[shrink=19,fragile]
	\begin{beamercolorbox}[rounded=true, left, shadow=true,wd=14.8cm]{que1}
		y = 9 - 2x\renewcommand{\arraystretch}{1.2}\newcolumntype{C}{>{\centering\arraybackslash} m{1cm} }\begin{tabular}{|C|C|C|C|} \hline x & 1 & 2 & 3 \\ \hline y & 7 & 5 & 3\\ \hline \end{tabular}\begin{gnuplot}[terminal=pdf] set nokey; set grid; set border 0; set xtics axis 1; set ytics axis 1; set size ratio 2; set size 1,1; set zeroaxis linewidth 2 linetype 1; set yrange [-1:10]; set xrange [-2:4]; plot -2*x+9 \end{gnuplot}
	\end{beamercolorbox}
\end{frame}

\begin{frame}[shrink=19,fragile]
	\begin{beamercolorbox}[rounded=true, left, shadow=true,wd=14.8cm]{que2}
		 Complete this table for y = 4 - 3x and draw the graph. \\[0.3cm] \renewcommand{\arraystretch}{1.2}\newcolumntype{C}{>{\centering\arraybackslash} m{1cm} }\begin{tabular}{|C|C|C|C|} \hline x & 1 & 2 & 3 \\ \hline y & & & \\ \hline \end{tabular}\\[0.3cm]
	\end{beamercolorbox}
\end{frame}
\begin{frame}[shrink=19,fragile]
	\begin{beamercolorbox}[rounded=true, left, shadow=true,wd=14.8cm]{que2}
		y = 4 - 3x\renewcommand{\arraystretch}{1.2}\newcolumntype{C}{>{\centering\arraybackslash} m{1cm} }\begin{tabular}{|C|C|C|C|} \hline x & 1 & 2 & 3 \\ \hline y & 1 & -2 & -5\\ \hline \end{tabular}\begin{gnuplot}[terminal=pdf] set nokey; set grid; set border 0; set xtics axis 1; set ytics axis 1; set size ratio 2; set size 1,1; set zeroaxis linewidth 2 linetype 1; set yrange [-6:5]; set xrange [-2:4]; plot -3*x+4 \end{gnuplot}
	\end{beamercolorbox}
\end{frame}

\begin{frame}[shrink=19,fragile]
	\begin{beamercolorbox}[rounded=true, left, shadow=true,wd=14.8cm]{que1}
		 Complete this table for y = 8 - x and draw the graph. \\[0.3cm] \renewcommand{\arraystretch}{1.2}\newcolumntype{C}{>{\centering\arraybackslash} m{1cm} }\begin{tabular}{|C|C|C|C|} \hline x & 1 & 2 & 3 \\ \hline y & & & \\ \hline \end{tabular}\\[0.3cm]
	\end{beamercolorbox}
\end{frame}
\begin{frame}[shrink=19,fragile]
	\begin{beamercolorbox}[rounded=true, left, shadow=true,wd=14.8cm]{que1}
		y = 8 - x\renewcommand{\arraystretch}{1.2}\newcolumntype{C}{>{\centering\arraybackslash} m{1cm} }\begin{tabular}{|C|C|C|C|} \hline x & 1 & 2 & 3 \\ \hline y & 7 & 6 & 5\\ \hline \end{tabular}\begin{gnuplot}[terminal=pdf] set nokey; set grid; set border 0; set xtics axis 1; set ytics axis 1; set size ratio 2; set size 1,1; set zeroaxis linewidth 2 linetype 1; set yrange [-1:9]; set xrange [-2:4]; plot -1*x+8 \end{gnuplot}
	\end{beamercolorbox}
\end{frame}

\begin{frame}[shrink=19,fragile]
	\begin{beamercolorbox}[rounded=true, left, shadow=true,wd=14.8cm]{que2}
		 Complete this table for y = 3 - 2x and draw the graph. \\[0.3cm] \renewcommand{\arraystretch}{1.2}\newcolumntype{C}{>{\centering\arraybackslash} m{1cm} }\begin{tabular}{|C|C|C|C|} \hline x & 1 & 2 & 3 \\ \hline y & & & \\ \hline \end{tabular}\\[0.3cm]
	\end{beamercolorbox}
\end{frame}
\begin{frame}[shrink=19,fragile]
	\begin{beamercolorbox}[rounded=true, left, shadow=true,wd=14.8cm]{que2}
		y = 3 - 2x\renewcommand{\arraystretch}{1.2}\newcolumntype{C}{>{\centering\arraybackslash} m{1cm} }\begin{tabular}{|C|C|C|C|} \hline x & 1 & 2 & 3 \\ \hline y & 1 & -1 & -3\\ \hline \end{tabular}\begin{gnuplot}[terminal=pdf] set nokey; set grid; set border 0; set xtics axis 1; set ytics axis 1; set size ratio 2; set size 1,1; set zeroaxis linewidth 2 linetype 1; set yrange [-4:4]; set xrange [-2:4]; plot -2*x+3 \end{gnuplot}
	\end{beamercolorbox}
\end{frame}

\begin{frame}[shrink=19,fragile]
	\begin{beamercolorbox}[rounded=true, left, shadow=true,wd=14.8cm]{que1}
		 Complete this table for y = 4 - 2x and draw the graph. \\[0.3cm] \renewcommand{\arraystretch}{1.2}\newcolumntype{C}{>{\centering\arraybackslash} m{1cm} }\begin{tabular}{|C|C|C|C|} \hline x & 1 & 2 & 3 \\ \hline y & & & \\ \hline \end{tabular}\\[0.3cm]
	\end{beamercolorbox}
\end{frame}
\begin{frame}[shrink=19,fragile]
	\begin{beamercolorbox}[rounded=true, left, shadow=true,wd=14.8cm]{que1}
		y = 4 - 2x\renewcommand{\arraystretch}{1.2}\newcolumntype{C}{>{\centering\arraybackslash} m{1cm} }\begin{tabular}{|C|C|C|C|} \hline x & 1 & 2 & 3 \\ \hline y & 2 & 0 & -2\\ \hline \end{tabular}\begin{gnuplot}[terminal=pdf] set nokey; set grid; set border 0; set xtics axis 1; set ytics axis 1; set size ratio 2; set size 1,1; set zeroaxis linewidth 2 linetype 1; set yrange [-3:5]; set xrange [-2:4]; plot -2*x+4 \end{gnuplot}
	\end{beamercolorbox}
\end{frame}

\begin{frame}[shrink=19,fragile]
	\begin{beamercolorbox}[rounded=true, left, shadow=true,wd=14.8cm]{que2}
		 Complete this table for y = 8 - 3x and draw the graph. \\[0.3cm] \renewcommand{\arraystretch}{1.2}\newcolumntype{C}{>{\centering\arraybackslash} m{1cm} }\begin{tabular}{|C|C|C|C|} \hline x & 1 & 2 & 3 \\ \hline y & & & \\ \hline \end{tabular}\\[0.3cm]
	\end{beamercolorbox}
\end{frame}
\begin{frame}[shrink=19,fragile]
	\begin{beamercolorbox}[rounded=true, left, shadow=true,wd=14.8cm]{que2}
		y = 8 - 3x\renewcommand{\arraystretch}{1.2}\newcolumntype{C}{>{\centering\arraybackslash} m{1cm} }\begin{tabular}{|C|C|C|C|} \hline x & 1 & 2 & 3 \\ \hline y & 5 & 2 & -1\\ \hline \end{tabular}\begin{gnuplot}[terminal=pdf] set nokey; set grid; set border 0; set xtics axis 1; set ytics axis 1; set size ratio 2; set size 1,1; set zeroaxis linewidth 2 linetype 1; set yrange [-2:9]; set xrange [-2:4]; plot -3*x+8 \end{gnuplot}
	\end{beamercolorbox}
\end{frame}

\begin{frame}[shrink=19,fragile]
	\begin{beamercolorbox}[rounded=true, left, shadow=true,wd=14.8cm]{que1}
		 Complete this table for y = 2 - 3x and draw the graph. \\[0.3cm] \renewcommand{\arraystretch}{1.2}\newcolumntype{C}{>{\centering\arraybackslash} m{1cm} }\begin{tabular}{|C|C|C|C|} \hline x & 1 & 2 & 3 \\ \hline y & & & \\ \hline \end{tabular}\\[0.3cm]
	\end{beamercolorbox}
\end{frame}
\begin{frame}[shrink=19,fragile]
	\begin{beamercolorbox}[rounded=true, left, shadow=true,wd=14.8cm]{que1}
		y = 2 - 3x\renewcommand{\arraystretch}{1.2}\newcolumntype{C}{>{\centering\arraybackslash} m{1cm} }\begin{tabular}{|C|C|C|C|} \hline x & 1 & 2 & 3 \\ \hline y & -1 & -4 & -7\\ \hline \end{tabular}\begin{gnuplot}[terminal=pdf] set nokey; set grid; set border 0; set xtics axis 1; set ytics axis 1; set size ratio 2; set size 1,1; set zeroaxis linewidth 2 linetype 1; set yrange [-8:3]; set xrange [-2:4]; plot -3*x+2 \end{gnuplot}
	\end{beamercolorbox}
\end{frame}

\begin{frame}[shrink=19,fragile]
	\begin{beamercolorbox}[rounded=true, left, shadow=true,wd=14.8cm]{que2}
		 Complete this table for y = 6 - x and draw the graph. \\[0.3cm] \renewcommand{\arraystretch}{1.2}\newcolumntype{C}{>{\centering\arraybackslash} m{1cm} }\begin{tabular}{|C|C|C|C|} \hline x & 1 & 2 & 3 \\ \hline y & & & \\ \hline \end{tabular}\\[0.3cm]
	\end{beamercolorbox}
\end{frame}
\begin{frame}[shrink=19,fragile]
	\begin{beamercolorbox}[rounded=true, left, shadow=true,wd=14.8cm]{que2}
		y = 6 - x\renewcommand{\arraystretch}{1.2}\newcolumntype{C}{>{\centering\arraybackslash} m{1cm} }\begin{tabular}{|C|C|C|C|} \hline x & 1 & 2 & 3 \\ \hline y & 5 & 4 & 3\\ \hline \end{tabular}\begin{gnuplot}[terminal=pdf] set nokey; set grid; set border 0; set xtics axis 1; set ytics axis 1; set size ratio 2; set size 1,1; set zeroaxis linewidth 2 linetype 1; set yrange [-1:7]; set xrange [-2:4]; plot -1*x+6 \end{gnuplot}
	\end{beamercolorbox}
\end{frame}

\begin{frame}[shrink=19,fragile]
	\begin{beamercolorbox}[rounded=true, left, shadow=true,wd=14.8cm]{que1}
		 Complete this table for y = 1 - x and draw the graph. \\[0.3cm] \renewcommand{\arraystretch}{1.2}\newcolumntype{C}{>{\centering\arraybackslash} m{1cm} }\begin{tabular}{|C|C|C|C|} \hline x & 1 & 2 & 3 \\ \hline y & & & \\ \hline \end{tabular}\\[0.3cm]
	\end{beamercolorbox}
\end{frame}
\begin{frame}[shrink=19,fragile]
	\begin{beamercolorbox}[rounded=true, left, shadow=true,wd=14.8cm]{que1}
		y = 1 - x\renewcommand{\arraystretch}{1.2}\newcolumntype{C}{>{\centering\arraybackslash} m{1cm} }\begin{tabular}{|C|C|C|C|} \hline x & 1 & 2 & 3 \\ \hline y & 0 & -1 & -2\\ \hline \end{tabular}\begin{gnuplot}[terminal=pdf] set nokey; set grid; set border 0; set xtics axis 1; set ytics axis 1; set size ratio 2; set size 1,1; set zeroaxis linewidth 2 linetype 1; set yrange [-3:2]; set xrange [-2:4]; plot -1*x+1 \end{gnuplot}
	\end{beamercolorbox}
\end{frame}

\begin{frame}[shrink=19,fragile]
	\begin{beamercolorbox}[rounded=true, left, shadow=true,wd=14.8cm]{que2}
		 Complete this table for y = 7 - x and draw the graph. \\[0.3cm] \renewcommand{\arraystretch}{1.2}\newcolumntype{C}{>{\centering\arraybackslash} m{1cm} }\begin{tabular}{|C|C|C|C|} \hline x & 1 & 2 & 3 \\ \hline y & & & \\ \hline \end{tabular}\\[0.3cm]
	\end{beamercolorbox}
\end{frame}
\begin{frame}[shrink=19,fragile]
	\begin{beamercolorbox}[rounded=true, left, shadow=true,wd=14.8cm]{que2}
		y = 7 - x\renewcommand{\arraystretch}{1.2}\newcolumntype{C}{>{\centering\arraybackslash} m{1cm} }\begin{tabular}{|C|C|C|C|} \hline x & 1 & 2 & 3 \\ \hline y & 6 & 5 & 4\\ \hline \end{tabular}\begin{gnuplot}[terminal=pdf] set nokey; set grid; set border 0; set xtics axis 1; set ytics axis 1; set size ratio 2; set size 1,1; set zeroaxis linewidth 2 linetype 1; set yrange [-1:8]; set xrange [-2:4]; plot -1*x+7 \end{gnuplot}
	\end{beamercolorbox}
\end{frame}

\begin{frame}[shrink=19,fragile]
	\begin{beamercolorbox}[rounded=true, left, shadow=true,wd=14.8cm]{que1}
		 Complete this table for y = -x and draw the graph. \\[0.3cm] \renewcommand{\arraystretch}{1.2}\newcolumntype{C}{>{\centering\arraybackslash} m{1cm} }\begin{tabular}{|C|C|C|C|} \hline x & 1 & 2 & 3 \\ \hline y & & & \\ \hline \end{tabular}\\[0.3cm]
	\end{beamercolorbox}
\end{frame}
\begin{frame}[shrink=19,fragile]
	\begin{beamercolorbox}[rounded=true, left, shadow=true,wd=14.8cm]{que1}
		y = -x\renewcommand{\arraystretch}{1.2}\newcolumntype{C}{>{\centering\arraybackslash} m{1cm} }\begin{tabular}{|C|C|C|C|} \hline x & 1 & 2 & 3 \\ \hline y & -1 & -2 & -3\\ \hline \end{tabular}\begin{gnuplot}[terminal=pdf] set nokey; set grid; set border 0; set xtics axis 1; set ytics axis 1; set size ratio 2; set size 1,1; set zeroaxis linewidth 2 linetype 1; set yrange [-4:1]; set xrange [-2:4]; plot -1*x+0 \end{gnuplot}
	\end{beamercolorbox}
\end{frame}

\begin{frame}[shrink=19,fragile]
	\begin{beamercolorbox}[rounded=true, left, shadow=true,wd=14.8cm]{que2}
		 Complete this table for y = 9 - 3x and draw the graph. \\[0.3cm] \renewcommand{\arraystretch}{1.2}\newcolumntype{C}{>{\centering\arraybackslash} m{1cm} }\begin{tabular}{|C|C|C|C|} \hline x & 1 & 2 & 3 \\ \hline y & & & \\ \hline \end{tabular}\\[0.3cm]
	\end{beamercolorbox}
\end{frame}
\begin{frame}[shrink=19,fragile]
	\begin{beamercolorbox}[rounded=true, left, shadow=true,wd=14.8cm]{que2}
		y = 9 - 3x\renewcommand{\arraystretch}{1.2}\newcolumntype{C}{>{\centering\arraybackslash} m{1cm} }\begin{tabular}{|C|C|C|C|} \hline x & 1 & 2 & 3 \\ \hline y & 6 & 3 & 0\\ \hline \end{tabular}\begin{gnuplot}[terminal=pdf] set nokey; set grid; set border 0; set xtics axis 1; set ytics axis 1; set size ratio 2; set size 1,1; set zeroaxis linewidth 2 linetype 1; set yrange [-1:10]; set xrange [-2:4]; plot -3*x+9 \end{gnuplot}
	\end{beamercolorbox}
\end{frame}

\begin{frame}[shrink=19,fragile]
	\begin{beamercolorbox}[rounded=true, left, shadow=true,wd=14.8cm]{que1}
		 Complete this table for y = 7 - 3x and draw the graph. \\[0.3cm] \renewcommand{\arraystretch}{1.2}\newcolumntype{C}{>{\centering\arraybackslash} m{1cm} }\begin{tabular}{|C|C|C|C|} \hline x & 1 & 2 & 3 \\ \hline y & & & \\ \hline \end{tabular}\\[0.3cm]
	\end{beamercolorbox}
\end{frame}
\begin{frame}[shrink=19,fragile]
	\begin{beamercolorbox}[rounded=true, left, shadow=true,wd=14.8cm]{que1}
		y = 7 - 3x\renewcommand{\arraystretch}{1.2}\newcolumntype{C}{>{\centering\arraybackslash} m{1cm} }\begin{tabular}{|C|C|C|C|} \hline x & 1 & 2 & 3 \\ \hline y & 4 & 1 & -2\\ \hline \end{tabular}\begin{gnuplot}[terminal=pdf] set nokey; set grid; set border 0; set xtics axis 1; set ytics axis 1; set size ratio 2; set size 1,1; set zeroaxis linewidth 2 linetype 1; set yrange [-3:8]; set xrange [-2:4]; plot -3*x+7 \end{gnuplot}
	\end{beamercolorbox}
\end{frame}

\begin{frame}[shrink=19,fragile]
	\begin{beamercolorbox}[rounded=true, left, shadow=true,wd=14.8cm]{que2}
		 Complete this table for y = -x and draw the graph. \\[0.3cm] \renewcommand{\arraystretch}{1.2}\newcolumntype{C}{>{\centering\arraybackslash} m{1cm} }\begin{tabular}{|C|C|C|C|} \hline x & 1 & 2 & 3 \\ \hline y & & & \\ \hline \end{tabular}\\[0.3cm]
	\end{beamercolorbox}
\end{frame}
\begin{frame}[shrink=19,fragile]
	\begin{beamercolorbox}[rounded=true, left, shadow=true,wd=14.8cm]{que2}
		y = -x\renewcommand{\arraystretch}{1.2}\newcolumntype{C}{>{\centering\arraybackslash} m{1cm} }\begin{tabular}{|C|C|C|C|} \hline x & 1 & 2 & 3 \\ \hline y & -1 & -2 & -3\\ \hline \end{tabular}\begin{gnuplot}[terminal=pdf] set nokey; set grid; set border 0; set xtics axis 1; set ytics axis 1; set size ratio 2; set size 1,1; set zeroaxis linewidth 2 linetype 1; set yrange [-4:1]; set xrange [-2:4]; plot -1*x+0 \end{gnuplot}
	\end{beamercolorbox}
\end{frame}

\begin{frame}[shrink=19,fragile]
	\begin{beamercolorbox}[rounded=true, left, shadow=true,wd=14.8cm]{que1}
		 Complete this table for y = 3 - 2x and draw the graph. \\[0.3cm] \renewcommand{\arraystretch}{1.2}\newcolumntype{C}{>{\centering\arraybackslash} m{1cm} }\begin{tabular}{|C|C|C|C|} \hline x & 1 & 2 & 3 \\ \hline y & & & \\ \hline \end{tabular}\\[0.3cm]
	\end{beamercolorbox}
\end{frame}
\begin{frame}[shrink=19,fragile]
	\begin{beamercolorbox}[rounded=true, left, shadow=true,wd=14.8cm]{que1}
		y = 3 - 2x\renewcommand{\arraystretch}{1.2}\newcolumntype{C}{>{\centering\arraybackslash} m{1cm} }\begin{tabular}{|C|C|C|C|} \hline x & 1 & 2 & 3 \\ \hline y & 1 & -1 & -3\\ \hline \end{tabular}\begin{gnuplot}[terminal=pdf] set nokey; set grid; set border 0; set xtics axis 1; set ytics axis 1; set size ratio 2; set size 1,1; set zeroaxis linewidth 2 linetype 1; set yrange [-4:4]; set xrange [-2:4]; plot -2*x+3 \end{gnuplot}
	\end{beamercolorbox}
\end{frame}

\begin{frame}[shrink=19,fragile]
	\begin{beamercolorbox}[rounded=true, left, shadow=true,wd=14.8cm]{que2}
		 Complete this table for y = 5 - x and draw the graph. \\[0.3cm] \renewcommand{\arraystretch}{1.2}\newcolumntype{C}{>{\centering\arraybackslash} m{1cm} }\begin{tabular}{|C|C|C|C|} \hline x & 1 & 2 & 3 \\ \hline y & & & \\ \hline \end{tabular}\\[0.3cm]
	\end{beamercolorbox}
\end{frame}
\begin{frame}[shrink=19,fragile]
	\begin{beamercolorbox}[rounded=true, left, shadow=true,wd=14.8cm]{que2}
		y = 5 - x\renewcommand{\arraystretch}{1.2}\newcolumntype{C}{>{\centering\arraybackslash} m{1cm} }\begin{tabular}{|C|C|C|C|} \hline x & 1 & 2 & 3 \\ \hline y & 4 & 3 & 2\\ \hline \end{tabular}\begin{gnuplot}[terminal=pdf] set nokey; set grid; set border 0; set xtics axis 1; set ytics axis 1; set size ratio 2; set size 1,1; set zeroaxis linewidth 2 linetype 1; set yrange [-1:6]; set xrange [-2:4]; plot -1*x+5 \end{gnuplot}
	\end{beamercolorbox}
\end{frame}

\begin{frame}[shrink=19,fragile]
	\begin{beamercolorbox}[rounded=true, left, shadow=true,wd=14.8cm]{que1}
		 Complete this table for y = 4 - 2x and draw the graph. \\[0.3cm] \renewcommand{\arraystretch}{1.2}\newcolumntype{C}{>{\centering\arraybackslash} m{1cm} }\begin{tabular}{|C|C|C|C|} \hline x & 1 & 2 & 3 \\ \hline y & & & \\ \hline \end{tabular}\\[0.3cm]
	\end{beamercolorbox}
\end{frame}
\begin{frame}[shrink=19,fragile]
	\begin{beamercolorbox}[rounded=true, left, shadow=true,wd=14.8cm]{que1}
		y = 4 - 2x\renewcommand{\arraystretch}{1.2}\newcolumntype{C}{>{\centering\arraybackslash} m{1cm} }\begin{tabular}{|C|C|C|C|} \hline x & 1 & 2 & 3 \\ \hline y & 2 & 0 & -2\\ \hline \end{tabular}\begin{gnuplot}[terminal=pdf] set nokey; set grid; set border 0; set xtics axis 1; set ytics axis 1; set size ratio 2; set size 1,1; set zeroaxis linewidth 2 linetype 1; set yrange [-3:5]; set xrange [-2:4]; plot -2*x+4 \end{gnuplot}
	\end{beamercolorbox}
\end{frame}

\begin{frame}[shrink=19,fragile]
	\begin{beamercolorbox}[rounded=true, left, shadow=true,wd=14.8cm]{que2}
		 Complete this table for y = 7 - 3x and draw the graph. \\[0.3cm] \renewcommand{\arraystretch}{1.2}\newcolumntype{C}{>{\centering\arraybackslash} m{1cm} }\begin{tabular}{|C|C|C|C|} \hline x & 1 & 2 & 3 \\ \hline y & & & \\ \hline \end{tabular}\\[0.3cm]
	\end{beamercolorbox}
\end{frame}
\begin{frame}[shrink=19,fragile]
	\begin{beamercolorbox}[rounded=true, left, shadow=true,wd=14.8cm]{que2}
		y = 7 - 3x\renewcommand{\arraystretch}{1.2}\newcolumntype{C}{>{\centering\arraybackslash} m{1cm} }\begin{tabular}{|C|C|C|C|} \hline x & 1 & 2 & 3 \\ \hline y & 4 & 1 & -2\\ \hline \end{tabular}\begin{gnuplot}[terminal=pdf] set nokey; set grid; set border 0; set xtics axis 1; set ytics axis 1; set size ratio 2; set size 1,1; set zeroaxis linewidth 2 linetype 1; set yrange [-3:8]; set xrange [-2:4]; plot -3*x+7 \end{gnuplot}
	\end{beamercolorbox}
\end{frame}

\begin{frame}[shrink=19,fragile]
	\begin{beamercolorbox}[rounded=true, left, shadow=true,wd=14.8cm]{que1}
		 Complete this table for y = 4 - 2x and draw the graph. \\[0.3cm] \renewcommand{\arraystretch}{1.2}\newcolumntype{C}{>{\centering\arraybackslash} m{1cm} }\begin{tabular}{|C|C|C|C|} \hline x & 1 & 2 & 3 \\ \hline y & & & \\ \hline \end{tabular}\\[0.3cm]
	\end{beamercolorbox}
\end{frame}
\begin{frame}[shrink=19,fragile]
	\begin{beamercolorbox}[rounded=true, left, shadow=true,wd=14.8cm]{que1}
		y = 4 - 2x\renewcommand{\arraystretch}{1.2}\newcolumntype{C}{>{\centering\arraybackslash} m{1cm} }\begin{tabular}{|C|C|C|C|} \hline x & 1 & 2 & 3 \\ \hline y & 2 & 0 & -2\\ \hline \end{tabular}\begin{gnuplot}[terminal=pdf] set nokey; set grid; set border 0; set xtics axis 1; set ytics axis 1; set size ratio 2; set size 1,1; set zeroaxis linewidth 2 linetype 1; set yrange [-3:5]; set xrange [-2:4]; plot -2*x+4 \end{gnuplot}
	\end{beamercolorbox}
\end{frame}

\begin{frame}[shrink=19,fragile]
	\begin{beamercolorbox}[rounded=true, left, shadow=true,wd=14.8cm]{que2}
		 Complete this table for y = 2 - 3x and draw the graph. \\[0.3cm] \renewcommand{\arraystretch}{1.2}\newcolumntype{C}{>{\centering\arraybackslash} m{1cm} }\begin{tabular}{|C|C|C|C|} \hline x & 1 & 2 & 3 \\ \hline y & & & \\ \hline \end{tabular}\\[0.3cm]
	\end{beamercolorbox}
\end{frame}
\begin{frame}[shrink=19,fragile]
	\begin{beamercolorbox}[rounded=true, left, shadow=true,wd=14.8cm]{que2}
		y = 2 - 3x\renewcommand{\arraystretch}{1.2}\newcolumntype{C}{>{\centering\arraybackslash} m{1cm} }\begin{tabular}{|C|C|C|C|} \hline x & 1 & 2 & 3 \\ \hline y & -1 & -4 & -7\\ \hline \end{tabular}\begin{gnuplot}[terminal=pdf] set nokey; set grid; set border 0; set xtics axis 1; set ytics axis 1; set size ratio 2; set size 1,1; set zeroaxis linewidth 2 linetype 1; set yrange [-8:3]; set xrange [-2:4]; plot -3*x+2 \end{gnuplot}
	\end{beamercolorbox}
\end{frame}

\begin{frame}[shrink=19,fragile]
	\begin{beamercolorbox}[rounded=true, left, shadow=true,wd=14.8cm]{que1}
		 Complete this table for y = 8 - 2x and draw the graph. \\[0.3cm] \renewcommand{\arraystretch}{1.2}\newcolumntype{C}{>{\centering\arraybackslash} m{1cm} }\begin{tabular}{|C|C|C|C|} \hline x & 1 & 2 & 3 \\ \hline y & & & \\ \hline \end{tabular}\\[0.3cm]
	\end{beamercolorbox}
\end{frame}
\begin{frame}[shrink=19,fragile]
	\begin{beamercolorbox}[rounded=true, left, shadow=true,wd=14.8cm]{que1}
		y = 8 - 2x\renewcommand{\arraystretch}{1.2}\newcolumntype{C}{>{\centering\arraybackslash} m{1cm} }\begin{tabular}{|C|C|C|C|} \hline x & 1 & 2 & 3 \\ \hline y & 6 & 4 & 2\\ \hline \end{tabular}\begin{gnuplot}[terminal=pdf] set nokey; set grid; set border 0; set xtics axis 1; set ytics axis 1; set size ratio 2; set size 1,1; set zeroaxis linewidth 2 linetype 1; set yrange [-1:9]; set xrange [-2:4]; plot -2*x+8 \end{gnuplot}
	\end{beamercolorbox}
\end{frame}

\begin{frame}[shrink=19,fragile]
	\begin{beamercolorbox}[rounded=true, left, shadow=true,wd=14.8cm]{que2}
		 Complete this table for y = 8 - 2x and draw the graph. \\[0.3cm] \renewcommand{\arraystretch}{1.2}\newcolumntype{C}{>{\centering\arraybackslash} m{1cm} }\begin{tabular}{|C|C|C|C|} \hline x & 1 & 2 & 3 \\ \hline y & & & \\ \hline \end{tabular}\\[0.3cm]
	\end{beamercolorbox}
\end{frame}
\begin{frame}[shrink=19,fragile]
	\begin{beamercolorbox}[rounded=true, left, shadow=true,wd=14.8cm]{que2}
 		y = 8 - 2x\renewcommand{\arraystretch}{1.2}\newcolumntype{C}{>{\centering\arraybackslash} m{1cm} }\begin{tabular}{|C|C|C|C|} \hline x & 1 & 2 & 3 \\ \hline y & 6 & 4 & 2\\ \hline \end{tabular}\begin{gnuplot}[terminal=pdf] set nokey; set grid; set border 0; set xtics axis 1; set ytics axis 1; set size ratio 2; set size 1,1; set zeroaxis linewidth 2 linetype 1; set yrange [-1:9]; set xrange [-2:4]; plot -2*x+8 \end{gnuplot}
	\end{beamercolorbox}
\end{frame}

\begin{frame}[shrink=19,fragile]
	\begin{beamercolorbox}[rounded=true, left, shadow=true,wd=14.8cm]{que1}
		 Complete this table for y = 1 - 3x and draw the graph. \\[0.3cm] \renewcommand{\arraystretch}{1.2}\newcolumntype{C}{>{\centering\arraybackslash} m{1cm} }\begin{tabular}{|C|C|C|C|} \hline x & 1 & 2 & 3 \\ \hline y & & & \\ \hline \end{tabular}\\[0.3cm]
	\end{beamercolorbox}
\end{frame}
\begin{frame}[shrink=19,fragile]
	\begin{beamercolorbox}[rounded=true, left, shadow=true,wd=14.8cm]{que1}
		y = 1 - 3x\renewcommand{\arraystretch}{1.2}\newcolumntype{C}{>{\centering\arraybackslash} m{1cm} }\begin{tabular}{|C|C|C|C|} \hline x & 1 & 2 & 3 \\ \hline y & -2 & -5 & -8\\ \hline \end{tabular}\begin{gnuplot}[terminal=pdf] set nokey; set grid; set border 0; set xtics axis 1; set ytics axis 1; set size ratio 2; set size 1,1; set zeroaxis linewidth 2 linetype 1; set yrange [-9:2]; set xrange [-2:4]; plot -3*x+1 \end{gnuplot}
	\end{beamercolorbox}
\end{frame}

\begin{frame}[shrink=19,fragile]
	\begin{beamercolorbox}[rounded=true, left, shadow=true,wd=14.8cm]{que2}
		 Complete this table for y = 6 - 3x and draw the graph. \\[0.3cm] \renewcommand{\arraystretch}{1.2}\newcolumntype{C}{>{\centering\arraybackslash} m{1cm} }\begin{tabular}{|C|C|C|C|} \hline x & 1 & 2 & 3 \\ \hline y & & & \\ \hline \end{tabular}\\[0.3cm]
	\end{beamercolorbox}
\end{frame}
\begin{frame}[shrink=19,fragile]
	\begin{beamercolorbox}[rounded=true, left, shadow=true,wd=14.8cm]{que2}
		y = 6 - 3x\renewcommand{\arraystretch}{1.2}\newcolumntype{C}{>{\centering\arraybackslash} m{1cm} }\begin{tabular}{|C|C|C|C|} \hline x & 1 & 2 & 3 \\ \hline y & 3 & 0 & -3\\ \hline \end{tabular}\begin{gnuplot}[terminal=pdf] set nokey; set grid; set border 0; set xtics axis 1; set ytics axis 1; set size ratio 2; set size 1,1; set zeroaxis linewidth 2 linetype 1; set yrange [-4:7]; set xrange [-2:4]; plot -3*x+6 \end{gnuplot}
	\end{beamercolorbox}
\end{frame}

\begin{frame}[shrink=19,fragile]
	\begin{beamercolorbox}[rounded=true, left, shadow=true,wd=14.8cm]{que1}
		 Complete this table for y = 2 - x and draw the graph. \\[0.3cm] \renewcommand{\arraystretch}{1.2}\newcolumntype{C}{>{\centering\arraybackslash} m{1cm} }\begin{tabular}{|C|C|C|C|} \hline x & 1 & 2 & 3 \\ \hline y & & & \\ \hline \end{tabular}\\[0.3cm]
	\end{beamercolorbox}
\end{frame}
\begin{frame}[shrink=19,fragile]
	\begin{beamercolorbox}[rounded=true, left, shadow=true,wd=14.8cm]{que1}
		y = 2 - x\renewcommand{\arraystretch}{1.2}\newcolumntype{C}{>{\centering\arraybackslash} m{1cm} }\begin{tabular}{|C|C|C|C|} \hline x & 1 & 2 & 3 \\ \hline y & 1 & 0 & -1\\ \hline \end{tabular}\begin{gnuplot}[terminal=pdf] set nokey; set grid; set border 0; set xtics axis 1; set ytics axis 1; set size ratio 2; set size 1,1; set zeroaxis linewidth 2 linetype 1; set yrange [-2:3]; set xrange [-2:4]; plot -1*x+2 \end{gnuplot}
	\end{beamercolorbox}
\end{frame}

\begin{frame}[shrink=19,fragile]
	\begin{beamercolorbox}[rounded=true, left, shadow=true,wd=14.8cm]{que2}
		 Complete this table for y = 7 - 2x and draw the graph. \\[0.3cm] \renewcommand{\arraystretch}{1.2}\newcolumntype{C}{>{\centering\arraybackslash} m{1cm} }\begin{tabular}{|C|C|C|C|} \hline x & 1 & 2 & 3 \\ \hline y & & & \\ \hline \end{tabular}\\[0.3cm]
	\end{beamercolorbox}
\end{frame}
\begin{frame}[shrink=19,fragile]
	\begin{beamercolorbox}[rounded=true, left, shadow=true,wd=14.8cm]{que2}
		y = 7 - 2x\renewcommand{\arraystretch}{1.2}\newcolumntype{C}{>{\centering\arraybackslash} m{1cm} }\begin{tabular}{|C|C|C|C|} \hline x & 1 & 2 & 3 \\ \hline y & 5 & 3 & 1\\ \hline \end{tabular}\begin{gnuplot}[terminal=pdf] set nokey; set grid; set border 0; set xtics axis 1; set ytics axis 1; set size ratio 2; set size 1,1; set zeroaxis linewidth 2 linetype 1; set yrange [-1:8]; set xrange [-2:4]; plot -2*x+7 \end{gnuplot}
	\end{beamercolorbox}
\end{frame}

\begin{frame}[shrink=19,fragile]
	\begin{beamercolorbox}[rounded=true, left, shadow=true,wd=14.8cm]{que1}
		 Complete this table for y = 9 - x and draw the graph. \\[0.3cm] \renewcommand{\arraystretch}{1.2}\newcolumntype{C}{>{\centering\arraybackslash} m{1cm} }\begin{tabular}{|C|C|C|C|} \hline x & 1 & 2 & 3 \\ \hline y & & & \\ \hline \end{tabular}\\[0.3cm]
	\end{beamercolorbox}
\end{frame}
\begin{frame}[shrink=19,fragile]
	\begin{beamercolorbox}[rounded=true, left, shadow=true,wd=14.8cm]{que1}
		y = 9 - x\renewcommand{\arraystretch}{1.2}\newcolumntype{C}{>{\centering\arraybackslash} m{1cm} }\begin{tabular}{|C|C|C|C|} \hline x & 1 & 2 & 3 \\ \hline y & 8 & 7 & 6\\ \hline \end{tabular}\begin{gnuplot}[terminal=pdf] set nokey; set grid; set border 0; set xtics axis 1; set ytics axis 1; set size ratio 2; set size 1,1; set zeroaxis linewidth 2 linetype 1; set yrange [-1:10]; set xrange [-2:4]; plot -1*x+9 \end{gnuplot}
	\end{beamercolorbox}
\end{frame}

\begin{frame}[shrink=19,fragile]
	\begin{beamercolorbox}[rounded=true, left, shadow=true,wd=14.8cm]{que2}
		 Complete this table for y = 9 - 3x and draw the graph. \\[0.3cm] \renewcommand{\arraystretch}{1.2}\newcolumntype{C}{>{\centering\arraybackslash} m{1cm} }\begin{tabular}{|C|C|C|C|} \hline x & 1 & 2 & 3 \\ \hline y & & & \\ \hline \end{tabular}\\[0.3cm]
	\end{beamercolorbox}
\end{frame}
\begin{frame}[shrink=19,fragile]
	\begin{beamercolorbox}[rounded=true, left, shadow=true,wd=14.8cm]{que2}
		y = 9 - 3x\renewcommand{\arraystretch}{1.2}\newcolumntype{C}{>{\centering\arraybackslash} m{1cm} }\begin{tabular}{|C|C|C|C|} \hline x & 1 & 2 & 3 \\ \hline y & 6 & 3 & 0\\ \hline \end{tabular}\begin{gnuplot}[terminal=pdf] set nokey; set grid; set border 0; set xtics axis 1; set ytics axis 1; set size ratio 2; set size 1,1; set zeroaxis linewidth 2 linetype 1; set yrange [-1:10]; set xrange [-2:4]; plot -3*x+9 \end{gnuplot}
	\end{beamercolorbox}
\end{frame}

\begin{frame}[shrink=19,fragile]
	\begin{beamercolorbox}[rounded=true, left, shadow=true,wd=14.8cm]{que1}
		 Complete this table for y = 7 - 3x and draw the graph. \\[0.3cm] \renewcommand{\arraystretch}{1.2}\newcolumntype{C}{>{\centering\arraybackslash} m{1cm} }\begin{tabular}{|C|C|C|C|} \hline x & 1 & 2 & 3 \\ \hline y & & & \\ \hline \end{tabular}\\[0.3cm]
	\end{beamercolorbox}
\end{frame}
\begin{frame}[shrink=19,fragile]
	\begin{beamercolorbox}[rounded=true, left, shadow=true,wd=14.8cm]{que1}
		y = 7 - 3x\renewcommand{\arraystretch}{1.2}\newcolumntype{C}{>{\centering\arraybackslash} m{1cm} }\begin{tabular}{|C|C|C|C|} \hline x & 1 & 2 & 3 \\ \hline y & 4 & 1 & -2\\ \hline \end{tabular}\begin{gnuplot}[terminal=pdf] set nokey; set grid; set border 0; set xtics axis 1; set ytics axis 1; set size ratio 2; set size 1,1; set zeroaxis linewidth 2 linetype 1; set yrange [-3:8]; set xrange [-2:4]; plot -3*x+7 \end{gnuplot}
	\end{beamercolorbox}
\end{frame}

\begin{frame}[shrink=19,fragile]
	\begin{beamercolorbox}[rounded=true, left, shadow=true,wd=14.8cm]{que2}
		 Complete this table for y = 2 - 3x and draw the graph. \\[0.3cm] \renewcommand{\arraystretch}{1.2}\newcolumntype{C}{>{\centering\arraybackslash} m{1cm} }\begin{tabular}{|C|C|C|C|} \hline x & 1 & 2 & 3 \\ \hline y & & & \\ \hline \end{tabular}\\[0.3cm]
	\end{beamercolorbox}
\end{frame}
\begin{frame}[shrink=19,fragile]
	\begin{beamercolorbox}[rounded=true, left, shadow=true,wd=14.8cm]{que2}
		y = 2 - 3x\renewcommand{\arraystretch}{1.2}\newcolumntype{C}{>{\centering\arraybackslash} m{1cm} }\begin{tabular}{|C|C|C|C|} \hline x & 1 & 2 & 3 \\ \hline y & -1 & -4 & -7\\ \hline \end{tabular}\begin{gnuplot}[terminal=pdf] set nokey; set grid; set border 0; set xtics axis 1; set ytics axis 1; set size ratio 2; set size 1,1; set zeroaxis linewidth 2 linetype 1; set yrange [-8:3]; set xrange [-2:4]; plot -3*x+2 \end{gnuplot}
	\end{beamercolorbox}
\end{frame}

\begin{frame}[shrink=19,fragile]
	\begin{beamercolorbox}[rounded=true, left, shadow=true,wd=14.8cm]{que1}
		 Complete this table for y = 9 - x and draw the graph. \\[0.3cm] \renewcommand{\arraystretch}{1.2}\newcolumntype{C}{>{\centering\arraybackslash} m{1cm} }\begin{tabular}{|C|C|C|C|} \hline x & 1 & 2 & 3 \\ \hline y & & & \\ \hline \end{tabular}\\[0.3cm]
	\end{beamercolorbox}
\end{frame}
\begin{frame}[shrink=19,fragile]
	\begin{beamercolorbox}[rounded=true, left, shadow=true,wd=14.8cm]{que1}
		y = 9 - x\renewcommand{\arraystretch}{1.2}\newcolumntype{C}{>{\centering\arraybackslash} m{1cm} }\begin{tabular}{|C|C|C|C|} \hline x & 1 & 2 & 3 \\ \hline y & 8 & 7 & 6\\ \hline \end{tabular}\begin{gnuplot}[terminal=pdf] set nokey; set grid; set border 0; set xtics axis 1; set ytics axis 1; set size ratio 2; set size 1,1; set zeroaxis linewidth 2 linetype 1; set yrange [-1:10]; set xrange [-2:4]; plot -1*x+9 \end{gnuplot}
	\end{beamercolorbox}
\end{frame}

\begin{frame}[shrink=19,fragile]
	\begin{beamercolorbox}[rounded=true, left, shadow=true,wd=14.8cm]{que2}
		 Complete this table for y = 5 - x and draw the graph. \\[0.3cm] \renewcommand{\arraystretch}{1.2}\newcolumntype{C}{>{\centering\arraybackslash} m{1cm} }\begin{tabular}{|C|C|C|C|} \hline x & 1 & 2 & 3 \\ \hline y & & & \\ \hline \end{tabular}\\[0.3cm]
	\end{beamercolorbox}
\end{frame}
\begin{frame}[shrink=19,fragile]
	\begin{beamercolorbox}[rounded=true, left, shadow=true,wd=14.8cm]{que2}
		y = 5 - x\renewcommand{\arraystretch}{1.2}\newcolumntype{C}{>{\centering\arraybackslash} m{1cm} }\begin{tabular}{|C|C|C|C|} \hline x & 1 & 2 & 3 \\ \hline y & 4 & 3 & 2\\ \hline \end{tabular}\begin{gnuplot}[terminal=pdf] set nokey; set grid; set border 0; set xtics axis 1; set ytics axis 1; set size ratio 2; set size 1,1; set zeroaxis linewidth 2 linetype 1; set yrange [-1:6]; set xrange [-2:4]; plot -1*x+5 \end{gnuplot}
	\end{beamercolorbox}
\end{frame}

\begin{frame}[shrink=19,fragile]
	\begin{beamercolorbox}[rounded=true, left, shadow=true,wd=14.8cm]{que1}
		 Complete this table for y = 8 - 2x and draw the graph. \\[0.3cm] \renewcommand{\arraystretch}{1.2}\newcolumntype{C}{>{\centering\arraybackslash} m{1cm} }\begin{tabular}{|C|C|C|C|} \hline x & 1 & 2 & 3 \\ \hline y & & & \\ \hline \end{tabular}\\[0.3cm]
	\end{beamercolorbox}
\end{frame}
\begin{frame}[shrink=19,fragile]
	\begin{beamercolorbox}[rounded=true, left, shadow=true,wd=14.8cm]{que1}
		y = 8 - 2x\renewcommand{\arraystretch}{1.2}\newcolumntype{C}{>{\centering\arraybackslash} m{1cm} }\begin{tabular}{|C|C|C|C|} \hline x & 1 & 2 & 3 \\ \hline y & 6 & 4 & 2\\ \hline \end{tabular}\begin{gnuplot}[terminal=pdf] set nokey; set grid; set border 0; set xtics axis 1; set ytics axis 1; set size ratio 2; set size 1,1; set zeroaxis linewidth 2 linetype 1; set yrange [-1:9]; set xrange [-2:4]; plot -2*x+8 \end{gnuplot}
	\end{beamercolorbox}
\end{frame}

\begin{frame}[shrink=19,fragile]
	\begin{beamercolorbox}[rounded=true, left, shadow=true,wd=14.8cm]{que2}
		 Complete this table for y = 3 - 2x and draw the graph. \\[0.3cm] \renewcommand{\arraystretch}{1.2}\newcolumntype{C}{>{\centering\arraybackslash} m{1cm} }\begin{tabular}{|C|C|C|C|} \hline x & 1 & 2 & 3 \\ \hline y & & & \\ \hline \end{tabular}\\[0.3cm]
	\end{beamercolorbox}
\end{frame}
\begin{frame}[shrink=19,fragile]
	\begin{beamercolorbox}[rounded=true, left, shadow=true,wd=14.8cm]{que2}
		y = 3 - 2x\renewcommand{\arraystretch}{1.2}\newcolumntype{C}{>{\centering\arraybackslash} m{1cm} }\begin{tabular}{|C|C|C|C|} \hline x & 1 & 2 & 3 \\ \hline y & 1 & -1 & -3\\ \hline \end{tabular}\begin{gnuplot}[terminal=pdf] set nokey; set grid; set border 0; set xtics axis 1; set ytics axis 1; set size ratio 2; set size 1,1; set zeroaxis linewidth 2 linetype 1; set yrange [-4:4]; set xrange [-2:4]; plot -2*x+3 \end{gnuplot}
	\end{beamercolorbox}
\end{frame}

\end{document}
