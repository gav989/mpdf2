\documentclass[13.5pt, varwidth=true]{beamer} 
\usepackage[scaled=1]{helvet}
\usepackage[T1]{fontenc}
\usepackage{overpic}
\usepackage{sansmath}
\usepackage{multicol}
\usepackage{gnuplottex}
\sansmath
\renewcommand\familydefault{\sfdefault}
%\DeclareMathSizes{11}{19}{13}{9}   % For size 11 text
\usefonttheme{professionalfonts}
\usepackage{colortbl}
\usepackage{xcolor}
\usepackage{booktabs}
\usepackage{fancybox}
\usepackage{relsize}
\usepackage{siunitx}
\sisetup{detect-all}
\definecolor{que1}{HTML}{FFC5FA}
\definecolor{que2}{HTML}{C6D7FD}
\definecolor{bgre}{HTML}{DCDCDC}
\definecolor{tgre}{HTML}{BCBCBC}
\definecolor{ared}{HTML}{850000}
\setbeamercolor{background canvas}{bg=bgre}
\setbeamercolor{que1}{fg=black,bg=que1}
\setbeamercolor{que2}{fg=black,bg=que2}
\usetheme{CambridgeUS}
\usecolortheme{beetle}
\setbeamertemplate{navigation symbols}{}
\setbeamertemplate{footline}{\centerline{\textcolor{tgre}{This work is licensed under a Creative Commons Attribution-ShareAlike 4.0 International License}}}
\usepackage{graphicx}
\setbeamercolor*{palette tertiary}{bg=que1}
\setbeamercolor*{palette primary}{bg=que2}
\begin{document}
\date{}

\section[Graphs]{what is this bit?}
\subsection[title]{and this?}

\begin{frame}[shrink=19,fragile]
	\begin{beamercolorbox}[rounded=true, left, shadow=true,wd=14.8cm]{que1}
		 Complete this table for $y = x^{2} + 4$ and draw the graph. \\[0.3cm] \renewcommand{\arraystretch}{1.2}\newcolumntype{C}{>{\centering\arraybackslash} m{1cm} }\begin{tabular}{|C|C|C|C|C|C|C|C|} \hline x & -2 & -1 & 0 & 1 & 2 & 3 & 4 \\ \hline y &  & 5 & 4 &  & 8 & 13 & \\ \hline \end{tabular}\\[0.3cm]
	\end{beamercolorbox}
\end{frame}
\begin{frame}[shrink=19,fragile]
	\begin{beamercolorbox}[rounded=true, left, shadow=true,wd=14.8cm]{que1}
		\renewcommand{\arraystretch}{1.2}\newcolumntype{C}{>{\centering\arraybackslash} m{1cm} }\begin{tabular}{|C|C|C|C|C|C|C|C|} \hline x & -2 & -1 & 0 & 1 & 2 & 3 & 4 \\ \hline y & 8 & 5 & 4 & 5 & 8 & 13 & 20\\ \hline \end{tabular}\begin{gnuplot}[terminal=pdf] set nokey; set grid; set border 0; set xtics axis 1; set ytics axis 1; set size ratio 2; set size 1,1; set zeroaxis linewidth 2 linetype 1; set yrange [-1:20]; set xrange [-2:4]; plot 1*x**2+0*x+4 \end{gnuplot}
	\end{beamercolorbox}
\end{frame}

\begin{frame}[shrink=19,fragile]
	\begin{beamercolorbox}[rounded=true, left, shadow=true,wd=14.8cm]{que2}
		 Complete this table for $y = x^{2} + 5$ and draw the graph. \\[0.3cm] \renewcommand{\arraystretch}{1.2}\newcolumntype{C}{>{\centering\arraybackslash} m{1cm} }\begin{tabular}{|C|C|C|C|C|C|C|C|} \hline x & -4 & -3 & -2 & -1 & 0 & 1 & 2 \\ \hline y &  & 14 & 9 &  & 5 & 6 & \\ \hline \end{tabular}\\[0.3cm]
	\end{beamercolorbox}
\end{frame}
\begin{frame}[shrink=19,fragile]
	\begin{beamercolorbox}[rounded=true, left, shadow=true,wd=14.8cm]{que2}
 		\renewcommand{\arraystretch}{1.2}\newcolumntype{C}{>{\centering\arraybackslash} m{1cm} }\begin{tabular}{|C|C|C|C|C|C|C|C|} \hline x & -4 & -3 & -2 & -1 & 0 & 1 & 2 \\ \hline y & 21 & 14 & 9 & 6 & 5 & 6 & 9\\ \hline \end{tabular}\begin{gnuplot}[terminal=pdf] set nokey; set grid; set border 0; set xtics axis 1; set ytics axis 1; set size ratio 2; set size 1,1; set zeroaxis linewidth 2 linetype 1; set yrange [-1:21]; set xrange [-4:2]; plot 1*x**2+0*x+5 \end{gnuplot}
	\end{beamercolorbox}
\end{frame}

\begin{frame}[shrink=19,fragile]
	\begin{beamercolorbox}[rounded=true, left, shadow=true,wd=14.8cm]{que1}
		 Complete this table for $y = x^{2} - 3$ and draw the graph. \\[0.3cm] \renewcommand{\arraystretch}{1.2}\newcolumntype{C}{>{\centering\arraybackslash} m{1cm} }\begin{tabular}{|C|C|C|C|C|C|C|C|} \hline x & -4 & -3 & -2 & -1 & 0 & 1 & 2 \\ \hline y &  & 6 & 1 & -2 &  & -2 & \\ \hline \end{tabular}\\[0.3cm]
	\end{beamercolorbox}
\end{frame}
\begin{frame}[shrink=19,fragile]
	\begin{beamercolorbox}[rounded=true, left, shadow=true,wd=14.8cm]{que1}
		\renewcommand{\arraystretch}{1.2}\newcolumntype{C}{>{\centering\arraybackslash} m{1cm} }\begin{tabular}{|C|C|C|C|C|C|C|C|} \hline x & -4 & -3 & -2 & -1 & 0 & 1 & 2 \\ \hline y & 13 & 6 & 1 & -2 & -3 & -2 & 1\\ \hline \end{tabular}\begin{gnuplot}[terminal=pdf] set nokey; set grid; set border 0; set xtics axis 1; set ytics axis 1; set size ratio 2; set size 1,1; set zeroaxis linewidth 2 linetype 1; set yrange [-3:13]; set xrange [-4:2]; plot 1*x**2+0*x+-3 \end{gnuplot}
	\end{beamercolorbox}
\end{frame}

\begin{frame}[shrink=19,fragile]
	\begin{beamercolorbox}[rounded=true, left, shadow=true,wd=14.8cm]{que2}
		 Complete this table for $y = x^{2}$ and draw the graph. \\[0.3cm] \renewcommand{\arraystretch}{1.2}\newcolumntype{C}{>{\centering\arraybackslash} m{1cm} }\begin{tabular}{|C|C|C|C|C|C|C|C|} \hline x & -2 & -1 & 0 & 1 & 2 & 3 & 4 \\ \hline y &  & 1 &  & 1 & 4 & 9 & \\ \hline \end{tabular}\\[0.3cm]
	\end{beamercolorbox}
\end{frame}
\begin{frame}[shrink=19,fragile]
	\begin{beamercolorbox}[rounded=true, left, shadow=true,wd=14.8cm]{que2}
		\renewcommand{\arraystretch}{1.2}\newcolumntype{C}{>{\centering\arraybackslash} m{1cm} }\begin{tabular}{|C|C|C|C|C|C|C|C|} \hline x & -2 & -1 & 0 & 1 & 2 & 3 & 4 \\ \hline y & 4 & 1 & 0 & 1 & 4 & 9 & 16\\ \hline \end{tabular}\begin{gnuplot}[terminal=pdf] set nokey; set grid; set border 0; set xtics axis 1; set ytics axis 1; set size ratio 2; set size 1,1; set zeroaxis linewidth 2 linetype 1; set yrange [-1:16]; set xrange [-2:4]; plot 1*x**2+0*x+0 \end{gnuplot}
	\end{beamercolorbox}
\end{frame}

\begin{frame}[shrink=19,fragile]
	\begin{beamercolorbox}[rounded=true, left, shadow=true,wd=14.8cm]{que1}
		 Complete this table for $y = x^{2} - 4$ and draw the graph. \\[0.3cm] \renewcommand{\arraystretch}{1.2}\newcolumntype{C}{>{\centering\arraybackslash} m{1cm} }\begin{tabular}{|C|C|C|C|C|C|C|C|} \hline x & -4 & -3 & -2 & -1 & 0 & 1 & 2 \\ \hline y & 12 &  & 0 & -3 &  &  & 0\\ \hline \end{tabular}\\[0.3cm]
	\end{beamercolorbox}
\end{frame}
\begin{frame}[shrink=19,fragile]
	\begin{beamercolorbox}[rounded=true, left, shadow=true,wd=14.8cm]{que1}
		\renewcommand{\arraystretch}{1.2}\newcolumntype{C}{>{\centering\arraybackslash} m{1cm} }\begin{tabular}{|C|C|C|C|C|C|C|C|} \hline x & -4 & -3 & -2 & -1 & 0 & 1 & 2 \\ \hline y & 12 & 5 & 0 & -3 & -4 & -3 & 0\\ \hline \end{tabular}\begin{gnuplot}[terminal=pdf] set nokey; set grid; set border 0; set xtics axis 1; set ytics axis 1; set size ratio 2; set size 1,1; set zeroaxis linewidth 2 linetype 1; set yrange [-4:12]; set xrange [-4:2]; plot 1*x**2+0*x+-4 \end{gnuplot}
	\end{beamercolorbox}
\end{frame}

\begin{frame}[shrink=19,fragile]
	\begin{beamercolorbox}[rounded=true, left, shadow=true,wd=14.8cm]{que2}
		 Complete this table for $y = x^{2} - 1$ and draw the graph. \\[0.3cm] \renewcommand{\arraystretch}{1.2}\newcolumntype{C}{>{\centering\arraybackslash} m{1cm} }\begin{tabular}{|C|C|C|C|C|C|C|C|} \hline x & -4 & -3 & -2 & -1 & 0 & 1 & 2 \\ \hline y & 15 &  & 3 &  & -1 &  & 3\\ \hline \end{tabular}\\[0.3cm]
	\end{beamercolorbox}
\end{frame}
\begin{frame}[shrink=19,fragile]
	\begin{beamercolorbox}[rounded=true, left, shadow=true,wd=14.8cm]{que2}
		\renewcommand{\arraystretch}{1.2}\newcolumntype{C}{>{\centering\arraybackslash} m{1cm} }\begin{tabular}{|C|C|C|C|C|C|C|C|} \hline x & -4 & -3 & -2 & -1 & 0 & 1 & 2 \\ \hline y & 15 & 8 & 3 & 0 & -1 & 0 & 3\\ \hline \end{tabular}\begin{gnuplot}[terminal=pdf] set nokey; set grid; set border 0; set xtics axis 1; set ytics axis 1; set size ratio 2; set size 1,1; set zeroaxis linewidth 2 linetype 1; set yrange [-1:15]; set xrange [-4:2]; plot 1*x**2+0*x+-1 \end{gnuplot}
	\end{beamercolorbox}
\end{frame}

\begin{frame}[shrink=19,fragile]
	\begin{beamercolorbox}[rounded=true, left, shadow=true,wd=14.8cm]{que1}
		 Complete this table for $y = x^{2} + 4$ and draw the graph. \\[0.3cm] \renewcommand{\arraystretch}{1.2}\newcolumntype{C}{>{\centering\arraybackslash} m{1cm} }\begin{tabular}{|C|C|C|C|C|C|C|C|} \hline x & -4 & -3 & -2 & -1 & 0 & 1 & 2 \\ \hline y & 20 &  & 8 & 5 &  & 5 & \\ \hline \end{tabular}\\[0.3cm]
	\end{beamercolorbox}
\end{frame}
\begin{frame}[shrink=19,fragile]
	\begin{beamercolorbox}[rounded=true, left, shadow=true,wd=14.8cm]{que1}
		\renewcommand{\arraystretch}{1.2}\newcolumntype{C}{>{\centering\arraybackslash} m{1cm} }\begin{tabular}{|C|C|C|C|C|C|C|C|} \hline x & -4 & -3 & -2 & -1 & 0 & 1 & 2 \\ \hline y & 20 & 13 & 8 & 5 & 4 & 5 & 8\\ \hline \end{tabular}\begin{gnuplot}[terminal=pdf] set nokey; set grid; set border 0; set xtics axis 1; set ytics axis 1; set size ratio 2; set size 1,1; set zeroaxis linewidth 2 linetype 1; set yrange [-1:20]; set xrange [-4:2]; plot 1*x**2+0*x+4 \end{gnuplot}
	\end{beamercolorbox}
\end{frame}

\begin{frame}[shrink=19,fragile]
	\begin{beamercolorbox}[rounded=true, left, shadow=true,wd=14.8cm]{que2}
		 Complete this table for $y = x^{2} - 1$ and draw the graph. \\[0.3cm] \renewcommand{\arraystretch}{1.2}\newcolumntype{C}{>{\centering\arraybackslash} m{1cm} }\begin{tabular}{|C|C|C|C|C|C|C|C|} \hline x & -4 & -3 & -2 & -1 & 0 & 1 & 2 \\ \hline y & 15 &  &  & 0 & -1 & 0 & \\ \hline \end{tabular}\\[0.3cm]
	\end{beamercolorbox}
\end{frame}
\begin{frame}[shrink=19,fragile]
	\begin{beamercolorbox}[rounded=true, left, shadow=true,wd=14.8cm]{que2}
		\renewcommand{\arraystretch}{1.2}\newcolumntype{C}{>{\centering\arraybackslash} m{1cm} }\begin{tabular}{|C|C|C|C|C|C|C|C|} \hline x & -4 & -3 & -2 & -1 & 0 & 1 & 2 \\ \hline y & 15 & 8 & 3 & 0 & -1 & 0 & 3\\ \hline \end{tabular}\begin{gnuplot}[terminal=pdf] set nokey; set grid; set border 0; set xtics axis 1; set ytics axis 1; set size ratio 2; set size 1,1; set zeroaxis linewidth 2 linetype 1; set yrange [-1:15]; set xrange [-4:2]; plot 1*x**2+0*x+-1 \end{gnuplot}
	\end{beamercolorbox}
\end{frame}

\begin{frame}[shrink=19,fragile]
	\begin{beamercolorbox}[rounded=true, left, shadow=true,wd=14.8cm]{que1}
		 Complete this table for $y = x^{2}$ and draw the graph. \\[0.3cm] \renewcommand{\arraystretch}{1.2}\newcolumntype{C}{>{\centering\arraybackslash} m{1cm} }\begin{tabular}{|C|C|C|C|C|C|C|C|} \hline x & -2 & -1 & 0 & 1 & 2 & 3 & 4 \\ \hline y & 4 &  & 0 & 1 &  & 9 & \\ \hline \end{tabular}\\[0.3cm]
	\end{beamercolorbox}
\end{frame}
\begin{frame}[shrink=19,fragile]
	\begin{beamercolorbox}[rounded=true, left, shadow=true,wd=14.8cm]{que1}
		\renewcommand{\arraystretch}{1.2}\newcolumntype{C}{>{\centering\arraybackslash} m{1cm} }\begin{tabular}{|C|C|C|C|C|C|C|C|} \hline x & -2 & -1 & 0 & 1 & 2 & 3 & 4 \\ \hline y & 4 & 1 & 0 & 1 & 4 & 9 & 16\\ \hline \end{tabular}\begin{gnuplot}[terminal=pdf] set nokey; set grid; set border 0; set xtics axis 1; set ytics axis 1; set size ratio 2; set size 1,1; set zeroaxis linewidth 2 linetype 1; set yrange [-1:16]; set xrange [-2:4]; plot 1*x**2+0*x+0 \end{gnuplot}
	\end{beamercolorbox}
\end{frame}

\begin{frame}[shrink=19,fragile]
	\begin{beamercolorbox}[rounded=true, left, shadow=true,wd=14.8cm]{que2}
		 Complete this table for $y = x^{2} - 2$ and draw the graph. \\[0.3cm] \renewcommand{\arraystretch}{1.2}\newcolumntype{C}{>{\centering\arraybackslash} m{1cm} }\begin{tabular}{|C|C|C|C|C|C|C|C|} \hline x & -4 & -3 & -2 & -1 & 0 & 1 & 2 \\ \hline y & 14 &  & 2 & -1 &  &  & 2\\ \hline \end{tabular}\\[0.3cm]
	\end{beamercolorbox}
\end{frame}
\begin{frame}[shrink=19,fragile]
	\begin{beamercolorbox}[rounded=true, left, shadow=true,wd=14.8cm]{que2}
		\renewcommand{\arraystretch}{1.2}\newcolumntype{C}{>{\centering\arraybackslash} m{1cm} }\begin{tabular}{|C|C|C|C|C|C|C|C|} \hline x & -4 & -3 & -2 & -1 & 0 & 1 & 2 \\ \hline y & 14 & 7 & 2 & -1 & -2 & -1 & 2\\ \hline \end{tabular}\begin{gnuplot}[terminal=pdf] set nokey; set grid; set border 0; set xtics axis 1; set ytics axis 1; set size ratio 2; set size 1,1; set zeroaxis linewidth 2 linetype 1; set yrange [-2:14]; set xrange [-4:2]; plot 1*x**2+0*x+-2 \end{gnuplot}
	\end{beamercolorbox}
\end{frame}

\begin{frame}[shrink=19,fragile]
	\begin{beamercolorbox}[rounded=true, left, shadow=true,wd=14.8cm]{que1}
		 Complete this table for $y = x^{2} + 5$ and draw the graph. \\[0.3cm] \renewcommand{\arraystretch}{1.2}\newcolumntype{C}{>{\centering\arraybackslash} m{1cm} }\begin{tabular}{|C|C|C|C|C|C|C|C|} \hline x & -2 & -1 & 0 & 1 & 2 & 3 & 4 \\ \hline y & 9 &  &  & 6 & 9 & 14 & \\ \hline \end{tabular}\\[0.3cm]
	\end{beamercolorbox}
\end{frame}
\begin{frame}[shrink=19,fragile]
	\begin{beamercolorbox}[rounded=true, left, shadow=true,wd=14.8cm]{que1}
		\renewcommand{\arraystretch}{1.2}\newcolumntype{C}{>{\centering\arraybackslash} m{1cm} }\begin{tabular}{|C|C|C|C|C|C|C|C|} \hline x & -2 & -1 & 0 & 1 & 2 & 3 & 4 \\ \hline y & 9 & 6 & 5 & 6 & 9 & 14 & 21\\ \hline \end{tabular}\begin{gnuplot}[terminal=pdf] set nokey; set grid; set border 0; set xtics axis 1; set ytics axis 1; set size ratio 2; set size 1,1; set zeroaxis linewidth 2 linetype 1; set yrange [-1:21]; set xrange [-2:4]; plot 1*x**2+0*x+5 \end{gnuplot}
	\end{beamercolorbox}
\end{frame}

\begin{frame}[shrink=19,fragile]
	\begin{beamercolorbox}[rounded=true, left, shadow=true,wd=14.8cm]{que2}
		 Complete this table for $y = x^{2} + 5$ and draw the graph. \\[0.3cm] \renewcommand{\arraystretch}{1.2}\newcolumntype{C}{>{\centering\arraybackslash} m{1cm} }\begin{tabular}{|C|C|C|C|C|C|C|C|} \hline x & -4 & -3 & -2 & -1 & 0 & 1 & 2 \\ \hline y & 21 &  &  & 6 & 5 &  & 9\\ \hline \end{tabular}\\[0.3cm]
	\end{beamercolorbox}
\end{frame}
\begin{frame}[shrink=19,fragile]
	\begin{beamercolorbox}[rounded=true, left, shadow=true,wd=14.8cm]{que2}
		\renewcommand{\arraystretch}{1.2}\newcolumntype{C}{>{\centering\arraybackslash} m{1cm} }\begin{tabular}{|C|C|C|C|C|C|C|C|} \hline x & -4 & -3 & -2 & -1 & 0 & 1 & 2 \\ \hline y & 21 & 14 & 9 & 6 & 5 & 6 & 9\\ \hline \end{tabular}\begin{gnuplot}[terminal=pdf] set nokey; set grid; set border 0; set xtics axis 1; set ytics axis 1; set size ratio 2; set size 1,1; set zeroaxis linewidth 2 linetype 1; set yrange [-1:21]; set xrange [-4:2]; plot 1*x**2+0*x+5 \end{gnuplot}
	\end{beamercolorbox}
\end{frame}

\begin{frame}[shrink=19,fragile]
	\begin{beamercolorbox}[rounded=true, left, shadow=true,wd=14.8cm]{que1}
		 Complete this table for $y = x^{2} + 4$ and draw the graph. \\[0.3cm] \renewcommand{\arraystretch}{1.2}\newcolumntype{C}{>{\centering\arraybackslash} m{1cm} }\begin{tabular}{|C|C|C|C|C|C|C|C|} \hline x & -4 & -3 & -2 & -1 & 0 & 1 & 2 \\ \hline y &  & 13 &  & 5 & 4 & 5 & \\ \hline \end{tabular}\\[0.3cm]
	\end{beamercolorbox}
\end{frame}
\begin{frame}[shrink=19,fragile]
	\begin{beamercolorbox}[rounded=true, left, shadow=true,wd=14.8cm]{que1}
		\renewcommand{\arraystretch}{1.2}\newcolumntype{C}{>{\centering\arraybackslash} m{1cm} }\begin{tabular}{|C|C|C|C|C|C|C|C|} \hline x & -4 & -3 & -2 & -1 & 0 & 1 & 2 \\ \hline y & 20 & 13 & 8 & 5 & 4 & 5 & 8\\ \hline \end{tabular}\begin{gnuplot}[terminal=pdf] set nokey; set grid; set border 0; set xtics axis 1; set ytics axis 1; set size ratio 2; set size 1,1; set zeroaxis linewidth 2 linetype 1; set yrange [-1:20]; set xrange [-4:2]; plot 1*x**2+0*x+4 \end{gnuplot}
	\end{beamercolorbox}
\end{frame}

\begin{frame}[shrink=19,fragile]
	\begin{beamercolorbox}[rounded=true, left, shadow=true,wd=14.8cm]{que2}
		 Complete this table for $y = x^{2} - 3$ and draw the graph. \\[0.3cm] \renewcommand{\arraystretch}{1.2}\newcolumntype{C}{>{\centering\arraybackslash} m{1cm} }\begin{tabular}{|C|C|C|C|C|C|C|C|} \hline x & -4 & -3 & -2 & -1 & 0 & 1 & 2 \\ \hline y &  & 6 & 1 & -2 &  & -2 & \\ \hline \end{tabular}\\[0.3cm]
	\end{beamercolorbox}
\end{frame}
\begin{frame}[shrink=19,fragile]
	\begin{beamercolorbox}[rounded=true, left, shadow=true,wd=14.8cm]{que2}
		\renewcommand{\arraystretch}{1.2}\newcolumntype{C}{>{\centering\arraybackslash} m{1cm} }\begin{tabular}{|C|C|C|C|C|C|C|C|} \hline x & -4 & -3 & -2 & -1 & 0 & 1 & 2 \\ \hline y & 13 & 6 & 1 & -2 & -3 & -2 & 1\\ \hline \end{tabular}\begin{gnuplot}[terminal=pdf] set nokey; set grid; set border 0; set xtics axis 1; set ytics axis 1; set size ratio 2; set size 1,1; set zeroaxis linewidth 2 linetype 1; set yrange [-3:13]; set xrange [-4:2]; plot 1*x**2+0*x+-3 \end{gnuplot}
	\end{beamercolorbox}
\end{frame}

\begin{frame}[shrink=19,fragile]
	\begin{beamercolorbox}[rounded=true, left, shadow=true,wd=14.8cm]{que1}
		 Complete this table for $y = x^{2} + 5$ and draw the graph. \\[0.3cm] \renewcommand{\arraystretch}{1.2}\newcolumntype{C}{>{\centering\arraybackslash} m{1cm} }\begin{tabular}{|C|C|C|C|C|C|C|C|} \hline x & -4 & -3 & -2 & -1 & 0 & 1 & 2 \\ \hline y &  & 14 &  & 6 & 5 &  & 9\\ \hline \end{tabular}\\[0.3cm]
	\end{beamercolorbox}
\end{frame}
\begin{frame}[shrink=19,fragile]
	\begin{beamercolorbox}[rounded=true, left, shadow=true,wd=14.8cm]{que1}
		\renewcommand{\arraystretch}{1.2}\newcolumntype{C}{>{\centering\arraybackslash} m{1cm} }\begin{tabular}{|C|C|C|C|C|C|C|C|} \hline x & -4 & -3 & -2 & -1 & 0 & 1 & 2 \\ \hline y & 21 & 14 & 9 & 6 & 5 & 6 & 9\\ \hline \end{tabular}\begin{gnuplot}[terminal=pdf] set nokey; set grid; set border 0; set xtics axis 1; set ytics axis 1; set size ratio 2; set size 1,1; set zeroaxis linewidth 2 linetype 1; set yrange [-1:21]; set xrange [-4:2]; plot 1*x**2+0*x+5 \end{gnuplot}
	\end{beamercolorbox}
\end{frame}

\begin{frame}[shrink=19,fragile]
	\begin{beamercolorbox}[rounded=true, left, shadow=true,wd=14.8cm]{que2}
		 Complete this table for $y = x^{2} + 2$ and draw the graph. \\[0.3cm] \renewcommand{\arraystretch}{1.2}\newcolumntype{C}{>{\centering\arraybackslash} m{1cm} }\begin{tabular}{|C|C|C|C|C|C|C|C|} \hline x & -2 & -1 & 0 & 1 & 2 & 3 & 4 \\ \hline y &  & 3 & 2 & 3 &  &  & 18\\ \hline \end{tabular}\\[0.3cm]
	\end{beamercolorbox}
\end{frame}
\begin{frame}[shrink=19,fragile]
	\begin{beamercolorbox}[rounded=true, left, shadow=true,wd=14.8cm]{que2}
		\renewcommand{\arraystretch}{1.2}\newcolumntype{C}{>{\centering\arraybackslash} m{1cm} }\begin{tabular}{|C|C|C|C|C|C|C|C|} \hline x & -2 & -1 & 0 & 1 & 2 & 3 & 4 \\ \hline y & 6 & 3 & 2 & 3 & 6 & 11 & 18\\ \hline \end{tabular}\begin{gnuplot}[terminal=pdf] set nokey; set grid; set border 0; set xtics axis 1; set ytics axis 1; set size ratio 2; set size 1,1; set zeroaxis linewidth 2 linetype 1; set yrange [-1:18]; set xrange [-2:4]; plot 1*x**2+0*x+2 \end{gnuplot}
	\end{beamercolorbox}
\end{frame}

\begin{frame}[shrink=19,fragile]
	\begin{beamercolorbox}[rounded=true, left, shadow=true,wd=14.8cm]{que1}
		 Complete this table for $y = x^{2} + 2$ and draw the graph. \\[0.3cm] \renewcommand{\arraystretch}{1.2}\newcolumntype{C}{>{\centering\arraybackslash} m{1cm} }\begin{tabular}{|C|C|C|C|C|C|C|C|} \hline x & -2 & -1 & 0 & 1 & 2 & 3 & 4 \\ \hline y & 6 &  & 2 & 3 &  & 11 & \\ \hline \end{tabular}\\[0.3cm]
	\end{beamercolorbox}
\end{frame}
\begin{frame}[shrink=19,fragile]
	\begin{beamercolorbox}[rounded=true, left, shadow=true,wd=14.8cm]{que1}
		\renewcommand{\arraystretch}{1.2}\newcolumntype{C}{>{\centering\arraybackslash} m{1cm} }\begin{tabular}{|C|C|C|C|C|C|C|C|} \hline x & -2 & -1 & 0 & 1 & 2 & 3 & 4 \\ \hline y & 6 & 3 & 2 & 3 & 6 & 11 & 18\\ \hline \end{tabular}\begin{gnuplot}[terminal=pdf] set nokey; set grid; set border 0; set xtics axis 1; set ytics axis 1; set size ratio 2; set size 1,1; set zeroaxis linewidth 2 linetype 1; set yrange [-1:18]; set xrange [-2:4]; plot 1*x**2+0*x+2 \end{gnuplot}
	\end{beamercolorbox}
\end{frame}

\begin{frame}[shrink=19,fragile]
	\begin{beamercolorbox}[rounded=true, left, shadow=true,wd=14.8cm]{que2}
		 Complete this table for $y = x^{2} + 5$ and draw the graph. \\[0.3cm] \renewcommand{\arraystretch}{1.2}\newcolumntype{C}{>{\centering\arraybackslash} m{1cm} }\begin{tabular}{|C|C|C|C|C|C|C|C|} \hline x & -4 & -3 & -2 & -1 & 0 & 1 & 2 \\ \hline y & 21 &  & 9 & 6 &  & 6 & \\ \hline \end{tabular}\\[0.3cm]
	\end{beamercolorbox}
\end{frame}
\begin{frame}[shrink=19,fragile]
	\begin{beamercolorbox}[rounded=true, left, shadow=true,wd=14.8cm]{que2}
		\renewcommand{\arraystretch}{1.2}\newcolumntype{C}{>{\centering\arraybackslash} m{1cm} }\begin{tabular}{|C|C|C|C|C|C|C|C|} \hline x & -4 & -3 & -2 & -1 & 0 & 1 & 2 \\ \hline y & 21 & 14 & 9 & 6 & 5 & 6 & 9\\ \hline \end{tabular}\begin{gnuplot}[terminal=pdf] set nokey; set grid; set border 0; set xtics axis 1; set ytics axis 1; set size ratio 2; set size 1,1; set zeroaxis linewidth 2 linetype 1; set yrange [-1:21]; set xrange [-4:2]; plot 1*x**2+0*x+5 \end{gnuplot}
	\end{beamercolorbox}
\end{frame}

\begin{frame}[shrink=19,fragile]
	\begin{beamercolorbox}[rounded=true, left, shadow=true,wd=14.8cm]{que1}
		 Complete this table for $y = x^{2} + 1$ and draw the graph. \\[0.3cm] \renewcommand{\arraystretch}{1.2}\newcolumntype{C}{>{\centering\arraybackslash} m{1cm} }\begin{tabular}{|C|C|C|C|C|C|C|C|} \hline x & -4 & -3 & -2 & -1 & 0 & 1 & 2 \\ \hline y & 17 &  & 5 &  & 1 &  & 5\\ \hline \end{tabular}\\[0.3cm]
	\end{beamercolorbox}
\end{frame}
\begin{frame}[shrink=19,fragile]
	\begin{beamercolorbox}[rounded=true, left, shadow=true,wd=14.8cm]{que1}
		\renewcommand{\arraystretch}{1.2}\newcolumntype{C}{>{\centering\arraybackslash} m{1cm} }\begin{tabular}{|C|C|C|C|C|C|C|C|} \hline x & -4 & -3 & -2 & -1 & 0 & 1 & 2 \\ \hline y & 17 & 10 & 5 & 2 & 1 & 2 & 5\\ \hline \end{tabular}\begin{gnuplot}[terminal=pdf] set nokey; set grid; set border 0; set xtics axis 1; set ytics axis 1; set size ratio 2; set size 1,1; set zeroaxis linewidth 2 linetype 1; set yrange [-1:17]; set xrange [-4:2]; plot 1*x**2+0*x+1 \end{gnuplot}
	\end{beamercolorbox}
\end{frame}

\begin{frame}[shrink=19,fragile]
	\begin{beamercolorbox}[rounded=true, left, shadow=true,wd=14.8cm]{que2}
		 Complete this table for $y = x^{2} - 5$ and draw the graph. \\[0.3cm] \renewcommand{\arraystretch}{1.2}\newcolumntype{C}{>{\centering\arraybackslash} m{1cm} }\begin{tabular}{|C|C|C|C|C|C|C|C|} \hline x & -2 & -1 & 0 & 1 & 2 & 3 & 4 \\ \hline y & -1 &  &  & -4 & -1 & 4 & \\ \hline \end{tabular}\\[0.3cm]
	\end{beamercolorbox}
\end{frame}
\begin{frame}[shrink=19,fragile]
	\begin{beamercolorbox}[rounded=true, left, shadow=true,wd=14.8cm]{que2}
		\renewcommand{\arraystretch}{1.2}\newcolumntype{C}{>{\centering\arraybackslash} m{1cm} }\begin{tabular}{|C|C|C|C|C|C|C|C|} \hline x & -2 & -1 & 0 & 1 & 2 & 3 & 4 \\ \hline y & -1 & -4 & -5 & -4 & -1 & 4 & 11\\ \hline \end{tabular}\begin{gnuplot}[terminal=pdf] set nokey; set grid; set border 0; set xtics axis 1; set ytics axis 1; set size ratio 2; set size 1,1; set zeroaxis linewidth 2 linetype 1; set yrange [-5:11]; set xrange [-2:4]; plot 1*x**2+0*x+-5 \end{gnuplot}
	\end{beamercolorbox}
\end{frame}

\begin{frame}[shrink=19,fragile]
	\begin{beamercolorbox}[rounded=true, left, shadow=true,wd=14.8cm]{que1}
		 Complete this table for $y = x^{2} + 3$ and draw the graph. \\[0.3cm] \renewcommand{\arraystretch}{1.2}\newcolumntype{C}{>{\centering\arraybackslash} m{1cm} }\begin{tabular}{|C|C|C|C|C|C|C|C|} \hline x & -4 & -3 & -2 & -1 & 0 & 1 & 2 \\ \hline y &  & 12 & 7 &  & 3 &  & 7\\ \hline \end{tabular}\\[0.3cm]
	\end{beamercolorbox}
\end{frame}
\begin{frame}[shrink=19,fragile]
	\begin{beamercolorbox}[rounded=true, left, shadow=true,wd=14.8cm]{que1}
		\renewcommand{\arraystretch}{1.2}\newcolumntype{C}{>{\centering\arraybackslash} m{1cm} }\begin{tabular}{|C|C|C|C|C|C|C|C|} \hline x & -4 & -3 & -2 & -1 & 0 & 1 & 2 \\ \hline y & 19 & 12 & 7 & 4 & 3 & 4 & 7\\ \hline \end{tabular}\begin{gnuplot}[terminal=pdf] set nokey; set grid; set border 0; set xtics axis 1; set ytics axis 1; set size ratio 2; set size 1,1; set zeroaxis linewidth 2 linetype 1; set yrange [-1:19]; set xrange [-4:2]; plot 1*x**2+0*x+3 \end{gnuplot}
	\end{beamercolorbox}
\end{frame}

\begin{frame}[shrink=19,fragile]
	\begin{beamercolorbox}[rounded=true, left, shadow=true,wd=14.8cm]{que2}
		 Complete this table for $y = x^{2} + 3$ and draw the graph. \\[0.3cm] \renewcommand{\arraystretch}{1.2}\newcolumntype{C}{>{\centering\arraybackslash} m{1cm} }\begin{tabular}{|C|C|C|C|C|C|C|C|} \hline x & -4 & -3 & -2 & -1 & 0 & 1 & 2 \\ \hline y & 19 &  & 7 &  & 3 &  & 7\\ \hline \end{tabular}\\[0.3cm]
	\end{beamercolorbox}
\end{frame}
\begin{frame}[shrink=19,fragile]
	\begin{beamercolorbox}[rounded=true, left, shadow=true,wd=14.8cm]{que2}
		\renewcommand{\arraystretch}{1.2}\newcolumntype{C}{>{\centering\arraybackslash} m{1cm} }\begin{tabular}{|C|C|C|C|C|C|C|C|} \hline x & -4 & -3 & -2 & -1 & 0 & 1 & 2 \\ \hline y & 19 & 12 & 7 & 4 & 3 & 4 & 7\\ \hline \end{tabular}\begin{gnuplot}[terminal=pdf] set nokey; set grid; set border 0; set xtics axis 1; set ytics axis 1; set size ratio 2; set size 1,1; set zeroaxis linewidth 2 linetype 1; set yrange [-1:19]; set xrange [-4:2]; plot 1*x**2+0*x+3 \end{gnuplot}
	\end{beamercolorbox}
\end{frame}

\begin{frame}[shrink=19,fragile]
	\begin{beamercolorbox}[rounded=true, left, shadow=true,wd=14.8cm]{que1}
		 Complete this table for $y = x^{2}$ and draw the graph. \\[0.3cm] \renewcommand{\arraystretch}{1.2}\newcolumntype{C}{>{\centering\arraybackslash} m{1cm} }\begin{tabular}{|C|C|C|C|C|C|C|C|} \hline x & -4 & -3 & -2 & -1 & 0 & 1 & 2 \\ \hline y & 16 &  &  & 1 & 0 &  & 4\\ \hline \end{tabular}\\[0.3cm]
	\end{beamercolorbox}
\end{frame}
\begin{frame}[shrink=19,fragile]
	\begin{beamercolorbox}[rounded=true, left, shadow=true,wd=14.8cm]{que1}
		\renewcommand{\arraystretch}{1.2}\newcolumntype{C}{>{\centering\arraybackslash} m{1cm} }\begin{tabular}{|C|C|C|C|C|C|C|C|} \hline x & -4 & -3 & -2 & -1 & 0 & 1 & 2 \\ \hline y & 16 & 9 & 4 & 1 & 0 & 1 & 4\\ \hline \end{tabular}\begin{gnuplot}[terminal=pdf] set nokey; set grid; set border 0; set xtics axis 1; set ytics axis 1; set size ratio 2; set size 1,1; set zeroaxis linewidth 2 linetype 1; set yrange [-1:16]; set xrange [-4:2]; plot 1*x**2+0*x+0 \end{gnuplot}
	\end{beamercolorbox}
\end{frame}

\begin{frame}[shrink=19,fragile]
	\begin{beamercolorbox}[rounded=true, left, shadow=true,wd=14.8cm]{que2}
		 Complete this table for $y = x^{2} - 2$ and draw the graph. \\[0.3cm] \renewcommand{\arraystretch}{1.2}\newcolumntype{C}{>{\centering\arraybackslash} m{1cm} }\begin{tabular}{|C|C|C|C|C|C|C|C|} \hline x & -4 & -3 & -2 & -1 & 0 & 1 & 2 \\ \hline y &  & 7 & 2 &  & -2 &  & 2\\ \hline \end{tabular}\\[0.3cm]
	\end{beamercolorbox}
\end{frame}
\begin{frame}[shrink=19,fragile]
	\begin{beamercolorbox}[rounded=true, left, shadow=true,wd=14.8cm]{que2}
		\renewcommand{\arraystretch}{1.2}\newcolumntype{C}{>{\centering\arraybackslash} m{1cm} }\begin{tabular}{|C|C|C|C|C|C|C|C|} \hline x & -4 & -3 & -2 & -1 & 0 & 1 & 2 \\ \hline y & 14 & 7 & 2 & -1 & -2 & -1 & 2\\ \hline \end{tabular}\begin{gnuplot}[terminal=pdf] set nokey; set grid; set border 0; set xtics axis 1; set ytics axis 1; set size ratio 2; set size 1,1; set zeroaxis linewidth 2 linetype 1; set yrange [-2:14]; set xrange [-4:2]; plot 1*x**2+0*x+-2 \end{gnuplot}
	\end{beamercolorbox}
\end{frame}

\begin{frame}[shrink=19,fragile]
	\begin{beamercolorbox}[rounded=true, left, shadow=true,wd=14.8cm]{que1}
		 Complete this table for $y = x^{2} + 1$ and draw the graph. \\[0.3cm] \renewcommand{\arraystretch}{1.2}\newcolumntype{C}{>{\centering\arraybackslash} m{1cm} }\begin{tabular}{|C|C|C|C|C|C|C|C|} \hline x & -4 & -3 & -2 & -1 & 0 & 1 & 2 \\ \hline y & 17 &  & 5 & 2 &  &  & 5\\ \hline \end{tabular}\\[0.3cm]
	\end{beamercolorbox}
\end{frame}
\begin{frame}[shrink=19,fragile]
	\begin{beamercolorbox}[rounded=true, left, shadow=true,wd=14.8cm]{que1}
		\renewcommand{\arraystretch}{1.2}\newcolumntype{C}{>{\centering\arraybackslash} m{1cm} }\begin{tabular}{|C|C|C|C|C|C|C|C|} \hline x & -4 & -3 & -2 & -1 & 0 & 1 & 2 \\ \hline y & 17 & 10 & 5 & 2 & 1 & 2 & 5\\ \hline \end{tabular}\begin{gnuplot}[terminal=pdf] set nokey; set grid; set border 0; set xtics axis 1; set ytics axis 1; set size ratio 2; set size 1,1; set zeroaxis linewidth 2 linetype 1; set yrange [-1:17]; set xrange [-4:2]; plot 1*x**2+0*x+1 \end{gnuplot}
	\end{beamercolorbox}
\end{frame}

\begin{frame}[shrink=19,fragile]
	\begin{beamercolorbox}[rounded=true, left, shadow=true,wd=14.8cm]{que2}
		 Complete this table for $y = x^{2} + 2$ and draw the graph. \\[0.3cm] \renewcommand{\arraystretch}{1.2}\newcolumntype{C}{>{\centering\arraybackslash} m{1cm} }\begin{tabular}{|C|C|C|C|C|C|C|C|} \hline x & -4 & -3 & -2 & -1 & 0 & 1 & 2 \\ \hline y &  & 11 &  & 3 & 2 &  & 6\\ \hline \end{tabular}\\[0.3cm]
	\end{beamercolorbox}
\end{frame}
\begin{frame}[shrink=19,fragile]
	\begin{beamercolorbox}[rounded=true, left, shadow=true,wd=14.8cm]{que2}
		\renewcommand{\arraystretch}{1.2}\newcolumntype{C}{>{\centering\arraybackslash} m{1cm} }\begin{tabular}{|C|C|C|C|C|C|C|C|} \hline x & -4 & -3 & -2 & -1 & 0 & 1 & 2 \\ \hline y & 18 & 11 & 6 & 3 & 2 & 3 & 6\\ \hline \end{tabular}\begin{gnuplot}[terminal=pdf] set nokey; set grid; set border 0; set xtics axis 1; set ytics axis 1; set size ratio 2; set size 1,1; set zeroaxis linewidth 2 linetype 1; set yrange [-1:18]; set xrange [-4:2]; plot 1*x**2+0*x+2 \end{gnuplot}
	\end{beamercolorbox}
\end{frame}

\begin{frame}[shrink=19,fragile]
	\begin{beamercolorbox}[rounded=true, left, shadow=true,wd=14.8cm]{que1}
		 Complete this table for $y = x^{2} - 4$ and draw the graph. \\[0.3cm] \renewcommand{\arraystretch}{1.2}\newcolumntype{C}{>{\centering\arraybackslash} m{1cm} }\begin{tabular}{|C|C|C|C|C|C|C|C|} \hline x & -2 & -1 & 0 & 1 & 2 & 3 & 4 \\ \hline y &  & -3 & -4 &  & 0 &  & 12\\ \hline \end{tabular}\\[0.3cm]
	\end{beamercolorbox}
\end{frame}
\begin{frame}[shrink=19,fragile]
	\begin{beamercolorbox}[rounded=true, left, shadow=true,wd=14.8cm]{que1}
		\renewcommand{\arraystretch}{1.2}\newcolumntype{C}{>{\centering\arraybackslash} m{1cm} }\begin{tabular}{|C|C|C|C|C|C|C|C|} \hline x & -2 & -1 & 0 & 1 & 2 & 3 & 4 \\ \hline y & 0 & -3 & -4 & -3 & 0 & 5 & 12\\ \hline \end{tabular}\begin{gnuplot}[terminal=pdf] set nokey; set grid; set border 0; set xtics axis 1; set ytics axis 1; set size ratio 2; set size 1,1; set zeroaxis linewidth 2 linetype 1; set yrange [-4:12]; set xrange [-2:4]; plot 1*x**2+0*x+-4 \end{gnuplot}
	\end{beamercolorbox}
\end{frame}

\begin{frame}[shrink=19,fragile]
	\begin{beamercolorbox}[rounded=true, left, shadow=true,wd=14.8cm]{que2}
		 Complete this table for $y = x^{2} - 2$ and draw the graph. \\[0.3cm] \renewcommand{\arraystretch}{1.2}\newcolumntype{C}{>{\centering\arraybackslash} m{1cm} }\begin{tabular}{|C|C|C|C|C|C|C|C|} \hline x & -4 & -3 & -2 & -1 & 0 & 1 & 2 \\ \hline y &  & 7 & 2 & -1 &  & -1 & \\ \hline \end{tabular}\\[0.3cm]
	\end{beamercolorbox}
\end{frame}
\begin{frame}[shrink=19,fragile]
	\begin{beamercolorbox}[rounded=true, left, shadow=true,wd=14.8cm]{que2}
		\renewcommand{\arraystretch}{1.2}\newcolumntype{C}{>{\centering\arraybackslash} m{1cm} }\begin{tabular}{|C|C|C|C|C|C|C|C|} \hline x & -4 & -3 & -2 & -1 & 0 & 1 & 2 \\ \hline y & 14 & 7 & 2 & -1 & -2 & -1 & 2\\ \hline \end{tabular}\begin{gnuplot}[terminal=pdf] set nokey; set grid; set border 0; set xtics axis 1; set ytics axis 1; set size ratio 2; set size 1,1; set zeroaxis linewidth 2 linetype 1; set yrange [-2:14]; set xrange [-4:2]; plot 1*x**2+0*x+-2 \end{gnuplot}
	\end{beamercolorbox}
\end{frame}

\begin{frame}[shrink=19,fragile]
	\begin{beamercolorbox}[rounded=true, left, shadow=true,wd=14.8cm]{que1}
		 Complete this table for $y = x^{2} + 5$ and draw the graph. \\[0.3cm] \renewcommand{\arraystretch}{1.2}\newcolumntype{C}{>{\centering\arraybackslash} m{1cm} }\begin{tabular}{|C|C|C|C|C|C|C|C|} \hline x & -4 & -3 & -2 & -1 & 0 & 1 & 2 \\ \hline y &  & 14 & 9 & 6 &  & 6 & \\ \hline \end{tabular}\\[0.3cm]
	\end{beamercolorbox}
\end{frame}
\begin{frame}[shrink=19,fragile]
	\begin{beamercolorbox}[rounded=true, left, shadow=true,wd=14.8cm]{que1}
		\renewcommand{\arraystretch}{1.2}\newcolumntype{C}{>{\centering\arraybackslash} m{1cm} }\begin{tabular}{|C|C|C|C|C|C|C|C|} \hline x & -4 & -3 & -2 & -1 & 0 & 1 & 2 \\ \hline y & 21 & 14 & 9 & 6 & 5 & 6 & 9\\ \hline \end{tabular}\begin{gnuplot}[terminal=pdf] set nokey; set grid; set border 0; set xtics axis 1; set ytics axis 1; set size ratio 2; set size 1,1; set zeroaxis linewidth 2 linetype 1; set yrange [-1:21]; set xrange [-4:2]; plot 1*x**2+0*x+5 \end{gnuplot}
	\end{beamercolorbox}
\end{frame}

\begin{frame}[shrink=19,fragile]
	\begin{beamercolorbox}[rounded=true, left, shadow=true,wd=14.8cm]{que2}
		 Complete this table for $y = x^{2} - 3$ and draw the graph. \\[0.3cm] \renewcommand{\arraystretch}{1.2}\newcolumntype{C}{>{\centering\arraybackslash} m{1cm} }\begin{tabular}{|C|C|C|C|C|C|C|C|} \hline x & -2 & -1 & 0 & 1 & 2 & 3 & 4 \\ \hline y & 1 &  &  & -2 & 1 & 6 & \\ \hline \end{tabular}\\[0.3cm]
	\end{beamercolorbox}
\end{frame}
\begin{frame}[shrink=19,fragile]
	\begin{beamercolorbox}[rounded=true, left, shadow=true,wd=14.8cm]{que2}
		\renewcommand{\arraystretch}{1.2}\newcolumntype{C}{>{\centering\arraybackslash} m{1cm} }\begin{tabular}{|C|C|C|C|C|C|C|C|} \hline x & -2 & -1 & 0 & 1 & 2 & 3 & 4 \\ \hline y & 1 & -2 & -3 & -2 & 1 & 6 & 13\\ \hline \end{tabular}\begin{gnuplot}[terminal=pdf] set nokey; set grid; set border 0; set xtics axis 1; set ytics axis 1; set size ratio 2; set size 1,1; set zeroaxis linewidth 2 linetype 1; set yrange [-3:13]; set xrange [-2:4]; plot 1*x**2+0*x+-3 \end{gnuplot}
	\end{beamercolorbox}
\end{frame}

\begin{frame}[shrink=19,fragile]
	\begin{beamercolorbox}[rounded=true, left, shadow=true,wd=14.8cm]{que1}
		 Complete this table for $y = x^{2} + 2$ and draw the graph. \\[0.3cm] \renewcommand{\arraystretch}{1.2}\newcolumntype{C}{>{\centering\arraybackslash} m{1cm} }\begin{tabular}{|C|C|C|C|C|C|C|C|} \hline x & -2 & -1 & 0 & 1 & 2 & 3 & 4 \\ \hline y & 6 &  &  & 3 & 6 &  & 18\\ \hline \end{tabular}\\[0.3cm]
	\end{beamercolorbox}
\end{frame}
\begin{frame}[shrink=19,fragile]
	\begin{beamercolorbox}[rounded=true, left, shadow=true,wd=14.8cm]{que1}
		\renewcommand{\arraystretch}{1.2}\newcolumntype{C}{>{\centering\arraybackslash} m{1cm} }\begin{tabular}{|C|C|C|C|C|C|C|C|} \hline x & -2 & -1 & 0 & 1 & 2 & 3 & 4 \\ \hline y & 6 & 3 & 2 & 3 & 6 & 11 & 18\\ \hline \end{tabular}\begin{gnuplot}[terminal=pdf] set nokey; set grid; set border 0; set xtics axis 1; set ytics axis 1; set size ratio 2; set size 1,1; set zeroaxis linewidth 2 linetype 1; set yrange [-1:18]; set xrange [-2:4]; plot 1*x**2+0*x+2 \end{gnuplot}
	\end{beamercolorbox}
\end{frame}

\begin{frame}[shrink=19,fragile]
	\begin{beamercolorbox}[rounded=true, left, shadow=true,wd=14.8cm]{que2}
		 Complete this table for $y = x^{2} + 1$ and draw the graph. \\[0.3cm] \renewcommand{\arraystretch}{1.2}\newcolumntype{C}{>{\centering\arraybackslash} m{1cm} }\begin{tabular}{|C|C|C|C|C|C|C|C|} \hline x & -4 & -3 & -2 & -1 & 0 & 1 & 2 \\ \hline y & 17 &  & 5 & 2 &  & 2 & \\ \hline \end{tabular}\\[0.3cm]
	\end{beamercolorbox}
\end{frame}
\begin{frame}[shrink=19,fragile]
	\begin{beamercolorbox}[rounded=true, left, shadow=true,wd=14.8cm]{que2}
		\renewcommand{\arraystretch}{1.2}\newcolumntype{C}{>{\centering\arraybackslash} m{1cm} }\begin{tabular}{|C|C|C|C|C|C|C|C|} \hline x & -4 & -3 & -2 & -1 & 0 & 1 & 2 \\ \hline y & 17 & 10 & 5 & 2 & 1 & 2 & 5\\ \hline \end{tabular}\begin{gnuplot}[terminal=pdf] set nokey; set grid; set border 0; set xtics axis 1; set ytics axis 1; set size ratio 2; set size 1,1; set zeroaxis linewidth 2 linetype 1; set yrange [-1:17]; set xrange [-4:2]; plot 1*x**2+0*x+1 \end{gnuplot}
	\end{beamercolorbox}
\end{frame}

\begin{frame}[shrink=19,fragile]
	\begin{beamercolorbox}[rounded=true, left, shadow=true,wd=14.8cm]{que1}
		 Complete this table for $y = x^{2} - 2$ and draw the graph. \\[0.3cm] \renewcommand{\arraystretch}{1.2}\newcolumntype{C}{>{\centering\arraybackslash} m{1cm} }\begin{tabular}{|C|C|C|C|C|C|C|C|} \hline x & -2 & -1 & 0 & 1 & 2 & 3 & 4 \\ \hline y & 2 &  & -2 & -1 &  &  & 14\\ \hline \end{tabular}\\[0.3cm]
	\end{beamercolorbox}
\end{frame}
\begin{frame}[shrink=19,fragile]
	\begin{beamercolorbox}[rounded=true, left, shadow=true,wd=14.8cm]{que1}
		\renewcommand{\arraystretch}{1.2}\newcolumntype{C}{>{\centering\arraybackslash} m{1cm} }\begin{tabular}{|C|C|C|C|C|C|C|C|} \hline x & -2 & -1 & 0 & 1 & 2 & 3 & 4 \\ \hline y & 2 & -1 & -2 & -1 & 2 & 7 & 14\\ \hline \end{tabular}\begin{gnuplot}[terminal=pdf] set nokey; set grid; set border 0; set xtics axis 1; set ytics axis 1; set size ratio 2; set size 1,1; set zeroaxis linewidth 2 linetype 1; set yrange [-2:14]; set xrange [-2:4]; plot 1*x**2+0*x+-2 \end{gnuplot}
	\end{beamercolorbox}
\end{frame}

\begin{frame}[shrink=19,fragile]
	\begin{beamercolorbox}[rounded=true, left, shadow=true,wd=14.8cm]{que2}
		 Complete this table for $y = x^{2} - 5$ and draw the graph. \\[0.3cm] \renewcommand{\arraystretch}{1.2}\newcolumntype{C}{>{\centering\arraybackslash} m{1cm} }\begin{tabular}{|C|C|C|C|C|C|C|C|} \hline x & -2 & -1 & 0 & 1 & 2 & 3 & 4 \\ \hline y & -1 &  &  & -4 & -1 & 4 & \\ \hline \end{tabular}\\[0.3cm]
	\end{beamercolorbox}
\end{frame}
\begin{frame}[shrink=19,fragile]
	\begin{beamercolorbox}[rounded=true, left, shadow=true,wd=14.8cm]{que2}
		\renewcommand{\arraystretch}{1.2}\newcolumntype{C}{>{\centering\arraybackslash} m{1cm} }\begin{tabular}{|C|C|C|C|C|C|C|C|} \hline x & -2 & -1 & 0 & 1 & 2 & 3 & 4 \\ \hline y & -1 & -4 & -5 & -4 & -1 & 4 & 11\\ \hline \end{tabular}\begin{gnuplot}[terminal=pdf] set nokey; set grid; set border 0; set xtics axis 1; set ytics axis 1; set size ratio 2; set size 1,1; set zeroaxis linewidth 2 linetype 1; set yrange [-5:11]; set xrange [-2:4]; plot 1*x**2+0*x+-5 \end{gnuplot}
	\end{beamercolorbox}
\end{frame}

\begin{frame}[shrink=19,fragile]
	\begin{beamercolorbox}[rounded=true, left, shadow=true,wd=14.8cm]{que1}
		 Complete this table for $y = x^{2} + 1$ and draw the graph. \\[0.3cm] \renewcommand{\arraystretch}{1.2}\newcolumntype{C}{>{\centering\arraybackslash} m{1cm} }\begin{tabular}{|C|C|C|C|C|C|C|C|} \hline x & -4 & -3 & -2 & -1 & 0 & 1 & 2 \\ \hline y &  & 10 &  & 2 & 1 & 2 & \\ \hline \end{tabular}\\[0.3cm]
	\end{beamercolorbox}
\end{frame}
\begin{frame}[shrink=19,fragile]
	\begin{beamercolorbox}[rounded=true, left, shadow=true,wd=14.8cm]{que1}
		\renewcommand{\arraystretch}{1.2}\newcolumntype{C}{>{\centering\arraybackslash} m{1cm} }\begin{tabular}{|C|C|C|C|C|C|C|C|} \hline x & -4 & -3 & -2 & -1 & 0 & 1 & 2 \\ \hline y & 17 & 10 & 5 & 2 & 1 & 2 & 5\\ \hline \end{tabular}\begin{gnuplot}[terminal=pdf] set nokey; set grid; set border 0; set xtics axis 1; set ytics axis 1; set size ratio 2; set size 1,1; set zeroaxis linewidth 2 linetype 1; set yrange [-1:17]; set xrange [-4:2]; plot 1*x**2+0*x+1 \end{gnuplot}
	\end{beamercolorbox}
\end{frame}

\begin{frame}[shrink=19,fragile]
	\begin{beamercolorbox}[rounded=true, left, shadow=true,wd=14.8cm]{que2}
		 Complete this table for $y = x^{2} + 5$ and draw the graph. \\[0.3cm] \renewcommand{\arraystretch}{1.2}\newcolumntype{C}{>{\centering\arraybackslash} m{1cm} }\begin{tabular}{|C|C|C|C|C|C|C|C|} \hline x & -2 & -1 & 0 & 1 & 2 & 3 & 4 \\ \hline y &  & 6 & 5 & 6 &  &  & 21\\ \hline \end{tabular}\\[0.3cm]
	\end{beamercolorbox}
\end{frame}
\begin{frame}[shrink=19,fragile]
	\begin{beamercolorbox}[rounded=true, left, shadow=true,wd=14.8cm]{que2}
 		\renewcommand{\arraystretch}{1.2}\newcolumntype{C}{>{\centering\arraybackslash} m{1cm} }\begin{tabular}{|C|C|C|C|C|C|C|C|} \hline x & -2 & -1 & 0 & 1 & 2 & 3 & 4 \\ \hline y & 9 & 6 & 5 & 6 & 9 & 14 & 21\\ \hline \end{tabular}\begin{gnuplot}[terminal=pdf] set nokey; set grid; set border 0; set xtics axis 1; set ytics axis 1; set size ratio 2; set size 1,1; set zeroaxis linewidth 2 linetype 1; set yrange [-1:21]; set xrange [-2:4]; plot 1*x**2+0*x+5 \end{gnuplot}
	\end{beamercolorbox}
\end{frame}

\begin{frame}[shrink=19,fragile]
	\begin{beamercolorbox}[rounded=true, left, shadow=true,wd=14.8cm]{que1}
		 Complete this table for $y = x^{2} - 5$ and draw the graph. \\[0.3cm] \renewcommand{\arraystretch}{1.2}\newcolumntype{C}{>{\centering\arraybackslash} m{1cm} }\begin{tabular}{|C|C|C|C|C|C|C|C|} \hline x & -4 & -3 & -2 & -1 & 0 & 1 & 2 \\ \hline y &  & 4 &  & -4 & -5 & -4 & \\ \hline \end{tabular}\\[0.3cm]
	\end{beamercolorbox}
\end{frame}
\begin{frame}[shrink=19,fragile]
	\begin{beamercolorbox}[rounded=true, left, shadow=true,wd=14.8cm]{que1}
		\renewcommand{\arraystretch}{1.2}\newcolumntype{C}{>{\centering\arraybackslash} m{1cm} }\begin{tabular}{|C|C|C|C|C|C|C|C|} \hline x & -4 & -3 & -2 & -1 & 0 & 1 & 2 \\ \hline y & 11 & 4 & -1 & -4 & -5 & -4 & -1\\ \hline \end{tabular}\begin{gnuplot}[terminal=pdf] set nokey; set grid; set border 0; set xtics axis 1; set ytics axis 1; set size ratio 2; set size 1,1; set zeroaxis linewidth 2 linetype 1; set yrange [-5:11]; set xrange [-4:2]; plot 1*x**2+0*x+-5 \end{gnuplot}
	\end{beamercolorbox}
\end{frame}

\begin{frame}[shrink=19,fragile]
	\begin{beamercolorbox}[rounded=true, left, shadow=true,wd=14.8cm]{que2}
		 Complete this table for $y = x^{2} + 4$ and draw the graph. \\[0.3cm] \renewcommand{\arraystretch}{1.2}\newcolumntype{C}{>{\centering\arraybackslash} m{1cm} }\begin{tabular}{|C|C|C|C|C|C|C|C|} \hline x & -4 & -3 & -2 & -1 & 0 & 1 & 2 \\ \hline y & 20 &  & 8 & 5 &  & 5 & \\ \hline \end{tabular}\\[0.3cm]
	\end{beamercolorbox}
\end{frame}
\begin{frame}[shrink=19,fragile]
	\begin{beamercolorbox}[rounded=true, left, shadow=true,wd=14.8cm]{que2}
		\renewcommand{\arraystretch}{1.2}\newcolumntype{C}{>{\centering\arraybackslash} m{1cm} }\begin{tabular}{|C|C|C|C|C|C|C|C|} \hline x & -4 & -3 & -2 & -1 & 0 & 1 & 2 \\ \hline y & 20 & 13 & 8 & 5 & 4 & 5 & 8\\ \hline \end{tabular}\begin{gnuplot}[terminal=pdf] set nokey; set grid; set border 0; set xtics axis 1; set ytics axis 1; set size ratio 2; set size 1,1; set zeroaxis linewidth 2 linetype 1; set yrange [-1:20]; set xrange [-4:2]; plot 1*x**2+0*x+4 \end{gnuplot}
	\end{beamercolorbox}
\end{frame}

\begin{frame}[shrink=19,fragile]
	\begin{beamercolorbox}[rounded=true, left, shadow=true,wd=14.8cm]{que1}
		 Complete this table for $y = x^{2} + 2$ and draw the graph. \\[0.3cm] \renewcommand{\arraystretch}{1.2}\newcolumntype{C}{>{\centering\arraybackslash} m{1cm} }\begin{tabular}{|C|C|C|C|C|C|C|C|} \hline x & -4 & -3 & -2 & -1 & 0 & 1 & 2 \\ \hline y &  & 11 & 6 &  & 2 &  & 6\\ \hline \end{tabular}\\[0.3cm]
	\end{beamercolorbox}
\end{frame}
\begin{frame}[shrink=19,fragile]
	\begin{beamercolorbox}[rounded=true, left, shadow=true,wd=14.8cm]{que1}
		\renewcommand{\arraystretch}{1.2}\newcolumntype{C}{>{\centering\arraybackslash} m{1cm} }\begin{tabular}{|C|C|C|C|C|C|C|C|} \hline x & -4 & -3 & -2 & -1 & 0 & 1 & 2 \\ \hline y & 18 & 11 & 6 & 3 & 2 & 3 & 6\\ \hline \end{tabular}\begin{gnuplot}[terminal=pdf] set nokey; set grid; set border 0; set xtics axis 1; set ytics axis 1; set size ratio 2; set size 1,1; set zeroaxis linewidth 2 linetype 1; set yrange [-1:18]; set xrange [-4:2]; plot 1*x**2+0*x+2 \end{gnuplot}
	\end{beamercolorbox}
\end{frame}

\begin{frame}[shrink=19,fragile]
	\begin{beamercolorbox}[rounded=true, left, shadow=true,wd=14.8cm]{que2}
		 Complete this table for $y = x^{2}$ and draw the graph. \\[0.3cm] \renewcommand{\arraystretch}{1.2}\newcolumntype{C}{>{\centering\arraybackslash} m{1cm} }\begin{tabular}{|C|C|C|C|C|C|C|C|} \hline x & -4 & -3 & -2 & -1 & 0 & 1 & 2 \\ \hline y & 16 &  & 4 & 1 &  & 1 & \\ \hline \end{tabular}\\[0.3cm]
	\end{beamercolorbox}
\end{frame}
\begin{frame}[shrink=19,fragile]
	\begin{beamercolorbox}[rounded=true, left, shadow=true,wd=14.8cm]{que2}
		\renewcommand{\arraystretch}{1.2}\newcolumntype{C}{>{\centering\arraybackslash} m{1cm} }\begin{tabular}{|C|C|C|C|C|C|C|C|} \hline x & -4 & -3 & -2 & -1 & 0 & 1 & 2 \\ \hline y & 16 & 9 & 4 & 1 & 0 & 1 & 4\\ \hline \end{tabular}\begin{gnuplot}[terminal=pdf] set nokey; set grid; set border 0; set xtics axis 1; set ytics axis 1; set size ratio 2; set size 1,1; set zeroaxis linewidth 2 linetype 1; set yrange [-1:16]; set xrange [-4:2]; plot 1*x**2+0*x+0 \end{gnuplot}
	\end{beamercolorbox}
\end{frame}

\begin{frame}[shrink=19,fragile]
	\begin{beamercolorbox}[rounded=true, left, shadow=true,wd=14.8cm]{que1}
		 Complete this table for $y = x^{2}$ and draw the graph. \\[0.3cm] \renewcommand{\arraystretch}{1.2}\newcolumntype{C}{>{\centering\arraybackslash} m{1cm} }\begin{tabular}{|C|C|C|C|C|C|C|C|} \hline x & -4 & -3 & -2 & -1 & 0 & 1 & 2 \\ \hline y & 16 &  &  & 1 & 0 & 1 & \\ \hline \end{tabular}\\[0.3cm]
	\end{beamercolorbox}
\end{frame}
\begin{frame}[shrink=19,fragile]
	\begin{beamercolorbox}[rounded=true, left, shadow=true,wd=14.8cm]{que1}
		\renewcommand{\arraystretch}{1.2}\newcolumntype{C}{>{\centering\arraybackslash} m{1cm} }\begin{tabular}{|C|C|C|C|C|C|C|C|} \hline x & -4 & -3 & -2 & -1 & 0 & 1 & 2 \\ \hline y & 16 & 9 & 4 & 1 & 0 & 1 & 4\\ \hline \end{tabular}\begin{gnuplot}[terminal=pdf] set nokey; set grid; set border 0; set xtics axis 1; set ytics axis 1; set size ratio 2; set size 1,1; set zeroaxis linewidth 2 linetype 1; set yrange [-1:16]; set xrange [-4:2]; plot 1*x**2+0*x+0 \end{gnuplot}
	\end{beamercolorbox}
\end{frame}

\begin{frame}[shrink=19,fragile]
	\begin{beamercolorbox}[rounded=true, left, shadow=true,wd=14.8cm]{que2}
		 Complete this table for $y = x^{2} - 4$ and draw the graph. \\[0.3cm] \renewcommand{\arraystretch}{1.2}\newcolumntype{C}{>{\centering\arraybackslash} m{1cm} }\begin{tabular}{|C|C|C|C|C|C|C|C|} \hline x & -4 & -3 & -2 & -1 & 0 & 1 & 2 \\ \hline y & 12 &  & 0 & -3 &  & -3 & \\ \hline \end{tabular}\\[0.3cm]
	\end{beamercolorbox}
\end{frame}
\begin{frame}[shrink=19,fragile]
	\begin{beamercolorbox}[rounded=true, left, shadow=true,wd=14.8cm]{que2}
		\renewcommand{\arraystretch}{1.2}\newcolumntype{C}{>{\centering\arraybackslash} m{1cm} }\begin{tabular}{|C|C|C|C|C|C|C|C|} \hline x & -4 & -3 & -2 & -1 & 0 & 1 & 2 \\ \hline y & 12 & 5 & 0 & -3 & -4 & -3 & 0\\ \hline \end{tabular}\begin{gnuplot}[terminal=pdf] set nokey; set grid; set border 0; set xtics axis 1; set ytics axis 1; set size ratio 2; set size 1,1; set zeroaxis linewidth 2 linetype 1; set yrange [-4:12]; set xrange [-4:2]; plot 1*x**2+0*x+-4 \end{gnuplot}
	\end{beamercolorbox}
\end{frame}

\begin{frame}[shrink=19,fragile]
	\begin{beamercolorbox}[rounded=true, left, shadow=true,wd=14.8cm]{que1}
		 Complete this table for $y = x^{2}$ and draw the graph. \\[0.3cm] \renewcommand{\arraystretch}{1.2}\newcolumntype{C}{>{\centering\arraybackslash} m{1cm} }\begin{tabular}{|C|C|C|C|C|C|C|C|} \hline x & -4 & -3 & -2 & -1 & 0 & 1 & 2 \\ \hline y & 16 &  &  & 1 & 0 & 1 & \\ \hline \end{tabular}\\[0.3cm]
	\end{beamercolorbox}
\end{frame}
\begin{frame}[shrink=19,fragile]
	\begin{beamercolorbox}[rounded=true, left, shadow=true,wd=14.8cm]{que1}
		\renewcommand{\arraystretch}{1.2}\newcolumntype{C}{>{\centering\arraybackslash} m{1cm} }\begin{tabular}{|C|C|C|C|C|C|C|C|} \hline x & -4 & -3 & -2 & -1 & 0 & 1 & 2 \\ \hline y & 16 & 9 & 4 & 1 & 0 & 1 & 4\\ \hline \end{tabular}\begin{gnuplot}[terminal=pdf] set nokey; set grid; set border 0; set xtics axis 1; set ytics axis 1; set size ratio 2; set size 1,1; set zeroaxis linewidth 2 linetype 1; set yrange [-1:16]; set xrange [-4:2]; plot 1*x**2+0*x+0 \end{gnuplot}
	\end{beamercolorbox}
\end{frame}

\begin{frame}[shrink=19,fragile]
	\begin{beamercolorbox}[rounded=true, left, shadow=true,wd=14.8cm]{que2}
		 Complete this table for $y = x^{2} - 4$ and draw the graph. \\[0.3cm] \renewcommand{\arraystretch}{1.2}\newcolumntype{C}{>{\centering\arraybackslash} m{1cm} }\begin{tabular}{|C|C|C|C|C|C|C|C|} \hline x & -2 & -1 & 0 & 1 & 2 & 3 & 4 \\ \hline y &  & -3 & -4 &  & 0 & 5 & \\ \hline \end{tabular}\\[0.3cm]
	\end{beamercolorbox}
\end{frame}
\begin{frame}[shrink=19,fragile]
	\begin{beamercolorbox}[rounded=true, left, shadow=true,wd=14.8cm]{que2}
		\renewcommand{\arraystretch}{1.2}\newcolumntype{C}{>{\centering\arraybackslash} m{1cm} }\begin{tabular}{|C|C|C|C|C|C|C|C|} \hline x & -2 & -1 & 0 & 1 & 2 & 3 & 4 \\ \hline y & 0 & -3 & -4 & -3 & 0 & 5 & 12\\ \hline \end{tabular}\begin{gnuplot}[terminal=pdf] set nokey; set grid; set border 0; set xtics axis 1; set ytics axis 1; set size ratio 2; set size 1,1; set zeroaxis linewidth 2 linetype 1; set yrange [-4:12]; set xrange [-2:4]; plot 1*x**2+0*x+-4 \end{gnuplot}
	\end{beamercolorbox}
\end{frame}

\begin{frame}[shrink=19,fragile]
	\begin{beamercolorbox}[rounded=true, left, shadow=true,wd=14.8cm]{que1}
		 Complete this table for $y = x^{2} - 2$ and draw the graph. \\[0.3cm] \renewcommand{\arraystretch}{1.2}\newcolumntype{C}{>{\centering\arraybackslash} m{1cm} }\begin{tabular}{|C|C|C|C|C|C|C|C|} \hline x & -4 & -3 & -2 & -1 & 0 & 1 & 2 \\ \hline y & 14 &  & 2 & -1 &  &  & 2\\ \hline \end{tabular}\\[0.3cm]
	\end{beamercolorbox}
\end{frame}
\begin{frame}[shrink=19,fragile]
	\begin{beamercolorbox}[rounded=true, left, shadow=true,wd=14.8cm]{que1}
		\renewcommand{\arraystretch}{1.2}\newcolumntype{C}{>{\centering\arraybackslash} m{1cm} }\begin{tabular}{|C|C|C|C|C|C|C|C|} \hline x & -4 & -3 & -2 & -1 & 0 & 1 & 2 \\ \hline y & 14 & 7 & 2 & -1 & -2 & -1 & 2\\ \hline \end{tabular}\begin{gnuplot}[terminal=pdf] set nokey; set grid; set border 0; set xtics axis 1; set ytics axis 1; set size ratio 2; set size 1,1; set zeroaxis linewidth 2 linetype 1; set yrange [-2:14]; set xrange [-4:2]; plot 1*x**2+0*x+-2 \end{gnuplot}
	\end{beamercolorbox}
\end{frame}

\begin{frame}[shrink=19,fragile]
	\begin{beamercolorbox}[rounded=true, left, shadow=true,wd=14.8cm]{que2}
		 Complete this table for $y = x^{2} + 4$ and draw the graph. \\[0.3cm] \renewcommand{\arraystretch}{1.2}\newcolumntype{C}{>{\centering\arraybackslash} m{1cm} }\begin{tabular}{|C|C|C|C|C|C|C|C|} \hline x & -4 & -3 & -2 & -1 & 0 & 1 & 2 \\ \hline y &  & 13 &  & 5 & 4 & 5 & \\ \hline \end{tabular}\\[0.3cm]
	\end{beamercolorbox}
\end{frame}
\begin{frame}[shrink=19,fragile]
	\begin{beamercolorbox}[rounded=true, left, shadow=true,wd=14.8cm]{que2}
		\renewcommand{\arraystretch}{1.2}\newcolumntype{C}{>{\centering\arraybackslash} m{1cm} }\begin{tabular}{|C|C|C|C|C|C|C|C|} \hline x & -4 & -3 & -2 & -1 & 0 & 1 & 2 \\ \hline y & 20 & 13 & 8 & 5 & 4 & 5 & 8\\ \hline \end{tabular}\begin{gnuplot}[terminal=pdf] set nokey; set grid; set border 0; set xtics axis 1; set ytics axis 1; set size ratio 2; set size 1,1; set zeroaxis linewidth 2 linetype 1; set yrange [-1:20]; set xrange [-4:2]; plot 1*x**2+0*x+4 \end{gnuplot}
	\end{beamercolorbox}
\end{frame}

\begin{frame}[shrink=19,fragile]
	\begin{beamercolorbox}[rounded=true, left, shadow=true,wd=14.8cm]{que1}
		 Complete this table for $y = x^{2} + 1$ and draw the graph. \\[0.3cm] \renewcommand{\arraystretch}{1.2}\newcolumntype{C}{>{\centering\arraybackslash} m{1cm} }\begin{tabular}{|C|C|C|C|C|C|C|C|} \hline x & -4 & -3 & -2 & -1 & 0 & 1 & 2 \\ \hline y &  & 10 & 5 & 2 &  & 2 & \\ \hline \end{tabular}\\[0.3cm]
	\end{beamercolorbox}
\end{frame}
\begin{frame}[shrink=19,fragile]
	\begin{beamercolorbox}[rounded=true, left, shadow=true,wd=14.8cm]{que1}
		\renewcommand{\arraystretch}{1.2}\newcolumntype{C}{>{\centering\arraybackslash} m{1cm} }\begin{tabular}{|C|C|C|C|C|C|C|C|} \hline x & -4 & -3 & -2 & -1 & 0 & 1 & 2 \\ \hline y & 17 & 10 & 5 & 2 & 1 & 2 & 5\\ \hline \end{tabular}\begin{gnuplot}[terminal=pdf] set nokey; set grid; set border 0; set xtics axis 1; set ytics axis 1; set size ratio 2; set size 1,1; set zeroaxis linewidth 2 linetype 1; set yrange [-1:17]; set xrange [-4:2]; plot 1*x**2+0*x+1 \end{gnuplot}
	\end{beamercolorbox}
\end{frame}

\begin{frame}[shrink=19,fragile]
	\begin{beamercolorbox}[rounded=true, left, shadow=true,wd=14.8cm]{que2}
		 Complete this table for $y = x^{2} - 4$ and draw the graph. \\[0.3cm] \renewcommand{\arraystretch}{1.2}\newcolumntype{C}{>{\centering\arraybackslash} m{1cm} }\begin{tabular}{|C|C|C|C|C|C|C|C|} \hline x & -2 & -1 & 0 & 1 & 2 & 3 & 4 \\ \hline y & 0 &  &  & -3 & 0 & 5 & \\ \hline \end{tabular}\\[0.3cm]
	\end{beamercolorbox}
\end{frame}
\begin{frame}[shrink=19,fragile]
	\begin{beamercolorbox}[rounded=true, left, shadow=true,wd=14.8cm]{que2}
		\renewcommand{\arraystretch}{1.2}\newcolumntype{C}{>{\centering\arraybackslash} m{1cm} }\begin{tabular}{|C|C|C|C|C|C|C|C|} \hline x & -2 & -1 & 0 & 1 & 2 & 3 & 4 \\ \hline y & 0 & -3 & -4 & -3 & 0 & 5 & 12\\ \hline \end{tabular}\begin{gnuplot}[terminal=pdf] set nokey; set grid; set border 0; set xtics axis 1; set ytics axis 1; set size ratio 2; set size 1,1; set zeroaxis linewidth 2 linetype 1; set yrange [-4:12]; set xrange [-2:4]; plot 1*x**2+0*x+-4 \end{gnuplot}
	\end{beamercolorbox}
\end{frame}

\end{document}
