\documentclass[13.5pt, varwidth=true]{beamer} 
\usepackage[scaled=1]{helvet}
\usepackage[T1]{fontenc}
\usepackage{overpic}
\usepackage{sansmath}
\usepackage{multicol}
\usepackage{gnuplottex}
\sansmath
\renewcommand\familydefault{\sfdefault}
%\DeclareMathSizes{11}{19}{13}{9}   % For size 11 text
\usefonttheme{professionalfonts}
\usepackage{colortbl}
\usepackage{xcolor}
\usepackage{booktabs}
\usepackage{fancybox}
\usepackage{relsize}
\usepackage{siunitx}
\sisetup{detect-all}
\definecolor{que1}{HTML}{FFC5FA}
\definecolor{que2}{HTML}{C6D7FD}
\definecolor{bgre}{HTML}{DCDCDC}
\definecolor{tgre}{HTML}{BCBCBC}
\definecolor{ared}{HTML}{850000}
\setbeamercolor{background canvas}{bg=bgre}
\setbeamercolor{que1}{fg=black,bg=que1}
\setbeamercolor{que2}{fg=black,bg=que2}
\usetheme{CambridgeUS}
\usecolortheme{beetle}
\setbeamertemplate{navigation symbols}{}
\setbeamertemplate{footline}{\centerline{\textcolor{tgre}{This work is licensed under a Creative Commons Attribution-ShareAlike 4.0 International License}}}
\usepackage{graphicx}
\setbeamercolor*{palette tertiary}{bg=que1}
\setbeamercolor*{palette primary}{bg=que2}
\begin{document}
\date{}

\section[Graphs]{what is this bit?}
\subsection[title]{and this?}

\begin{frame}[shrink=19,fragile]
	\begin{beamercolorbox}[rounded=true, left, shadow=true,wd=14.8cm]{que1}
		 Complete this table for $y = x^{2} + x$ and draw the graph. \\[0.3cm] \renewcommand{\arraystretch}{1.2}\newcolumntype{C}{>{\centering\arraybackslash} m{1cm} }\begin{tabular}{|C|C|C|C|C|C|C|C|} \hline x & -4 & -3 & -2 & -1 & 0 & 1 & 2 \\ \hline y & 12 &  & 2 &  & 0 &  & 6\\ \hline \end{tabular}\\[0.3cm]
	\end{beamercolorbox}
\end{frame}
\begin{frame}[shrink=19,fragile]
	\begin{beamercolorbox}[rounded=true, left, shadow=true,wd=14.8cm]{que1}
		\renewcommand{\arraystretch}{1.2}\newcolumntype{C}{>{\centering\arraybackslash} m{1cm} }\begin{tabular}{|C|C|C|C|C|C|C|C|} \hline x & -4 & -3 & -2 & -1 & 0 & 1 & 2 \\ \hline y & 12 & 6 & 2 & 0 & 0 & 2 & 6\\ \hline \end{tabular}\begin{gnuplot}[terminal=pdf] set nokey; set grid; set border 0; set xtics axis 1; set ytics axis 1; set size ratio 2; set size 1,1; set zeroaxis linewidth 2 linetype 1; set yrange [-1:12]; set xrange [-4:2]; plot 1*x**2+1*x+0 \end{gnuplot}
	\end{beamercolorbox}
\end{frame}

\begin{frame}[shrink=19,fragile]
	\begin{beamercolorbox}[rounded=true, left, shadow=true,wd=14.8cm]{que2}
		 Complete this table for $y = x^{2} + 3x$ and draw the graph. \\[0.3cm] \renewcommand{\arraystretch}{1.2}\newcolumntype{C}{>{\centering\arraybackslash} m{1cm} }\begin{tabular}{|C|C|C|C|C|C|C|C|} \hline x & -5 & -4 & -3 & -2 & -1 & 0 & 1 \\ \hline y & 10 &  &  & -2 & -2 &  & 4\\ \hline \end{tabular}\\[0.3cm]
	\end{beamercolorbox}
\end{frame}
\begin{frame}[shrink=19,fragile]
	\begin{beamercolorbox}[rounded=true, left, shadow=true,wd=14.8cm]{que2}
 		\renewcommand{\arraystretch}{1.2}\newcolumntype{C}{>{\centering\arraybackslash} m{1cm} }\begin{tabular}{|C|C|C|C|C|C|C|C|} \hline x & -5 & -4 & -3 & -2 & -1 & 0 & 1 \\ \hline y & 10 & 4 & 0 & -2 & -2 & 0 & 4\\ \hline \end{tabular}\begin{gnuplot}[terminal=pdf] set nokey; set grid; set border 0; set xtics axis 1; set ytics axis 1; set size ratio 2; set size 1,1; set zeroaxis linewidth 2 linetype 1; set yrange [-3:10]; set xrange [-5:1]; plot 1*x**2+3*x+0 \end{gnuplot}
	\end{beamercolorbox}
\end{frame}

\begin{frame}[shrink=19,fragile]
	\begin{beamercolorbox}[rounded=true, left, shadow=true,wd=14.8cm]{que1}
		 Complete this table for $y = x^{2} + x$ and draw the graph. \\[0.3cm] \renewcommand{\arraystretch}{1.2}\newcolumntype{C}{>{\centering\arraybackslash} m{1cm} }\begin{tabular}{|C|C|C|C|C|C|C|C|} \hline x & -3 & -2 & -1 & 0 & 1 & 2 & 3 \\ \hline y & 6 &  & 0 & 0 &  &  & 12\\ \hline \end{tabular}\\[0.3cm]
	\end{beamercolorbox}
\end{frame}
\begin{frame}[shrink=19,fragile]
	\begin{beamercolorbox}[rounded=true, left, shadow=true,wd=14.8cm]{que1}
		\renewcommand{\arraystretch}{1.2}\newcolumntype{C}{>{\centering\arraybackslash} m{1cm} }\begin{tabular}{|C|C|C|C|C|C|C|C|} \hline x & -3 & -2 & -1 & 0 & 1 & 2 & 3 \\ \hline y & 6 & 2 & 0 & 0 & 2 & 6 & 12\\ \hline \end{tabular}\begin{gnuplot}[terminal=pdf] set nokey; set grid; set border 0; set xtics axis 1; set ytics axis 1; set size ratio 2; set size 1,1; set zeroaxis linewidth 2 linetype 1; set yrange [-1:12]; set xrange [-3:3]; plot 1*x**2+1*x+0 \end{gnuplot}
	\end{beamercolorbox}
\end{frame}

\begin{frame}[shrink=19,fragile]
	\begin{beamercolorbox}[rounded=true, left, shadow=true,wd=14.8cm]{que2}
		 Complete this table for $y = x^{2} + 2x$ and draw the graph. \\[0.3cm] \renewcommand{\arraystretch}{1.2}\newcolumntype{C}{>{\centering\arraybackslash} m{1cm} }\begin{tabular}{|C|C|C|C|C|C|C|C|} \hline x & -5 & -4 & -3 & -2 & -1 & 0 & 1 \\ \hline y & 15 &  & 3 &  & -1 & 0 & \\ \hline \end{tabular}\\[0.3cm]
	\end{beamercolorbox}
\end{frame}
\begin{frame}[shrink=19,fragile]
	\begin{beamercolorbox}[rounded=true, left, shadow=true,wd=14.8cm]{que2}
		\renewcommand{\arraystretch}{1.2}\newcolumntype{C}{>{\centering\arraybackslash} m{1cm} }\begin{tabular}{|C|C|C|C|C|C|C|C|} \hline x & -5 & -4 & -3 & -2 & -1 & 0 & 1 \\ \hline y & 15 & 8 & 3 & 0 & -1 & 0 & 3\\ \hline \end{tabular}\begin{gnuplot}[terminal=pdf] set nokey; set grid; set border 0; set xtics axis 1; set ytics axis 1; set size ratio 2; set size 1,1; set zeroaxis linewidth 2 linetype 1; set yrange [-1:15]; set xrange [-5:1]; plot 1*x**2+2*x+0 \end{gnuplot}
	\end{beamercolorbox}
\end{frame}

\begin{frame}[shrink=19,fragile]
	\begin{beamercolorbox}[rounded=true, left, shadow=true,wd=14.8cm]{que1}
		 Complete this table for $y = x^{2} + 3x$ and draw the graph. \\[0.3cm] \renewcommand{\arraystretch}{1.2}\newcolumntype{C}{>{\centering\arraybackslash} m{1cm} }\begin{tabular}{|C|C|C|C|C|C|C|C|} \hline x & -4 & -3 & -2 & -1 & 0 & 1 & 2 \\ \hline y & 4 &  & -2 &  & 0 &  & 10\\ \hline \end{tabular}\\[0.3cm]
	\end{beamercolorbox}
\end{frame}
\begin{frame}[shrink=19,fragile]
	\begin{beamercolorbox}[rounded=true, left, shadow=true,wd=14.8cm]{que1}
		\renewcommand{\arraystretch}{1.2}\newcolumntype{C}{>{\centering\arraybackslash} m{1cm} }\begin{tabular}{|C|C|C|C|C|C|C|C|} \hline x & -4 & -3 & -2 & -1 & 0 & 1 & 2 \\ \hline y & 4 & 0 & -2 & -2 & 0 & 4 & 10\\ \hline \end{tabular}\begin{gnuplot}[terminal=pdf] set nokey; set grid; set border 0; set xtics axis 1; set ytics axis 1; set size ratio 2; set size 1,1; set zeroaxis linewidth 2 linetype 1; set yrange [-3:10]; set xrange [-4:2]; plot 1*x**2+3*x+0 \end{gnuplot}
	\end{beamercolorbox}
\end{frame}

\begin{frame}[shrink=19,fragile]
	\begin{beamercolorbox}[rounded=true, left, shadow=true,wd=14.8cm]{que2}
		 Complete this table for $y = x^{2} - 2x$ and draw the graph. \\[0.3cm] \renewcommand{\arraystretch}{1.2}\newcolumntype{C}{>{\centering\arraybackslash} m{1cm} }\begin{tabular}{|C|C|C|C|C|C|C|C|} \hline x & -3 & -2 & -1 & 0 & 1 & 2 & 3 \\ \hline y & 15 &  & 3 &  & -1 & 0 & \\ \hline \end{tabular}\\[0.3cm]
	\end{beamercolorbox}
\end{frame}
\begin{frame}[shrink=19,fragile]
	\begin{beamercolorbox}[rounded=true, left, shadow=true,wd=14.8cm]{que2}
		\renewcommand{\arraystretch}{1.2}\newcolumntype{C}{>{\centering\arraybackslash} m{1cm} }\begin{tabular}{|C|C|C|C|C|C|C|C|} \hline x & -3 & -2 & -1 & 0 & 1 & 2 & 3 \\ \hline y & 15 & 8 & 3 & 0 & -1 & 0 & 3\\ \hline \end{tabular}\begin{gnuplot}[terminal=pdf] set nokey; set grid; set border 0; set xtics axis 1; set ytics axis 1; set size ratio 2; set size 1,1; set zeroaxis linewidth 2 linetype 1; set yrange [-1:15]; set xrange [-3:3]; plot 1*x**2+-2*x+0 \end{gnuplot}
	\end{beamercolorbox}
\end{frame}

\begin{frame}[shrink=19,fragile]
	\begin{beamercolorbox}[rounded=true, left, shadow=true,wd=14.8cm]{que1}
		 Complete this table for $y = x^{2} - 3x$ and draw the graph. \\[0.3cm] \renewcommand{\arraystretch}{1.2}\newcolumntype{C}{>{\centering\arraybackslash} m{1cm} }\begin{tabular}{|C|C|C|C|C|C|C|C|} \hline x & -2 & -1 & 0 & 1 & 2 & 3 & 4 \\ \hline y & 10 &  & 0 &  & -2 &  & 4\\ \hline \end{tabular}\\[0.3cm]
	\end{beamercolorbox}
\end{frame}
\begin{frame}[shrink=19,fragile]
	\begin{beamercolorbox}[rounded=true, left, shadow=true,wd=14.8cm]{que1}
		\renewcommand{\arraystretch}{1.2}\newcolumntype{C}{>{\centering\arraybackslash} m{1cm} }\begin{tabular}{|C|C|C|C|C|C|C|C|} \hline x & -2 & -1 & 0 & 1 & 2 & 3 & 4 \\ \hline y & 10 & 4 & 0 & -2 & -2 & 0 & 4\\ \hline \end{tabular}\begin{gnuplot}[terminal=pdf] set nokey; set grid; set border 0; set xtics axis 1; set ytics axis 1; set size ratio 2; set size 1,1; set zeroaxis linewidth 2 linetype 1; set yrange [-3:10]; set xrange [-2:4]; plot 1*x**2+-3*x+0 \end{gnuplot}
	\end{beamercolorbox}
\end{frame}

\begin{frame}[shrink=19,fragile]
	\begin{beamercolorbox}[rounded=true, left, shadow=true,wd=14.8cm]{que2}
		 Complete this table for $y = x^{2} - 3x$ and draw the graph. \\[0.3cm] \renewcommand{\arraystretch}{1.2}\newcolumntype{C}{>{\centering\arraybackslash} m{1cm} }\begin{tabular}{|C|C|C|C|C|C|C|C|} \hline x & -1 & 0 & 1 & 2 & 3 & 4 & 5 \\ \hline y & 4 &  & -2 & -2 &  &  & 10\\ \hline \end{tabular}\\[0.3cm]
	\end{beamercolorbox}
\end{frame}
\begin{frame}[shrink=19,fragile]
	\begin{beamercolorbox}[rounded=true, left, shadow=true,wd=14.8cm]{que2}
		\renewcommand{\arraystretch}{1.2}\newcolumntype{C}{>{\centering\arraybackslash} m{1cm} }\begin{tabular}{|C|C|C|C|C|C|C|C|} \hline x & -1 & 0 & 1 & 2 & 3 & 4 & 5 \\ \hline y & 4 & 0 & -2 & -2 & 0 & 4 & 10\\ \hline \end{tabular}\begin{gnuplot}[terminal=pdf] set nokey; set grid; set border 0; set xtics axis 1; set ytics axis 1; set size ratio 2; set size 1,1; set zeroaxis linewidth 2 linetype 1; set yrange [-3:10]; set xrange [-1:5]; plot 1*x**2+-3*x+0 \end{gnuplot}
	\end{beamercolorbox}
\end{frame}

\begin{frame}[shrink=19,fragile]
	\begin{beamercolorbox}[rounded=true, left, shadow=true,wd=14.8cm]{que1}
		 Complete this table for $y = x^{2} + 3x$ and draw the graph. \\[0.3cm] \renewcommand{\arraystretch}{1.2}\newcolumntype{C}{>{\centering\arraybackslash} m{1cm} }\begin{tabular}{|C|C|C|C|C|C|C|C|} \hline x & -5 & -4 & -3 & -2 & -1 & 0 & 1 \\ \hline y &  & 4 & 0 & -2 &  & 0 & \\ \hline \end{tabular}\\[0.3cm]
	\end{beamercolorbox}
\end{frame}
\begin{frame}[shrink=19,fragile]
	\begin{beamercolorbox}[rounded=true, left, shadow=true,wd=14.8cm]{que1}
		\renewcommand{\arraystretch}{1.2}\newcolumntype{C}{>{\centering\arraybackslash} m{1cm} }\begin{tabular}{|C|C|C|C|C|C|C|C|} \hline x & -5 & -4 & -3 & -2 & -1 & 0 & 1 \\ \hline y & 10 & 4 & 0 & -2 & -2 & 0 & 4\\ \hline \end{tabular}\begin{gnuplot}[terminal=pdf] set nokey; set grid; set border 0; set xtics axis 1; set ytics axis 1; set size ratio 2; set size 1,1; set zeroaxis linewidth 2 linetype 1; set yrange [-3:10]; set xrange [-5:1]; plot 1*x**2+3*x+0 \end{gnuplot}
	\end{beamercolorbox}
\end{frame}

\begin{frame}[shrink=19,fragile]
	\begin{beamercolorbox}[rounded=true, left, shadow=true,wd=14.8cm]{que2}
		 Complete this table for $y = x^{2} + 2x$ and draw the graph. \\[0.3cm] \renewcommand{\arraystretch}{1.2}\newcolumntype{C}{>{\centering\arraybackslash} m{1cm} }\begin{tabular}{|C|C|C|C|C|C|C|C|} \hline x & -3 & -2 & -1 & 0 & 1 & 2 & 3 \\ \hline y &  & 0 & -1 &  & 3 & 8 & \\ \hline \end{tabular}\\[0.3cm]
	\end{beamercolorbox}
\end{frame}
\begin{frame}[shrink=19,fragile]
	\begin{beamercolorbox}[rounded=true, left, shadow=true,wd=14.8cm]{que2}
		\renewcommand{\arraystretch}{1.2}\newcolumntype{C}{>{\centering\arraybackslash} m{1cm} }\begin{tabular}{|C|C|C|C|C|C|C|C|} \hline x & -3 & -2 & -1 & 0 & 1 & 2 & 3 \\ \hline y & 3 & 0 & -1 & 0 & 3 & 8 & 15\\ \hline \end{tabular}\begin{gnuplot}[terminal=pdf] set nokey; set grid; set border 0; set xtics axis 1; set ytics axis 1; set size ratio 2; set size 1,1; set zeroaxis linewidth 2 linetype 1; set yrange [-1:15]; set xrange [-3:3]; plot 1*x**2+2*x+0 \end{gnuplot}
	\end{beamercolorbox}
\end{frame}

\begin{frame}[shrink=19,fragile]
	\begin{beamercolorbox}[rounded=true, left, shadow=true,wd=14.8cm]{que1}
		 Complete this table for $y = x^{2} + 2x$ and draw the graph. \\[0.3cm] \renewcommand{\arraystretch}{1.2}\newcolumntype{C}{>{\centering\arraybackslash} m{1cm} }\begin{tabular}{|C|C|C|C|C|C|C|C|} \hline x & -3 & -2 & -1 & 0 & 1 & 2 & 3 \\ \hline y &  & 0 & -1 & 0 &  & 8 & \\ \hline \end{tabular}\\[0.3cm]
	\end{beamercolorbox}
\end{frame}
\begin{frame}[shrink=19,fragile]
	\begin{beamercolorbox}[rounded=true, left, shadow=true,wd=14.8cm]{que1}
		\renewcommand{\arraystretch}{1.2}\newcolumntype{C}{>{\centering\arraybackslash} m{1cm} }\begin{tabular}{|C|C|C|C|C|C|C|C|} \hline x & -3 & -2 & -1 & 0 & 1 & 2 & 3 \\ \hline y & 3 & 0 & -1 & 0 & 3 & 8 & 15\\ \hline \end{tabular}\begin{gnuplot}[terminal=pdf] set nokey; set grid; set border 0; set xtics axis 1; set ytics axis 1; set size ratio 2; set size 1,1; set zeroaxis linewidth 2 linetype 1; set yrange [-1:15]; set xrange [-3:3]; plot 1*x**2+2*x+0 \end{gnuplot}
	\end{beamercolorbox}
\end{frame}

\begin{frame}[shrink=19,fragile]
	\begin{beamercolorbox}[rounded=true, left, shadow=true,wd=14.8cm]{que2}
		 Complete this table for $y = x^{2} - 3x$ and draw the graph. \\[0.3cm] \renewcommand{\arraystretch}{1.2}\newcolumntype{C}{>{\centering\arraybackslash} m{1cm} }\begin{tabular}{|C|C|C|C|C|C|C|C|} \hline x & -2 & -1 & 0 & 1 & 2 & 3 & 4 \\ \hline y &  & 4 &  & -2 & -2 & 0 & \\ \hline \end{tabular}\\[0.3cm]
	\end{beamercolorbox}
\end{frame}
\begin{frame}[shrink=19,fragile]
	\begin{beamercolorbox}[rounded=true, left, shadow=true,wd=14.8cm]{que2}
		\renewcommand{\arraystretch}{1.2}\newcolumntype{C}{>{\centering\arraybackslash} m{1cm} }\begin{tabular}{|C|C|C|C|C|C|C|C|} \hline x & -2 & -1 & 0 & 1 & 2 & 3 & 4 \\ \hline y & 10 & 4 & 0 & -2 & -2 & 0 & 4\\ \hline \end{tabular}\begin{gnuplot}[terminal=pdf] set nokey; set grid; set border 0; set xtics axis 1; set ytics axis 1; set size ratio 2; set size 1,1; set zeroaxis linewidth 2 linetype 1; set yrange [-3:10]; set xrange [-2:4]; plot 1*x**2+-3*x+0 \end{gnuplot}
	\end{beamercolorbox}
\end{frame}

\begin{frame}[shrink=19,fragile]
	\begin{beamercolorbox}[rounded=true, left, shadow=true,wd=14.8cm]{que1}
		 Complete this table for $y = x^{2} - 3x$ and draw the graph. \\[0.3cm] \renewcommand{\arraystretch}{1.2}\newcolumntype{C}{>{\centering\arraybackslash} m{1cm} }\begin{tabular}{|C|C|C|C|C|C|C|C|} \hline x & -2 & -1 & 0 & 1 & 2 & 3 & 4 \\ \hline y &  & 4 &  & -2 & -2 & 0 & \\ \hline \end{tabular}\\[0.3cm]
	\end{beamercolorbox}
\end{frame}
\begin{frame}[shrink=19,fragile]
	\begin{beamercolorbox}[rounded=true, left, shadow=true,wd=14.8cm]{que1}
		\renewcommand{\arraystretch}{1.2}\newcolumntype{C}{>{\centering\arraybackslash} m{1cm} }\begin{tabular}{|C|C|C|C|C|C|C|C|} \hline x & -2 & -1 & 0 & 1 & 2 & 3 & 4 \\ \hline y & 10 & 4 & 0 & -2 & -2 & 0 & 4\\ \hline \end{tabular}\begin{gnuplot}[terminal=pdf] set nokey; set grid; set border 0; set xtics axis 1; set ytics axis 1; set size ratio 2; set size 1,1; set zeroaxis linewidth 2 linetype 1; set yrange [-3:10]; set xrange [-2:4]; plot 1*x**2+-3*x+0 \end{gnuplot}
	\end{beamercolorbox}
\end{frame}

\begin{frame}[shrink=19,fragile]
	\begin{beamercolorbox}[rounded=true, left, shadow=true,wd=14.8cm]{que2}
		 Complete this table for $y = x^{2} + 3x$ and draw the graph. \\[0.3cm] \renewcommand{\arraystretch}{1.2}\newcolumntype{C}{>{\centering\arraybackslash} m{1cm} }\begin{tabular}{|C|C|C|C|C|C|C|C|} \hline x & -5 & -4 & -3 & -2 & -1 & 0 & 1 \\ \hline y & 10 &  & 0 & -2 &  & 0 & \\ \hline \end{tabular}\\[0.3cm]
	\end{beamercolorbox}
\end{frame}
\begin{frame}[shrink=19,fragile]
	\begin{beamercolorbox}[rounded=true, left, shadow=true,wd=14.8cm]{que2}
		\renewcommand{\arraystretch}{1.2}\newcolumntype{C}{>{\centering\arraybackslash} m{1cm} }\begin{tabular}{|C|C|C|C|C|C|C|C|} \hline x & -5 & -4 & -3 & -2 & -1 & 0 & 1 \\ \hline y & 10 & 4 & 0 & -2 & -2 & 0 & 4\\ \hline \end{tabular}\begin{gnuplot}[terminal=pdf] set nokey; set grid; set border 0; set xtics axis 1; set ytics axis 1; set size ratio 2; set size 1,1; set zeroaxis linewidth 2 linetype 1; set yrange [-3:10]; set xrange [-5:1]; plot 1*x**2+3*x+0 \end{gnuplot}
	\end{beamercolorbox}
\end{frame}

\begin{frame}[shrink=19,fragile]
	\begin{beamercolorbox}[rounded=true, left, shadow=true,wd=14.8cm]{que1}
		 Complete this table for $y = x^{2} + 4x$ and draw the graph. \\[0.3cm] \renewcommand{\arraystretch}{1.2}\newcolumntype{C}{>{\centering\arraybackslash} m{1cm} }\begin{tabular}{|C|C|C|C|C|C|C|C|} \hline x & -6 & -5 & -4 & -3 & -2 & -1 & 0 \\ \hline y &  & 5 &  & -3 & -4 & -3 & \\ \hline \end{tabular}\\[0.3cm]
	\end{beamercolorbox}
\end{frame}
\begin{frame}[shrink=19,fragile]
	\begin{beamercolorbox}[rounded=true, left, shadow=true,wd=14.8cm]{que1}
		\renewcommand{\arraystretch}{1.2}\newcolumntype{C}{>{\centering\arraybackslash} m{1cm} }\begin{tabular}{|C|C|C|C|C|C|C|C|} \hline x & -6 & -5 & -4 & -3 & -2 & -1 & 0 \\ \hline y & 12 & 5 & 0 & -3 & -4 & -3 & 0\\ \hline \end{tabular}\begin{gnuplot}[terminal=pdf] set nokey; set grid; set border 0; set xtics axis 1; set ytics axis 1; set size ratio 2; set size 1,1; set zeroaxis linewidth 2 linetype 1; set yrange [-4:12]; set xrange [-6:0]; plot 1*x**2+4*x+0 \end{gnuplot}
	\end{beamercolorbox}
\end{frame}

\begin{frame}[shrink=19,fragile]
	\begin{beamercolorbox}[rounded=true, left, shadow=true,wd=14.8cm]{que2}
		 Complete this table for $y = x^{2} - x$ and draw the graph. \\[0.3cm] \renewcommand{\arraystretch}{1.2}\newcolumntype{C}{>{\centering\arraybackslash} m{1cm} }\begin{tabular}{|C|C|C|C|C|C|C|C|} \hline x & -3 & -2 & -1 & 0 & 1 & 2 & 3 \\ \hline y &  & 6 & 2 &  & 0 & 2 & \\ \hline \end{tabular}\\[0.3cm]
	\end{beamercolorbox}
\end{frame}
\begin{frame}[shrink=19,fragile]
	\begin{beamercolorbox}[rounded=true, left, shadow=true,wd=14.8cm]{que2}
		\renewcommand{\arraystretch}{1.2}\newcolumntype{C}{>{\centering\arraybackslash} m{1cm} }\begin{tabular}{|C|C|C|C|C|C|C|C|} \hline x & -3 & -2 & -1 & 0 & 1 & 2 & 3 \\ \hline y & 12 & 6 & 2 & 0 & 0 & 2 & 6\\ \hline \end{tabular}\begin{gnuplot}[terminal=pdf] set nokey; set grid; set border 0; set xtics axis 1; set ytics axis 1; set size ratio 2; set size 1,1; set zeroaxis linewidth 2 linetype 1; set yrange [-1:12]; set xrange [-3:3]; plot 1*x**2+-1*x+0 \end{gnuplot}
	\end{beamercolorbox}
\end{frame}

\begin{frame}[shrink=19,fragile]
	\begin{beamercolorbox}[rounded=true, left, shadow=true,wd=14.8cm]{que1}
		 Complete this table for $y = x^{2} + 4x$ and draw the graph. \\[0.3cm] \renewcommand{\arraystretch}{1.2}\newcolumntype{C}{>{\centering\arraybackslash} m{1cm} }\begin{tabular}{|C|C|C|C|C|C|C|C|} \hline x & -6 & -5 & -4 & -3 & -2 & -1 & 0 \\ \hline y & 12 &  &  & -3 & -4 &  & 0\\ \hline \end{tabular}\\[0.3cm]
	\end{beamercolorbox}
\end{frame}
\begin{frame}[shrink=19,fragile]
	\begin{beamercolorbox}[rounded=true, left, shadow=true,wd=14.8cm]{que1}
		\renewcommand{\arraystretch}{1.2}\newcolumntype{C}{>{\centering\arraybackslash} m{1cm} }\begin{tabular}{|C|C|C|C|C|C|C|C|} \hline x & -6 & -5 & -4 & -3 & -2 & -1 & 0 \\ \hline y & 12 & 5 & 0 & -3 & -4 & -3 & 0\\ \hline \end{tabular}\begin{gnuplot}[terminal=pdf] set nokey; set grid; set border 0; set xtics axis 1; set ytics axis 1; set size ratio 2; set size 1,1; set zeroaxis linewidth 2 linetype 1; set yrange [-4:12]; set xrange [-6:0]; plot 1*x**2+4*x+0 \end{gnuplot}
	\end{beamercolorbox}
\end{frame}

\begin{frame}[shrink=19,fragile]
	\begin{beamercolorbox}[rounded=true, left, shadow=true,wd=14.8cm]{que2}
		 Complete this table for $y = x^{2} + 5x$ and draw the graph. \\[0.3cm] \renewcommand{\arraystretch}{1.2}\newcolumntype{C}{>{\centering\arraybackslash} m{1cm} }\begin{tabular}{|C|C|C|C|C|C|C|C|} \hline x & -5 & -4 & -3 & -2 & -1 & 0 & 1 \\ \hline y & 0 &  &  & -6 & -4 &  & 6\\ \hline \end{tabular}\\[0.3cm]
	\end{beamercolorbox}
\end{frame}
\begin{frame}[shrink=19,fragile]
	\begin{beamercolorbox}[rounded=true, left, shadow=true,wd=14.8cm]{que2}
		\renewcommand{\arraystretch}{1.2}\newcolumntype{C}{>{\centering\arraybackslash} m{1cm} }\begin{tabular}{|C|C|C|C|C|C|C|C|} \hline x & -5 & -4 & -3 & -2 & -1 & 0 & 1 \\ \hline y & 0 & -4 & -6 & -6 & -4 & 0 & 6\\ \hline \end{tabular}\begin{gnuplot}[terminal=pdf] set nokey; set grid; set border 0; set xtics axis 1; set ytics axis 1; set size ratio 2; set size 1,1; set zeroaxis linewidth 2 linetype 1; set yrange [-7:6]; set xrange [-5:1]; plot 1*x**2+5*x+0 \end{gnuplot}
	\end{beamercolorbox}
\end{frame}

\begin{frame}[shrink=19,fragile]
	\begin{beamercolorbox}[rounded=true, left, shadow=true,wd=14.8cm]{que1}
		 Complete this table for $y = x^{2} - 5x$ and draw the graph. \\[0.3cm] \renewcommand{\arraystretch}{1.2}\newcolumntype{C}{>{\centering\arraybackslash} m{1cm} }\begin{tabular}{|C|C|C|C|C|C|C|C|} \hline x & 0 & 1 & 2 & 3 & 4 & 5 & 6 \\ \hline y &  & -4 & -6 & -6 &  & 0 & \\ \hline \end{tabular}\\[0.3cm]
	\end{beamercolorbox}
\end{frame}
\begin{frame}[shrink=19,fragile]
	\begin{beamercolorbox}[rounded=true, left, shadow=true,wd=14.8cm]{que1}
		\renewcommand{\arraystretch}{1.2}\newcolumntype{C}{>{\centering\arraybackslash} m{1cm} }\begin{tabular}{|C|C|C|C|C|C|C|C|} \hline x & 0 & 1 & 2 & 3 & 4 & 5 & 6 \\ \hline y & 0 & -4 & -6 & -6 & -4 & 0 & 6\\ \hline \end{tabular}\begin{gnuplot}[terminal=pdf] set nokey; set grid; set border 0; set xtics axis 1; set ytics axis 1; set size ratio 2; set size 1,1; set zeroaxis linewidth 2 linetype 1; set yrange [-7:6]; set xrange [0:6]; plot 1*x**2+-5*x+0 \end{gnuplot}
	\end{beamercolorbox}
\end{frame}

\begin{frame}[shrink=19,fragile]
	\begin{beamercolorbox}[rounded=true, left, shadow=true,wd=14.8cm]{que2}
		 Complete this table for $y = x^{2} - 4x$ and draw the graph. \\[0.3cm] \renewcommand{\arraystretch}{1.2}\newcolumntype{C}{>{\centering\arraybackslash} m{1cm} }\begin{tabular}{|C|C|C|C|C|C|C|C|} \hline x & -2 & -1 & 0 & 1 & 2 & 3 & 4 \\ \hline y & 12 &  & 0 & -3 &  &  & 0\\ \hline \end{tabular}\\[0.3cm]
	\end{beamercolorbox}
\end{frame}
\begin{frame}[shrink=19,fragile]
	\begin{beamercolorbox}[rounded=true, left, shadow=true,wd=14.8cm]{que2}
		\renewcommand{\arraystretch}{1.2}\newcolumntype{C}{>{\centering\arraybackslash} m{1cm} }\begin{tabular}{|C|C|C|C|C|C|C|C|} \hline x & -2 & -1 & 0 & 1 & 2 & 3 & 4 \\ \hline y & 12 & 5 & 0 & -3 & -4 & -3 & 0\\ \hline \end{tabular}\begin{gnuplot}[terminal=pdf] set nokey; set grid; set border 0; set xtics axis 1; set ytics axis 1; set size ratio 2; set size 1,1; set zeroaxis linewidth 2 linetype 1; set yrange [-4:12]; set xrange [-2:4]; plot 1*x**2+-4*x+0 \end{gnuplot}
	\end{beamercolorbox}
\end{frame}

\begin{frame}[shrink=19,fragile]
	\begin{beamercolorbox}[rounded=true, left, shadow=true,wd=14.8cm]{que1}
		 Complete this table for $y = x^{2} + 4x$ and draw the graph. \\[0.3cm] \renewcommand{\arraystretch}{1.2}\newcolumntype{C}{>{\centering\arraybackslash} m{1cm} }\begin{tabular}{|C|C|C|C|C|C|C|C|} \hline x & -4 & -3 & -2 & -1 & 0 & 1 & 2 \\ \hline y & 0 &  & -4 &  & 0 &  & 12\\ \hline \end{tabular}\\[0.3cm]
	\end{beamercolorbox}
\end{frame}
\begin{frame}[shrink=19,fragile]
	\begin{beamercolorbox}[rounded=true, left, shadow=true,wd=14.8cm]{que1}
		\renewcommand{\arraystretch}{1.2}\newcolumntype{C}{>{\centering\arraybackslash} m{1cm} }\begin{tabular}{|C|C|C|C|C|C|C|C|} \hline x & -4 & -3 & -2 & -1 & 0 & 1 & 2 \\ \hline y & 0 & -3 & -4 & -3 & 0 & 5 & 12\\ \hline \end{tabular}\begin{gnuplot}[terminal=pdf] set nokey; set grid; set border 0; set xtics axis 1; set ytics axis 1; set size ratio 2; set size 1,1; set zeroaxis linewidth 2 linetype 1; set yrange [-4:12]; set xrange [-4:2]; plot 1*x**2+4*x+0 \end{gnuplot}
	\end{beamercolorbox}
\end{frame}

\begin{frame}[shrink=19,fragile]
	\begin{beamercolorbox}[rounded=true, left, shadow=true,wd=14.8cm]{que2}
		 Complete this table for $y = x^{2} + 5x$ and draw the graph. \\[0.3cm] \renewcommand{\arraystretch}{1.2}\newcolumntype{C}{>{\centering\arraybackslash} m{1cm} }\begin{tabular}{|C|C|C|C|C|C|C|C|} \hline x & -6 & -5 & -4 & -3 & -2 & -1 & 0 \\ \hline y &  & 0 & -4 &  & -6 & -4 & \\ \hline \end{tabular}\\[0.3cm]
	\end{beamercolorbox}
\end{frame}
\begin{frame}[shrink=19,fragile]
	\begin{beamercolorbox}[rounded=true, left, shadow=true,wd=14.8cm]{que2}
		\renewcommand{\arraystretch}{1.2}\newcolumntype{C}{>{\centering\arraybackslash} m{1cm} }\begin{tabular}{|C|C|C|C|C|C|C|C|} \hline x & -6 & -5 & -4 & -3 & -2 & -1 & 0 \\ \hline y & 6 & 0 & -4 & -6 & -6 & -4 & 0\\ \hline \end{tabular}\begin{gnuplot}[terminal=pdf] set nokey; set grid; set border 0; set xtics axis 1; set ytics axis 1; set size ratio 2; set size 1,1; set zeroaxis linewidth 2 linetype 1; set yrange [-7:6]; set xrange [-6:0]; plot 1*x**2+5*x+0 \end{gnuplot}
	\end{beamercolorbox}
\end{frame}

\begin{frame}[shrink=19,fragile]
	\begin{beamercolorbox}[rounded=true, left, shadow=true,wd=14.8cm]{que1}
		 Complete this table for $y = x^{2} + 2x$ and draw the graph. \\[0.3cm] \renewcommand{\arraystretch}{1.2}\newcolumntype{C}{>{\centering\arraybackslash} m{1cm} }\begin{tabular}{|C|C|C|C|C|C|C|C|} \hline x & -5 & -4 & -3 & -2 & -1 & 0 & 1 \\ \hline y &  & 8 & 3 & 0 &  &  & 3\\ \hline \end{tabular}\\[0.3cm]
	\end{beamercolorbox}
\end{frame}
\begin{frame}[shrink=19,fragile]
	\begin{beamercolorbox}[rounded=true, left, shadow=true,wd=14.8cm]{que1}
		\renewcommand{\arraystretch}{1.2}\newcolumntype{C}{>{\centering\arraybackslash} m{1cm} }\begin{tabular}{|C|C|C|C|C|C|C|C|} \hline x & -5 & -4 & -3 & -2 & -1 & 0 & 1 \\ \hline y & 15 & 8 & 3 & 0 & -1 & 0 & 3\\ \hline \end{tabular}\begin{gnuplot}[terminal=pdf] set nokey; set grid; set border 0; set xtics axis 1; set ytics axis 1; set size ratio 2; set size 1,1; set zeroaxis linewidth 2 linetype 1; set yrange [-1:15]; set xrange [-5:1]; plot 1*x**2+2*x+0 \end{gnuplot}
	\end{beamercolorbox}
\end{frame}

\begin{frame}[shrink=19,fragile]
	\begin{beamercolorbox}[rounded=true, left, shadow=true,wd=14.8cm]{que2}
		 Complete this table for $y = x^{2} + 2x$ and draw the graph. \\[0.3cm] \renewcommand{\arraystretch}{1.2}\newcolumntype{C}{>{\centering\arraybackslash} m{1cm} }\begin{tabular}{|C|C|C|C|C|C|C|C|} \hline x & -3 & -2 & -1 & 0 & 1 & 2 & 3 \\ \hline y & 3 &  &  & 0 & 3 &  & 15\\ \hline \end{tabular}\\[0.3cm]
	\end{beamercolorbox}
\end{frame}
\begin{frame}[shrink=19,fragile]
	\begin{beamercolorbox}[rounded=true, left, shadow=true,wd=14.8cm]{que2}
		\renewcommand{\arraystretch}{1.2}\newcolumntype{C}{>{\centering\arraybackslash} m{1cm} }\begin{tabular}{|C|C|C|C|C|C|C|C|} \hline x & -3 & -2 & -1 & 0 & 1 & 2 & 3 \\ \hline y & 3 & 0 & -1 & 0 & 3 & 8 & 15\\ \hline \end{tabular}\begin{gnuplot}[terminal=pdf] set nokey; set grid; set border 0; set xtics axis 1; set ytics axis 1; set size ratio 2; set size 1,1; set zeroaxis linewidth 2 linetype 1; set yrange [-1:15]; set xrange [-3:3]; plot 1*x**2+2*x+0 \end{gnuplot}
	\end{beamercolorbox}
\end{frame}

\begin{frame}[shrink=19,fragile]
	\begin{beamercolorbox}[rounded=true, left, shadow=true,wd=14.8cm]{que1}
		 Complete this table for $y = x^{2} + x$ and draw the graph. \\[0.3cm] \renewcommand{\arraystretch}{1.2}\newcolumntype{C}{>{\centering\arraybackslash} m{1cm} }\begin{tabular}{|C|C|C|C|C|C|C|C|} \hline x & -3 & -2 & -1 & 0 & 1 & 2 & 3 \\ \hline y & 6 &  & 0 &  & 2 & 6 & \\ \hline \end{tabular}\\[0.3cm]
	\end{beamercolorbox}
\end{frame}
\begin{frame}[shrink=19,fragile]
	\begin{beamercolorbox}[rounded=true, left, shadow=true,wd=14.8cm]{que1}
		\renewcommand{\arraystretch}{1.2}\newcolumntype{C}{>{\centering\arraybackslash} m{1cm} }\begin{tabular}{|C|C|C|C|C|C|C|C|} \hline x & -3 & -2 & -1 & 0 & 1 & 2 & 3 \\ \hline y & 6 & 2 & 0 & 0 & 2 & 6 & 12\\ \hline \end{tabular}\begin{gnuplot}[terminal=pdf] set nokey; set grid; set border 0; set xtics axis 1; set ytics axis 1; set size ratio 2; set size 1,1; set zeroaxis linewidth 2 linetype 1; set yrange [-1:12]; set xrange [-3:3]; plot 1*x**2+1*x+0 \end{gnuplot}
	\end{beamercolorbox}
\end{frame}

\begin{frame}[shrink=19,fragile]
	\begin{beamercolorbox}[rounded=true, left, shadow=true,wd=14.8cm]{que2}
		 Complete this table for $y = x^{2} + 2x$ and draw the graph. \\[0.3cm] \renewcommand{\arraystretch}{1.2}\newcolumntype{C}{>{\centering\arraybackslash} m{1cm} }\begin{tabular}{|C|C|C|C|C|C|C|C|} \hline x & -5 & -4 & -3 & -2 & -1 & 0 & 1 \\ \hline y & 15 &  & 3 &  & -1 & 0 & \\ \hline \end{tabular}\\[0.3cm]
	\end{beamercolorbox}
\end{frame}
\begin{frame}[shrink=19,fragile]
	\begin{beamercolorbox}[rounded=true, left, shadow=true,wd=14.8cm]{que2}
		\renewcommand{\arraystretch}{1.2}\newcolumntype{C}{>{\centering\arraybackslash} m{1cm} }\begin{tabular}{|C|C|C|C|C|C|C|C|} \hline x & -5 & -4 & -3 & -2 & -1 & 0 & 1 \\ \hline y & 15 & 8 & 3 & 0 & -1 & 0 & 3\\ \hline \end{tabular}\begin{gnuplot}[terminal=pdf] set nokey; set grid; set border 0; set xtics axis 1; set ytics axis 1; set size ratio 2; set size 1,1; set zeroaxis linewidth 2 linetype 1; set yrange [-1:15]; set xrange [-5:1]; plot 1*x**2+2*x+0 \end{gnuplot}
	\end{beamercolorbox}
\end{frame}

\begin{frame}[shrink=19,fragile]
	\begin{beamercolorbox}[rounded=true, left, shadow=true,wd=14.8cm]{que1}
		 Complete this table for $y = x^{2} - 4x$ and draw the graph. \\[0.3cm] \renewcommand{\arraystretch}{1.2}\newcolumntype{C}{>{\centering\arraybackslash} m{1cm} }\begin{tabular}{|C|C|C|C|C|C|C|C|} \hline x & -2 & -1 & 0 & 1 & 2 & 3 & 4 \\ \hline y & 12 &  & 0 & -3 &  & -3 & \\ \hline \end{tabular}\\[0.3cm]
	\end{beamercolorbox}
\end{frame}
\begin{frame}[shrink=19,fragile]
	\begin{beamercolorbox}[rounded=true, left, shadow=true,wd=14.8cm]{que1}
		\renewcommand{\arraystretch}{1.2}\newcolumntype{C}{>{\centering\arraybackslash} m{1cm} }\begin{tabular}{|C|C|C|C|C|C|C|C|} \hline x & -2 & -1 & 0 & 1 & 2 & 3 & 4 \\ \hline y & 12 & 5 & 0 & -3 & -4 & -3 & 0\\ \hline \end{tabular}\begin{gnuplot}[terminal=pdf] set nokey; set grid; set border 0; set xtics axis 1; set ytics axis 1; set size ratio 2; set size 1,1; set zeroaxis linewidth 2 linetype 1; set yrange [-4:12]; set xrange [-2:4]; plot 1*x**2+-4*x+0 \end{gnuplot}
	\end{beamercolorbox}
\end{frame}

\begin{frame}[shrink=19,fragile]
	\begin{beamercolorbox}[rounded=true, left, shadow=true,wd=14.8cm]{que2}
		 Complete this table for $y = x^{2} - x$ and draw the graph. \\[0.3cm] \renewcommand{\arraystretch}{1.2}\newcolumntype{C}{>{\centering\arraybackslash} m{1cm} }\begin{tabular}{|C|C|C|C|C|C|C|C|} \hline x & -3 & -2 & -1 & 0 & 1 & 2 & 3 \\ \hline y & 12 &  & 2 &  & 0 &  & 6\\ \hline \end{tabular}\\[0.3cm]
	\end{beamercolorbox}
\end{frame}
\begin{frame}[shrink=19,fragile]
	\begin{beamercolorbox}[rounded=true, left, shadow=true,wd=14.8cm]{que2}
		\renewcommand{\arraystretch}{1.2}\newcolumntype{C}{>{\centering\arraybackslash} m{1cm} }\begin{tabular}{|C|C|C|C|C|C|C|C|} \hline x & -3 & -2 & -1 & 0 & 1 & 2 & 3 \\ \hline y & 12 & 6 & 2 & 0 & 0 & 2 & 6\\ \hline \end{tabular}\begin{gnuplot}[terminal=pdf] set nokey; set grid; set border 0; set xtics axis 1; set ytics axis 1; set size ratio 2; set size 1,1; set zeroaxis linewidth 2 linetype 1; set yrange [-1:12]; set xrange [-3:3]; plot 1*x**2+-1*x+0 \end{gnuplot}
	\end{beamercolorbox}
\end{frame}

\begin{frame}[shrink=19,fragile]
	\begin{beamercolorbox}[rounded=true, left, shadow=true,wd=14.8cm]{que1}
		 Complete this table for $y = x^{2} - 4x$ and draw the graph. \\[0.3cm] \renewcommand{\arraystretch}{1.2}\newcolumntype{C}{>{\centering\arraybackslash} m{1cm} }\begin{tabular}{|C|C|C|C|C|C|C|C|} \hline x & 0 & 1 & 2 & 3 & 4 & 5 & 6 \\ \hline y & 0 &  &  & -3 & 0 &  & 12\\ \hline \end{tabular}\\[0.3cm]
	\end{beamercolorbox}
\end{frame}
\begin{frame}[shrink=19,fragile]
	\begin{beamercolorbox}[rounded=true, left, shadow=true,wd=14.8cm]{que1}
		\renewcommand{\arraystretch}{1.2}\newcolumntype{C}{>{\centering\arraybackslash} m{1cm} }\begin{tabular}{|C|C|C|C|C|C|C|C|} \hline x & 0 & 1 & 2 & 3 & 4 & 5 & 6 \\ \hline y & 0 & -3 & -4 & -3 & 0 & 5 & 12\\ \hline \end{tabular}\begin{gnuplot}[terminal=pdf] set nokey; set grid; set border 0; set xtics axis 1; set ytics axis 1; set size ratio 2; set size 1,1; set zeroaxis linewidth 2 linetype 1; set yrange [-4:12]; set xrange [0:6]; plot 1*x**2+-4*x+0 \end{gnuplot}
	\end{beamercolorbox}
\end{frame}

\begin{frame}[shrink=19,fragile]
	\begin{beamercolorbox}[rounded=true, left, shadow=true,wd=14.8cm]{que2}
		 Complete this table for $y = x^{2} - 3x$ and draw the graph. \\[0.3cm] \renewcommand{\arraystretch}{1.2}\newcolumntype{C}{>{\centering\arraybackslash} m{1cm} }\begin{tabular}{|C|C|C|C|C|C|C|C|} \hline x & -2 & -1 & 0 & 1 & 2 & 3 & 4 \\ \hline y & 10 &  &  & -2 & -2 &  & 4\\ \hline \end{tabular}\\[0.3cm]
	\end{beamercolorbox}
\end{frame}
\begin{frame}[shrink=19,fragile]
	\begin{beamercolorbox}[rounded=true, left, shadow=true,wd=14.8cm]{que2}
		\renewcommand{\arraystretch}{1.2}\newcolumntype{C}{>{\centering\arraybackslash} m{1cm} }\begin{tabular}{|C|C|C|C|C|C|C|C|} \hline x & -2 & -1 & 0 & 1 & 2 & 3 & 4 \\ \hline y & 10 & 4 & 0 & -2 & -2 & 0 & 4\\ \hline \end{tabular}\begin{gnuplot}[terminal=pdf] set nokey; set grid; set border 0; set xtics axis 1; set ytics axis 1; set size ratio 2; set size 1,1; set zeroaxis linewidth 2 linetype 1; set yrange [-3:10]; set xrange [-2:4]; plot 1*x**2+-3*x+0 \end{gnuplot}
	\end{beamercolorbox}
\end{frame}

\begin{frame}[shrink=19,fragile]
	\begin{beamercolorbox}[rounded=true, left, shadow=true,wd=14.8cm]{que1}
		 Complete this table for $y = x^{2} - 4x$ and draw the graph. \\[0.3cm] \renewcommand{\arraystretch}{1.2}\newcolumntype{C}{>{\centering\arraybackslash} m{1cm} }\begin{tabular}{|C|C|C|C|C|C|C|C|} \hline x & -2 & -1 & 0 & 1 & 2 & 3 & 4 \\ \hline y &  & 5 & 0 &  & -4 &  & 0\\ \hline \end{tabular}\\[0.3cm]
	\end{beamercolorbox}
\end{frame}
\begin{frame}[shrink=19,fragile]
	\begin{beamercolorbox}[rounded=true, left, shadow=true,wd=14.8cm]{que1}
		\renewcommand{\arraystretch}{1.2}\newcolumntype{C}{>{\centering\arraybackslash} m{1cm} }\begin{tabular}{|C|C|C|C|C|C|C|C|} \hline x & -2 & -1 & 0 & 1 & 2 & 3 & 4 \\ \hline y & 12 & 5 & 0 & -3 & -4 & -3 & 0\\ \hline \end{tabular}\begin{gnuplot}[terminal=pdf] set nokey; set grid; set border 0; set xtics axis 1; set ytics axis 1; set size ratio 2; set size 1,1; set zeroaxis linewidth 2 linetype 1; set yrange [-4:12]; set xrange [-2:4]; plot 1*x**2+-4*x+0 \end{gnuplot}
	\end{beamercolorbox}
\end{frame}

\begin{frame}[shrink=19,fragile]
	\begin{beamercolorbox}[rounded=true, left, shadow=true,wd=14.8cm]{que2}
		 Complete this table for $y = x^{2} - 5x$ and draw the graph. \\[0.3cm] \renewcommand{\arraystretch}{1.2}\newcolumntype{C}{>{\centering\arraybackslash} m{1cm} }\begin{tabular}{|C|C|C|C|C|C|C|C|} \hline x & 0 & 1 & 2 & 3 & 4 & 5 & 6 \\ \hline y & 0 &  & -6 & -6 &  & 0 & \\ \hline \end{tabular}\\[0.3cm]
	\end{beamercolorbox}
\end{frame}
\begin{frame}[shrink=19,fragile]
	\begin{beamercolorbox}[rounded=true, left, shadow=true,wd=14.8cm]{que2}
		\renewcommand{\arraystretch}{1.2}\newcolumntype{C}{>{\centering\arraybackslash} m{1cm} }\begin{tabular}{|C|C|C|C|C|C|C|C|} \hline x & 0 & 1 & 2 & 3 & 4 & 5 & 6 \\ \hline y & 0 & -4 & -6 & -6 & -4 & 0 & 6\\ \hline \end{tabular}\begin{gnuplot}[terminal=pdf] set nokey; set grid; set border 0; set xtics axis 1; set ytics axis 1; set size ratio 2; set size 1,1; set zeroaxis linewidth 2 linetype 1; set yrange [-7:6]; set xrange [0:6]; plot 1*x**2+-5*x+0 \end{gnuplot}
	\end{beamercolorbox}
\end{frame}

\begin{frame}[shrink=19,fragile]
	\begin{beamercolorbox}[rounded=true, left, shadow=true,wd=14.8cm]{que1}
		 Complete this table for $y = x^{2} - 4x$ and draw the graph. \\[0.3cm] \renewcommand{\arraystretch}{1.2}\newcolumntype{C}{>{\centering\arraybackslash} m{1cm} }\begin{tabular}{|C|C|C|C|C|C|C|C|} \hline x & -2 & -1 & 0 & 1 & 2 & 3 & 4 \\ \hline y & 12 &  &  & -3 & -4 & -3 & \\ \hline \end{tabular}\\[0.3cm]
	\end{beamercolorbox}
\end{frame}
\begin{frame}[shrink=19,fragile]
	\begin{beamercolorbox}[rounded=true, left, shadow=true,wd=14.8cm]{que1}
		\renewcommand{\arraystretch}{1.2}\newcolumntype{C}{>{\centering\arraybackslash} m{1cm} }\begin{tabular}{|C|C|C|C|C|C|C|C|} \hline x & -2 & -1 & 0 & 1 & 2 & 3 & 4 \\ \hline y & 12 & 5 & 0 & -3 & -4 & -3 & 0\\ \hline \end{tabular}\begin{gnuplot}[terminal=pdf] set nokey; set grid; set border 0; set xtics axis 1; set ytics axis 1; set size ratio 2; set size 1,1; set zeroaxis linewidth 2 linetype 1; set yrange [-4:12]; set xrange [-2:4]; plot 1*x**2+-4*x+0 \end{gnuplot}
	\end{beamercolorbox}
\end{frame}

\begin{frame}[shrink=19,fragile]
	\begin{beamercolorbox}[rounded=true, left, shadow=true,wd=14.8cm]{que2}
		 Complete this table for $y = x^{2} - 4x$ and draw the graph. \\[0.3cm] \renewcommand{\arraystretch}{1.2}\newcolumntype{C}{>{\centering\arraybackslash} m{1cm} }\begin{tabular}{|C|C|C|C|C|C|C|C|} \hline x & -2 & -1 & 0 & 1 & 2 & 3 & 4 \\ \hline y & 12 &  & 0 &  & -4 &  & 0\\ \hline \end{tabular}\\[0.3cm]
	\end{beamercolorbox}
\end{frame}
\begin{frame}[shrink=19,fragile]
	\begin{beamercolorbox}[rounded=true, left, shadow=true,wd=14.8cm]{que2}
		\renewcommand{\arraystretch}{1.2}\newcolumntype{C}{>{\centering\arraybackslash} m{1cm} }\begin{tabular}{|C|C|C|C|C|C|C|C|} \hline x & -2 & -1 & 0 & 1 & 2 & 3 & 4 \\ \hline y & 12 & 5 & 0 & -3 & -4 & -3 & 0\\ \hline \end{tabular}\begin{gnuplot}[terminal=pdf] set nokey; set grid; set border 0; set xtics axis 1; set ytics axis 1; set size ratio 2; set size 1,1; set zeroaxis linewidth 2 linetype 1; set yrange [-4:12]; set xrange [-2:4]; plot 1*x**2+-4*x+0 \end{gnuplot}
	\end{beamercolorbox}
\end{frame}

\begin{frame}[shrink=19,fragile]
	\begin{beamercolorbox}[rounded=true, left, shadow=true,wd=14.8cm]{que1}
		 Complete this table for $y = x^{2} + 5x$ and draw the graph. \\[0.3cm] \renewcommand{\arraystretch}{1.2}\newcolumntype{C}{>{\centering\arraybackslash} m{1cm} }\begin{tabular}{|C|C|C|C|C|C|C|C|} \hline x & -6 & -5 & -4 & -3 & -2 & -1 & 0 \\ \hline y &  & 0 & -4 &  & -6 &  & 0\\ \hline \end{tabular}\\[0.3cm]
	\end{beamercolorbox}
\end{frame}
\begin{frame}[shrink=19,fragile]
	\begin{beamercolorbox}[rounded=true, left, shadow=true,wd=14.8cm]{que1}
		\renewcommand{\arraystretch}{1.2}\newcolumntype{C}{>{\centering\arraybackslash} m{1cm} }\begin{tabular}{|C|C|C|C|C|C|C|C|} \hline x & -6 & -5 & -4 & -3 & -2 & -1 & 0 \\ \hline y & 6 & 0 & -4 & -6 & -6 & -4 & 0\\ \hline \end{tabular}\begin{gnuplot}[terminal=pdf] set nokey; set grid; set border 0; set xtics axis 1; set ytics axis 1; set size ratio 2; set size 1,1; set zeroaxis linewidth 2 linetype 1; set yrange [-7:6]; set xrange [-6:0]; plot 1*x**2+5*x+0 \end{gnuplot}
	\end{beamercolorbox}
\end{frame}

\begin{frame}[shrink=19,fragile]
	\begin{beamercolorbox}[rounded=true, left, shadow=true,wd=14.8cm]{que2}
		 Complete this table for $y = x^{2} + 3x$ and draw the graph. \\[0.3cm] \renewcommand{\arraystretch}{1.2}\newcolumntype{C}{>{\centering\arraybackslash} m{1cm} }\begin{tabular}{|C|C|C|C|C|C|C|C|} \hline x & -4 & -3 & -2 & -1 & 0 & 1 & 2 \\ \hline y & 4 &  & -2 & -2 &  &  & 10\\ \hline \end{tabular}\\[0.3cm]
	\end{beamercolorbox}
\end{frame}
\begin{frame}[shrink=19,fragile]
	\begin{beamercolorbox}[rounded=true, left, shadow=true,wd=14.8cm]{que2}
 		\renewcommand{\arraystretch}{1.2}\newcolumntype{C}{>{\centering\arraybackslash} m{1cm} }\begin{tabular}{|C|C|C|C|C|C|C|C|} \hline x & -4 & -3 & -2 & -1 & 0 & 1 & 2 \\ \hline y & 4 & 0 & -2 & -2 & 0 & 4 & 10\\ \hline \end{tabular}\begin{gnuplot}[terminal=pdf] set nokey; set grid; set border 0; set xtics axis 1; set ytics axis 1; set size ratio 2; set size 1,1; set zeroaxis linewidth 2 linetype 1; set yrange [-3:10]; set xrange [-4:2]; plot 1*x**2+3*x+0 \end{gnuplot}
	\end{beamercolorbox}
\end{frame}

\begin{frame}[shrink=19,fragile]
	\begin{beamercolorbox}[rounded=true, left, shadow=true,wd=14.8cm]{que1}
		 Complete this table for $y = x^{2} + 2x$ and draw the graph. \\[0.3cm] \renewcommand{\arraystretch}{1.2}\newcolumntype{C}{>{\centering\arraybackslash} m{1cm} }\begin{tabular}{|C|C|C|C|C|C|C|C|} \hline x & -5 & -4 & -3 & -2 & -1 & 0 & 1 \\ \hline y & 15 &  &  & 0 & -1 &  & 3\\ \hline \end{tabular}\\[0.3cm]
	\end{beamercolorbox}
\end{frame}
\begin{frame}[shrink=19,fragile]
	\begin{beamercolorbox}[rounded=true, left, shadow=true,wd=14.8cm]{que1}
		\renewcommand{\arraystretch}{1.2}\newcolumntype{C}{>{\centering\arraybackslash} m{1cm} }\begin{tabular}{|C|C|C|C|C|C|C|C|} \hline x & -5 & -4 & -3 & -2 & -1 & 0 & 1 \\ \hline y & 15 & 8 & 3 & 0 & -1 & 0 & 3\\ \hline \end{tabular}\begin{gnuplot}[terminal=pdf] set nokey; set grid; set border 0; set xtics axis 1; set ytics axis 1; set size ratio 2; set size 1,1; set zeroaxis linewidth 2 linetype 1; set yrange [-1:15]; set xrange [-5:1]; plot 1*x**2+2*x+0 \end{gnuplot}
	\end{beamercolorbox}
\end{frame}

\begin{frame}[shrink=19,fragile]
	\begin{beamercolorbox}[rounded=true, left, shadow=true,wd=14.8cm]{que2}
		 Complete this table for $y = x^{2} - 2x$ and draw the graph. \\[0.3cm] \renewcommand{\arraystretch}{1.2}\newcolumntype{C}{>{\centering\arraybackslash} m{1cm} }\begin{tabular}{|C|C|C|C|C|C|C|C|} \hline x & -1 & 0 & 1 & 2 & 3 & 4 & 5 \\ \hline y & 3 &  &  & 0 & 3 &  & 15\\ \hline \end{tabular}\\[0.3cm]
	\end{beamercolorbox}
\end{frame}
\begin{frame}[shrink=19,fragile]
	\begin{beamercolorbox}[rounded=true, left, shadow=true,wd=14.8cm]{que2}
		\renewcommand{\arraystretch}{1.2}\newcolumntype{C}{>{\centering\arraybackslash} m{1cm} }\begin{tabular}{|C|C|C|C|C|C|C|C|} \hline x & -1 & 0 & 1 & 2 & 3 & 4 & 5 \\ \hline y & 3 & 0 & -1 & 0 & 3 & 8 & 15\\ \hline \end{tabular}\begin{gnuplot}[terminal=pdf] set nokey; set grid; set border 0; set xtics axis 1; set ytics axis 1; set size ratio 2; set size 1,1; set zeroaxis linewidth 2 linetype 1; set yrange [-1:15]; set xrange [-1:5]; plot 1*x**2+-2*x+0 \end{gnuplot}
	\end{beamercolorbox}
\end{frame}

\begin{frame}[shrink=19,fragile]
	\begin{beamercolorbox}[rounded=true, left, shadow=true,wd=14.8cm]{que1}
		 Complete this table for $y = x^{2} - x$ and draw the graph. \\[0.3cm] \renewcommand{\arraystretch}{1.2}\newcolumntype{C}{>{\centering\arraybackslash} m{1cm} }\begin{tabular}{|C|C|C|C|C|C|C|C|} \hline x & -2 & -1 & 0 & 1 & 2 & 3 & 4 \\ \hline y & 6 &  & 0 &  & 2 &  & 12\\ \hline \end{tabular}\\[0.3cm]
	\end{beamercolorbox}
\end{frame}
\begin{frame}[shrink=19,fragile]
	\begin{beamercolorbox}[rounded=true, left, shadow=true,wd=14.8cm]{que1}
		\renewcommand{\arraystretch}{1.2}\newcolumntype{C}{>{\centering\arraybackslash} m{1cm} }\begin{tabular}{|C|C|C|C|C|C|C|C|} \hline x & -2 & -1 & 0 & 1 & 2 & 3 & 4 \\ \hline y & 6 & 2 & 0 & 0 & 2 & 6 & 12\\ \hline \end{tabular}\begin{gnuplot}[terminal=pdf] set nokey; set grid; set border 0; set xtics axis 1; set ytics axis 1; set size ratio 2; set size 1,1; set zeroaxis linewidth 2 linetype 1; set yrange [-1:12]; set xrange [-2:4]; plot 1*x**2+-1*x+0 \end{gnuplot}
	\end{beamercolorbox}
\end{frame}

\begin{frame}[shrink=19,fragile]
	\begin{beamercolorbox}[rounded=true, left, shadow=true,wd=14.8cm]{que2}
		 Complete this table for $y = x^{2} + 2x$ and draw the graph. \\[0.3cm] \renewcommand{\arraystretch}{1.2}\newcolumntype{C}{>{\centering\arraybackslash} m{1cm} }\begin{tabular}{|C|C|C|C|C|C|C|C|} \hline x & -3 & -2 & -1 & 0 & 1 & 2 & 3 \\ \hline y &  & 0 &  & 0 & 3 &  & 15\\ \hline \end{tabular}\\[0.3cm]
	\end{beamercolorbox}
\end{frame}
\begin{frame}[shrink=19,fragile]
	\begin{beamercolorbox}[rounded=true, left, shadow=true,wd=14.8cm]{que2}
		\renewcommand{\arraystretch}{1.2}\newcolumntype{C}{>{\centering\arraybackslash} m{1cm} }\begin{tabular}{|C|C|C|C|C|C|C|C|} \hline x & -3 & -2 & -1 & 0 & 1 & 2 & 3 \\ \hline y & 3 & 0 & -1 & 0 & 3 & 8 & 15\\ \hline \end{tabular}\begin{gnuplot}[terminal=pdf] set nokey; set grid; set border 0; set xtics axis 1; set ytics axis 1; set size ratio 2; set size 1,1; set zeroaxis linewidth 2 linetype 1; set yrange [-1:15]; set xrange [-3:3]; plot 1*x**2+2*x+0 \end{gnuplot}
	\end{beamercolorbox}
\end{frame}

\begin{frame}[shrink=19,fragile]
	\begin{beamercolorbox}[rounded=true, left, shadow=true,wd=14.8cm]{que1}
		 Complete this table for $y = x^{2} - 5x$ and draw the graph. \\[0.3cm] \renewcommand{\arraystretch}{1.2}\newcolumntype{C}{>{\centering\arraybackslash} m{1cm} }\begin{tabular}{|C|C|C|C|C|C|C|C|} \hline x & -1 & 0 & 1 & 2 & 3 & 4 & 5 \\ \hline y &  & 0 & -4 & -6 &  & -4 & \\ \hline \end{tabular}\\[0.3cm]
	\end{beamercolorbox}
\end{frame}
\begin{frame}[shrink=19,fragile]
	\begin{beamercolorbox}[rounded=true, left, shadow=true,wd=14.8cm]{que1}
		\renewcommand{\arraystretch}{1.2}\newcolumntype{C}{>{\centering\arraybackslash} m{1cm} }\begin{tabular}{|C|C|C|C|C|C|C|C|} \hline x & -1 & 0 & 1 & 2 & 3 & 4 & 5 \\ \hline y & 6 & 0 & -4 & -6 & -6 & -4 & 0\\ \hline \end{tabular}\begin{gnuplot}[terminal=pdf] set nokey; set grid; set border 0; set xtics axis 1; set ytics axis 1; set size ratio 2; set size 1,1; set zeroaxis linewidth 2 linetype 1; set yrange [-7:6]; set xrange [-1:5]; plot 1*x**2+-5*x+0 \end{gnuplot}
	\end{beamercolorbox}
\end{frame}

\begin{frame}[shrink=19,fragile]
	\begin{beamercolorbox}[rounded=true, left, shadow=true,wd=14.8cm]{que2}
		 Complete this table for $y = x^{2} + x$ and draw the graph. \\[0.3cm] \renewcommand{\arraystretch}{1.2}\newcolumntype{C}{>{\centering\arraybackslash} m{1cm} }\begin{tabular}{|C|C|C|C|C|C|C|C|} \hline x & -4 & -3 & -2 & -1 & 0 & 1 & 2 \\ \hline y &  & 6 & 2 & 0 &  &  & 6\\ \hline \end{tabular}\\[0.3cm]
	\end{beamercolorbox}
\end{frame}
\begin{frame}[shrink=19,fragile]
	\begin{beamercolorbox}[rounded=true, left, shadow=true,wd=14.8cm]{que2}
		\renewcommand{\arraystretch}{1.2}\newcolumntype{C}{>{\centering\arraybackslash} m{1cm} }\begin{tabular}{|C|C|C|C|C|C|C|C|} \hline x & -4 & -3 & -2 & -1 & 0 & 1 & 2 \\ \hline y & 12 & 6 & 2 & 0 & 0 & 2 & 6\\ \hline \end{tabular}\begin{gnuplot}[terminal=pdf] set nokey; set grid; set border 0; set xtics axis 1; set ytics axis 1; set size ratio 2; set size 1,1; set zeroaxis linewidth 2 linetype 1; set yrange [-1:12]; set xrange [-4:2]; plot 1*x**2+1*x+0 \end{gnuplot}
	\end{beamercolorbox}
\end{frame}

\begin{frame}[shrink=19,fragile]
	\begin{beamercolorbox}[rounded=true, left, shadow=true,wd=14.8cm]{que1}
		 Complete this table for $y = x^{2} - 3x$ and draw the graph. \\[0.3cm] \renewcommand{\arraystretch}{1.2}\newcolumntype{C}{>{\centering\arraybackslash} m{1cm} }\begin{tabular}{|C|C|C|C|C|C|C|C|} \hline x & -1 & 0 & 1 & 2 & 3 & 4 & 5 \\ \hline y &  & 0 &  & -2 & 0 & 4 & \\ \hline \end{tabular}\\[0.3cm]
	\end{beamercolorbox}
\end{frame}
\begin{frame}[shrink=19,fragile]
	\begin{beamercolorbox}[rounded=true, left, shadow=true,wd=14.8cm]{que1}
		\renewcommand{\arraystretch}{1.2}\newcolumntype{C}{>{\centering\arraybackslash} m{1cm} }\begin{tabular}{|C|C|C|C|C|C|C|C|} \hline x & -1 & 0 & 1 & 2 & 3 & 4 & 5 \\ \hline y & 4 & 0 & -2 & -2 & 0 & 4 & 10\\ \hline \end{tabular}\begin{gnuplot}[terminal=pdf] set nokey; set grid; set border 0; set xtics axis 1; set ytics axis 1; set size ratio 2; set size 1,1; set zeroaxis linewidth 2 linetype 1; set yrange [-3:10]; set xrange [-1:5]; plot 1*x**2+-3*x+0 \end{gnuplot}
	\end{beamercolorbox}
\end{frame}

\begin{frame}[shrink=19,fragile]
	\begin{beamercolorbox}[rounded=true, left, shadow=true,wd=14.8cm]{que2}
		 Complete this table for $y = x^{2} + 2x$ and draw the graph. \\[0.3cm] \renewcommand{\arraystretch}{1.2}\newcolumntype{C}{>{\centering\arraybackslash} m{1cm} }\begin{tabular}{|C|C|C|C|C|C|C|C|} \hline x & -5 & -4 & -3 & -2 & -1 & 0 & 1 \\ \hline y & 15 &  & 3 & 0 &  & 0 & \\ \hline \end{tabular}\\[0.3cm]
	\end{beamercolorbox}
\end{frame}
\begin{frame}[shrink=19,fragile]
	\begin{beamercolorbox}[rounded=true, left, shadow=true,wd=14.8cm]{que2}
		\renewcommand{\arraystretch}{1.2}\newcolumntype{C}{>{\centering\arraybackslash} m{1cm} }\begin{tabular}{|C|C|C|C|C|C|C|C|} \hline x & -5 & -4 & -3 & -2 & -1 & 0 & 1 \\ \hline y & 15 & 8 & 3 & 0 & -1 & 0 & 3\\ \hline \end{tabular}\begin{gnuplot}[terminal=pdf] set nokey; set grid; set border 0; set xtics axis 1; set ytics axis 1; set size ratio 2; set size 1,1; set zeroaxis linewidth 2 linetype 1; set yrange [-1:15]; set xrange [-5:1]; plot 1*x**2+2*x+0 \end{gnuplot}
	\end{beamercolorbox}
\end{frame}

\begin{frame}[shrink=19,fragile]
	\begin{beamercolorbox}[rounded=true, left, shadow=true,wd=14.8cm]{que1}
		 Complete this table for $y = x^{2} - 3x$ and draw the graph. \\[0.3cm] \renewcommand{\arraystretch}{1.2}\newcolumntype{C}{>{\centering\arraybackslash} m{1cm} }\begin{tabular}{|C|C|C|C|C|C|C|C|} \hline x & -2 & -1 & 0 & 1 & 2 & 3 & 4 \\ \hline y &  & 4 & 0 &  & -2 &  & 4\\ \hline \end{tabular}\\[0.3cm]
	\end{beamercolorbox}
\end{frame}
\begin{frame}[shrink=19,fragile]
	\begin{beamercolorbox}[rounded=true, left, shadow=true,wd=14.8cm]{que1}
		\renewcommand{\arraystretch}{1.2}\newcolumntype{C}{>{\centering\arraybackslash} m{1cm} }\begin{tabular}{|C|C|C|C|C|C|C|C|} \hline x & -2 & -1 & 0 & 1 & 2 & 3 & 4 \\ \hline y & 10 & 4 & 0 & -2 & -2 & 0 & 4\\ \hline \end{tabular}\begin{gnuplot}[terminal=pdf] set nokey; set grid; set border 0; set xtics axis 1; set ytics axis 1; set size ratio 2; set size 1,1; set zeroaxis linewidth 2 linetype 1; set yrange [-3:10]; set xrange [-2:4]; plot 1*x**2+-3*x+0 \end{gnuplot}
	\end{beamercolorbox}
\end{frame}

\begin{frame}[shrink=19,fragile]
	\begin{beamercolorbox}[rounded=true, left, shadow=true,wd=14.8cm]{que2}
		 Complete this table for $y = x^{2} - 3x$ and draw the graph. \\[0.3cm] \renewcommand{\arraystretch}{1.2}\newcolumntype{C}{>{\centering\arraybackslash} m{1cm} }\begin{tabular}{|C|C|C|C|C|C|C|C|} \hline x & -1 & 0 & 1 & 2 & 3 & 4 & 5 \\ \hline y & 4 &  & -2 & -2 &  & 4 & \\ \hline \end{tabular}\\[0.3cm]
	\end{beamercolorbox}
\end{frame}
\begin{frame}[shrink=19,fragile]
	\begin{beamercolorbox}[rounded=true, left, shadow=true,wd=14.8cm]{que2}
		\renewcommand{\arraystretch}{1.2}\newcolumntype{C}{>{\centering\arraybackslash} m{1cm} }\begin{tabular}{|C|C|C|C|C|C|C|C|} \hline x & -1 & 0 & 1 & 2 & 3 & 4 & 5 \\ \hline y & 4 & 0 & -2 & -2 & 0 & 4 & 10\\ \hline \end{tabular}\begin{gnuplot}[terminal=pdf] set nokey; set grid; set border 0; set xtics axis 1; set ytics axis 1; set size ratio 2; set size 1,1; set zeroaxis linewidth 2 linetype 1; set yrange [-3:10]; set xrange [-1:5]; plot 1*x**2+-3*x+0 \end{gnuplot}
	\end{beamercolorbox}
\end{frame}

\begin{frame}[shrink=19,fragile]
	\begin{beamercolorbox}[rounded=true, left, shadow=true,wd=14.8cm]{que1}
		 Complete this table for $y = x^{2} - 3x$ and draw the graph. \\[0.3cm] \renewcommand{\arraystretch}{1.2}\newcolumntype{C}{>{\centering\arraybackslash} m{1cm} }\begin{tabular}{|C|C|C|C|C|C|C|C|} \hline x & -2 & -1 & 0 & 1 & 2 & 3 & 4 \\ \hline y & 10 &  &  & -2 & -2 & 0 & \\ \hline \end{tabular}\\[0.3cm]
	\end{beamercolorbox}
\end{frame}
\begin{frame}[shrink=19,fragile]
	\begin{beamercolorbox}[rounded=true, left, shadow=true,wd=14.8cm]{que1}
		\renewcommand{\arraystretch}{1.2}\newcolumntype{C}{>{\centering\arraybackslash} m{1cm} }\begin{tabular}{|C|C|C|C|C|C|C|C|} \hline x & -2 & -1 & 0 & 1 & 2 & 3 & 4 \\ \hline y & 10 & 4 & 0 & -2 & -2 & 0 & 4\\ \hline \end{tabular}\begin{gnuplot}[terminal=pdf] set nokey; set grid; set border 0; set xtics axis 1; set ytics axis 1; set size ratio 2; set size 1,1; set zeroaxis linewidth 2 linetype 1; set yrange [-3:10]; set xrange [-2:4]; plot 1*x**2+-3*x+0 \end{gnuplot}
	\end{beamercolorbox}
\end{frame}

\begin{frame}[shrink=19,fragile]
	\begin{beamercolorbox}[rounded=true, left, shadow=true,wd=14.8cm]{que2}
		 Complete this table for $y = x^{2} - 3x$ and draw the graph. \\[0.3cm] \renewcommand{\arraystretch}{1.2}\newcolumntype{C}{>{\centering\arraybackslash} m{1cm} }\begin{tabular}{|C|C|C|C|C|C|C|C|} \hline x & -2 & -1 & 0 & 1 & 2 & 3 & 4 \\ \hline y & 10 &  &  & -2 & -2 &  & 4\\ \hline \end{tabular}\\[0.3cm]
	\end{beamercolorbox}
\end{frame}
\begin{frame}[shrink=19,fragile]
	\begin{beamercolorbox}[rounded=true, left, shadow=true,wd=14.8cm]{que2}
		\renewcommand{\arraystretch}{1.2}\newcolumntype{C}{>{\centering\arraybackslash} m{1cm} }\begin{tabular}{|C|C|C|C|C|C|C|C|} \hline x & -2 & -1 & 0 & 1 & 2 & 3 & 4 \\ \hline y & 10 & 4 & 0 & -2 & -2 & 0 & 4\\ \hline \end{tabular}\begin{gnuplot}[terminal=pdf] set nokey; set grid; set border 0; set xtics axis 1; set ytics axis 1; set size ratio 2; set size 1,1; set zeroaxis linewidth 2 linetype 1; set yrange [-3:10]; set xrange [-2:4]; plot 1*x**2+-3*x+0 \end{gnuplot}
	\end{beamercolorbox}
\end{frame}

\end{document}
